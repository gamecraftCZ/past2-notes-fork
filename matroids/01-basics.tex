\chapter{Basic definitions}

\begin{defn}
	\textbf{Matroid} $\M = (E, \I)$ is for $E$ finite non-empty set and $\I \subseteq 2^E$ (also called as independent sets) satisfying these properties:
	
	\begin{enumerate}[(\text{I}1)]
		\item $\emptyset \in \I$,
		\item $I \in \I \Rightarrow \forall I' \subseteq I : I' \in \I$,
		\item $I_1, I_2 \in \I, \abs{I_1} < \abs{I_2} \Rightarrow \exists e \in I_2 \setminus I_1: I_1 \cup \{e\} \in \I$.
	\end{enumerate}
\end{defn}

\begin{notation}
	For further use and simplification we will sometimes use $I + e$ as a substitution for $I \cup \{e\}$. Similarly also $I - e$ for $I \setminus \{e\}$.
\end{notation}

\begin{example}
	For a given multi-graph $G = (V,F)$ we will set $E = F$ (or in other words $E$ stands for edges and the set). Independent sets $\I$ will be all acyclic subsets of $E$. Easily seen (I1) and (I2) is satisfied. For the third one (I3) it is also quite easily seen, because if we have one larger and smaller non-cycles then we can append one edge from the larger to the smaller.
\end{example}

\begin{example}
	Let $E$ be some elements of a vector space $V$. If $X \subseteq E$ is independent then it is linearly independent in $V$.
\end{example}

\begin{defn}
	Matroid \textbf{isomorphism} for two matroids $\M_i = (E_i, \I_i)$ for $i = 1,2$ is a bijection $f: E_1 \to E_2$ satisfying $\forall X \subseteq E_i : X \in \I_1 \Leftrightarrow f(X) \in \I_2$.
\end{defn}

\section{Circuits}

\begin{defn}
	$X \subseteq E$ is a \textbf{circuit} if $X \notin \I$ and $\forall x \in X : X - x \in \I$. Also we will denote $\C(\M)$ as the set of all circuits of $\M$.
\end{defn}

\begin{lemma}
	Let $\M = (E, \I)$ be a matroid and $\C$ its collection of circuits, then
	
	\begin{enumerate}[(C1)]
		\item $\emptyset \notin \C$,
		\item $\forall C_1, C_2 \in \C : C_1 \subseteq C_2 \Rightarrow C_1 = C_2$ and
		\item $C_1, C_2 \in \C, C_1 \neq C_2, e \in C_1 \cap C_2 \Rightarrow C_3 \subseteq (C_1 \cup C_2) -e , C_e \in \C$.
	\end{enumerate}
\end{lemma}

\begin{proof}
	(C1) and (C2) are easily seen from (I1) and (I2). Now for the third part (C3). So for contradiction let $C_1, C_2, e$ be as mentioned in the first part, but $(C_1 \cup C_2) - e \in \I$. Then $\exists f \in C_2 \setminus C_1 : C_2 - f \in \I$. Now find $I \in \I$ max s.t. $C_2 \setminus \{f\} \subseteq I \subseteq C_1 \cup C_2$. If $f \notin I$ then it would contain $C_2$ which is dependent and $\exists g \in C_1 \setminus C_2 : g \notin I$ otherwise it would contain $C_1$ which is dependent. Therefore
	
	$$
	|I| \leq | C_1 \cup C_2 | - 2 < |(C_1 \cup C_2) - e|
	$$
	
	\noindent and now we may use the third axiom (I3) that is $\exists x \in |(C_1 \cup C_2) - e| \setminus I$ s.t. $I + e \in \I$ (this cannot be otherwise $I$ contains the whole $C_2$). Now $I + x$ contradicts the maximality of $I$.
\end{proof}

\begin{claim}
	Lets have $E$ and $\C \subseteq 2^E$ satisfying all (C1), (C2) and (C3). Then set $\I = \{X \subseteq E | \forall C \in \C : C \nsubseteq X\}$ and $\M = (E, \I)$ is a matroid.
\end{claim}

\begin{proof}
	We have to show all properties of matroid. That is (I1) is trivially satisfied and (I2) also trivially holds. For the last (I3) we use a contradiction. For that we have $I_1, I_2 \in \I$, then $\forall e \in I_2 \setminus I_1 : I_1 + e \notin \I$. Let $I_3 \subseteq I_1 \cup I_2$ s.t. $|I_3| > |I_1|$ and $|I_1 \setminus I_3|$ is minimal. If $|I_1 \setminus I_3|$ would be empty then (I3) will hold, therefore assume it is non-empty.
	
	Fix $e \in I_1 \setminus I_3$. Let $I_k = |I_3 - f| +e$ for $(f \in I_3 \setminus I_1)$. This cannot be independent ($\notin \I$) therefore $\exists C_k \subseteq T_k : C_k \in \C$ and $f \notin C_k, e \in C_k$.
	
	$(I_3 \setminus I_1) \cap C_k = \emptyset$ hence $C_k \subseteq T_k \setminus (I_3 \setminus I_1) = (I_1 \cap I_3) + e \subseteq I_1$ this is not possible so it must be non-empty. Then $\exists g \in (I_3 \setminus I_1) \cap C_k \Rightarrow C_k, C_g \in \C, e \in C_k \cap C_g, f \notin C_k, g \notin C_g$ but $(C_k \cup C_g) - e \subseteq I_3$ which is contradiction with (C3).
\end{proof}

\section{Basis}

\begin{defn}
	Let $\M = (E, \I)$ be a matroid. Then $B$ is a \textbf{basis} iff $B \in \I, \forall x \in E \setminus B: B + x \notin \I$.
\end{defn}

\begin{prop}
	Let $B_1, B_2$ be bases of $\M$, then $\abs{B_1} = \abs{B_2}$.
\end{prop}

\begin{proof}
	If $|B_1| < |B_2|$ then by (I3) $\exists x \in B_2 \setminus B_1 : B_1 + x \in \I$.
\end{proof}

\begin{defn}
	Let $\B (\M) = \{B \subseteq E , B \text{ is a basis}\}$ be a collection of basis satisfying
	
	\begin{enumerate}[(B1)]
		\item $\B \neq \emptyset$ and
		\item $B_1, B_2 \in \B, e \in B_1 \setminus B_2 \Rightarrow \exists f \in B_2 \setminus B_1 : |B_1 - e| + f \in \B$.
	\end{enumerate}
\end{defn}

One can see that (B2) can be proven using $I_1 - e =: B_1$ and $I_2 = B_2$.

\begin{prop}
	Let $E \neq \emptyset$ finite set and $\B \subseteq 2^E$ satisfying (B1) and (B2). Let $\I = \{X \subseteq E : \exists B \in \B\ : X \subseteq B\}$ then $\M = (E, \I)$ is a matroid.
\end{prop}

\begin{proof}
	(I1) and (I2) are trivial. For (I3) use the following lemma.
	
	\begin{lemma}
		Let $\B$ be such that it satisfies (B1) and (B2). Then $\forall B_1, B_2 \in \B : |B_1| = |B_2|$.
	\end{lemma}
	
	\begin{proof}
		By contradiction suppose $\abs{B_1} > \abs{B_2}$ with minimal $\abs{B_1 \setminus B_2}$. Then $e \in B_1 \setminus B_2 \Rightarrow \exists f \in B_2 \setminus B_1: (B_1 - e) + f \in \B$ and also $\abs{(B_1 - e) + f} = |B_1|$ which leads to $\abs{((B_1 - e) + f) \setminus B_2} < \abs{B_1 \setminus B_2}$ which is a contradiction with the minimality.
	\end{proof}
\end{proof}

\section{Rank function}

\begin{defn}
	For a matroid $\M = (E, \I)$ define a \textbf{rank function} $\rank : 2^E \to \Z^+_0$, such that $\rank(X) = \max_{I \subseteq X, I \in \I} |I|$ and $\rank (\M) = \rank(E)$.
\end{defn}

\begin{claim}
	Rank function has the following properties:
	
	\begin{enumerate}[(R1)]
		\item $X \subseteq E: 0 \leq \rank (X) \leq |X|$,
		\item $X \subseteq Y \subseteq E \Rightarrow \rank (X) \leq \rank(Y)$ and
		\item $X, Y \subseteq E : \rank(X \cup Y) + \rank(X \cap Y) \leq \rank(X) + \rank(Y)$ (which is called \textbf{submodularity}).
	\end{enumerate}
\end{claim}

\begin{proof}[Proof of the properties]
	While (R1) and (R2) are obvious and now we will show that (R3) also holds. Let $I_1$ be the max independent in $X \cap Y$ and $I_2$ be an extension $I_2 \supseteq I_1$ and max independent in $X \cup Y$. Now $\rank(X \cup Y) + \rank(X \cap Y) = \abs{I_2} + \abs{I_1}$ and also $|I_2 \cap X| \leq \rank(X)$ and $|I_2 \cap Y| \leq \rank(Y)$. We apply simple rule $|A| + |B| = |A \cup B| + |A \cap B|$ and get
	
	$$
	\rank(X) + \rank(Y) \geq |I_2 \cap X| + |I_2 \cap Y| = |I_2| + |I_1| = \rank(X \cup Y) + \rank(X \cap Y)
	$$
\end{proof}

\begin{thm}
	For $E \neq \emptyset$ finite set and $\rank: 2^E \to \Z^+_0$ satisfying (R1), (R2) and (R3). Then $\I = \{X \subseteq E | \rank(X) = |X| \}$ and $\M = (E, \I)$ is a matroid.
	\label{rank-func-thm}
\end{thm}

\begin{lemma}
	For $E \neq \emptyset$ finite set and $\rank: 2^E \to \Z^+_0$ satisfying (R1), (R2) and (R3). It holds that if $X, Y \subseteq E$ $\forall y \in Y: \rank(X) = \rank(X + y)$ then $\rank(X) = \rank(X \cup Y)$.
	\label{rank-func-lemma}
\end{lemma}

\begin{proof}[Proof of lemma \ref{rank-func-lemma}]
	Let $Y \setminus X = \{y_1, y_2, \dots, y_k\}$ and now we will prove it by induction on $k$. For $k = 1$ it obviously holds. For $k \geq 2$ we use the submodularity.
	
	$$
	\begin{aligned}
		&\rank(X) + \rank(X) &= &\rank(X \cup \{y_1, y_2, \dots, y_{k-1}\}) + \rank(X + y_k) &\geq \rank(X \cup \{y_1, y_2, \dots, y_k\}) + \rank(X)\\
		&                    &  & \text{\small{(by induction hypothesis)}}\\
		&\rank(X)            &= &                                                            &\geq \rank(X \cup \{y_1, y_2, \dots, y_k\})\\
		&\rank(X)            &  &                                                            &\geq \rank(X \cup Y)\\
	\end{aligned}
	$$
	
	\noindent For the other inequality we use (R2) and hence we obtain equality.
\end{proof}

\begin{proof}[Proof of theorem \ref{rank-func-thm}]
	\TODO{Finish the proof.}
\end{proof}

\TODO{Insert missing part.}

\section{Uniform matroids}

\begin{defn}
	For $0 \leq r \leq n \neq 0, |E| = n$ and $\I = \{X \subseteq E : |X| \leq r\}$ is \textbf{Uniform matroid} $U_{r,n} = (E, \I)$.
\end{defn}

All the properties should be formally proven, but one can already see that all (I1), (I2) and (I3) are really satisfied. Now we will show us some examples.

\TODO{Add examples.}

\section{Visualization of matroids}

\TODO{Finish this part.}

\section{(Direct) Sum of matroids (also disjoint union)}

\begin{defn}
	We have two matroids $\M_i = (E_i, \I_i)$ for $i= 1,2$, then the (direct) sum $\M_1 \bigoplus \M_2$ is defined as a matroid $\M = (E, \I)$ where $E = E_1 \dot\cup E_2$ and $\I = \{X \subseteq E, X \cap E_i \in \I_i, i=1,2\}$.
\end{defn}

\begin{observ}
	Lets see the basis and circuits:
	
	$$
	\begin{aligned}
		&\B(\M_1 \bigoplus \M_2) = \{B_1 \cup B_2, B_i \in \B_i, i = 1,2\}\\
		&\C(\M_1 \bigoplus \M_2) = \C_1 \cup \C_2\\
		&X \subseteq E : \rank(X) = \rank_1(X \cap E_1) + \rank_2(X \cap E_2)
	\end{aligned}
	$$
\end{observ}