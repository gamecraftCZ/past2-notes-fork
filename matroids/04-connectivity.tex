\chapter{Connectivity}

We have seen some properties that were derived from graph theory. But now this is not the case. That is the connectivity of matroids is not directly translated from graphs, but instead 2-(vertex)-connectivity.

\begin{prop}
	$G$ without isolated vertices, $G$ is 2-connected $\iff$ $\forall e,f \in E$ and $e \neq f$ $\exists C$ circuit s.t. $e,f \in E(C)$.
\end{prop}

\begin{proof}
	This can be easily seen. Consider subdividing $e$ (and $f$) by $w_e$ ($w_f$) which by itself do not affect 2-connectivity, thic constructs $G'$. Therefore $\exists C'$ in $G'$ where $w_e, w_f \in V(C')$.
\end{proof}

\begin{prop}
	The following properties are equivalent.

	\begin{itemize}[]
		\item (C3) $\forall C_1, C_2 \in \C, C_1 \neq C_2, e \in C_1 \cap C_2, \exists C_3 \subseteq (C_1 \cup C_2) -e , C_3 \in \C$
		\item (C3') $\forall C_1, C_2 \in \C, C_1 \neq C_2, f \in C_1 \cap C_2, \exists C_3  : f \in C_3 \subseteq (C_1 \cup C_2) - e$
	\end{itemize}
\end{prop}

\begin{proof}
	For contradiction let $C_1, C_2$ violates (C3') and $|C_1 \cup C_2|$ is minimal. By (C3) $\exists C_3$ where $e,f \notin C_3$ and by (C2) $\exists g \in C_2 \setminus C_1, g \in C_2 \cap C_3$. Since $f \notin C_2$ it implies $f \notin C_2 \cup C_3$ hence $|C_2 \cup C_3| < |C_1 \cup C_2|$. Use (C3') on $C_2, C_3, g \in C_2 \cap C_3, e \in C_2 \setminus C_3$ then $\exists C_4 : e \in C_4 \subseteq (C_2 \cup C_3) - g$. Now $g \notin C_1, C_4$ so $|C_1 \cup C_4| < |C_2 \cup C_3|$ therefore we again use (C3') $C_1, C_4, e \in C_1 \cap C_4, f \in C_1 \setminus C_4, \exists C: f \in (C_1 \cup C_4) - e \subseteq (C_1 \cup C_2) - e$ which is a contradiction.
\end{proof}

Now we define a relation $\sim$ for $\M = (E, \I)$ on $E \times E$ s.t. $e \sim f \iff (e = f) \lor (\exists C \in \C : e,f \in C)$.

\begin{prop}
	Relation $\sim$ is an equivalence on $E$.
\end{prop}

\begin{proof}
	We can easily see taht reflexivity and symmetry holds. Now for transitivity. $e \sim f$ and $f \sim g$ if one of these relations were obtained by $=$, then we can rename the elements and obtain $e \sim g$, thus we will assume both are obtained by the circuit property.

	Suppose $\exists C_1' \supseteq \{e,f\}, C_2' \supseteq \{f,g\}$ and we want to find $C_3' \supseteq \{e,g\}$. Let $C_1, C_2$ be s.t. $e \in C_1, g \in C_2$ and $C_1 \cap C_2 \neq \emptyset$ such that $|C_1 \cup C_2|$ is minimal. Denote $h \in C_1 \cap C_2, e \in C_1 \setminus C_2$ (otherwise we are done), this implies by (C3') that $e \in C_3 \subseteq (C_1 \cup C_2) - h$. If $g \in C_3$ we are done. Therefore assume $g \notin C_3$.

	$\exists i \in C_2 \cap C_3, i \notin C_1, |C_2 \cup C_3| < |C_1 \cup C_2|, g \in C_2 \setminus C_3$ and by (C3') we get that $\exists C_4 : g \in C_4 \subseteq (C_2 \cup C_3) - i$. Due to the fact that $C_4 \cap (C_3 \setminus C_2) \neq \emptyset$, otherwise it would be a proper subset, and $C_3 \setminus C_2 \subsetneq C_1 \setminus C_2 \Rightarrow C_1 \cap C_4 \neq \emptyset$ and $i \notin C_! \cup C_4$ so $|C_1 \cup C_4| < |C_1 \cup C_2|$ which contradicts the minimality.
\end{proof}

The classes of the equivalence $\sim$ on $E(\M)$ form so called "\textbf{components}". Also one can easily see that $\M$ connected $\iff |E| = 1$ or $\forall e,f \in E, \exists C \in \C : e, f \in C$.

\section{Separators}

\begin{defn}
	\textbf{Separation} $X \subseteq E$ s.t. $X$ is union of components of $\M$.
\end{defn}

\begin{prop}
	$\M = (E, \I)$, $X$ is separation $\iff$ $\forall C \in \C : C \subseteq X$ or $C \subseteq E \setminus X$.
\end{prop}

\begin{prop}
	$\M = (E, \I)$, $X$ is separation $\iff$ $\rank(X) + \rank(E \setminus X) = \rank(E)$.
\end{prop}

\begin{proof}
	"$\Rightarrow$" by using submodularity we may see that $\rank(X) + \rank(E \setminus X) \geq \rank(E)$ and now we have to show the other inequality. Let $B_{1}$ be max independent subset of $X$, hence $|B_{1}| = \rank(X)$ and $B_{2}$ be max independent subset of $E \setminus X$, hence $|B_{2}| = \rank(E \setminus X)$. Create $B = B_{1} \cup B_{2}$. If $X$ is a seprator then $B \in \I$ since $B_{1}$ and $B_{2}$ do not include a circuit and also no circuit is present in their union from the fact that $X$ is a separator. Moreover

	$$
	\rank(E) \geq |B| = |B_{1}| + |B_{2}| = \rank(X) + \rank(E \setminus X).
	$$

	"$\Leftarrow$" Assume that $\rank(X) + \rank(E \setminus X) = \rank(E)$ and $X$ is not a separator. Then $\exists C \in \C : C \cap X \neq \emptyset$ and $C \cap (E \setminus X) \neq \emptyset$. Now $C \cap X \in \I$ can be completed to $B_{1}'$ max independent subset of $X$ and similarly $C \cap (E \setminus X) \in \I$ can be completed to $B_{2}'$. Let $B' = B_{1}' \cup B_{2}'$ but $C \subseteq B' \Rightarrow B' \notin \I$ so $\rank(B') < |B'|$, but on the other hand $|B'| = |B_{1}'| + |B_{2}'| = \rank(X) + \rank(E \setminus X) = \rank(E)$. If $\rank(B') < \rank(E)$ this cannot happen, because otherwise we could add an elemnt so $\rank(B') = \rank(E)$.
\end{proof}

\begin{prop}
	$\M = (E, \I)$, $X$ is separation $\Rightarrow \forall F \subseteq E : \rank(F) = \rank(F \cap X) + \rank(F \cap (E \setminus X)) = \rank(F \cap X) + \rank(F \setminus X)$.
\end{prop}

\begin{proof}
	THis proof is similar as in previous proposition, therefore is omitted.
\end{proof}

\begin{prop}
	$\M = (E, \I), X \subseteq E, \M \setminus X = \M / X \iff \rank(X) + \rank(E \setminus X) = \rank(E)$
\end{prop}

\begin{proof}
	"$\Rightarrow$" Let $B_{x}$ be max independent subset of $X$ and $B$ be max independent subset of $\M \setminus X = \M / X$. From $\M / X$ we get that $B \cup B_{x} \in \B(\M)$ and from $\M \setminus X$ we get

	$$
	\rank(E) = |B \cup B_{x}| = |B| + |B_{x}| = \rank(E \setminus X) + \rank(X).
	$$

	"$\Leftarrow$" Assume that $\rank(X) + \rank(E \setminus X) = \rank(E)$ holds, compare $\I(\M / X)$ and $\I(\M \setminus X)$. Easily seen $\I(\M / X) \subseteq \I(\M \setminus X)$ is true for all $X$. Now $I \in \I(\M \setminus X) \Rightarrow \exists B$ max independent in $\M \setminus X$, $I \subseteq B \Rightarrow \exists B'$ s.t. $B \cup B' \in \B$ max independent in $\M$ therefore $|B \cup B'| = \rank(X)$ and $|B| = \rank(E \setminus X)$. From both we get the following

	$$
	|B \cup B'| = \rank(E) = \rank(X) + \rank(E \setminus X) = \rank(X) + |B| \Rightarrow |B'| = \rank(X)
	$$

	\noindent and hence $B'$ is max independent set of $X$. Lastly

	$$
	I \cup B' \subseteq B \cup B' \in \I \Rightarrow I \cup B' \in \I \Rightarrow I \in \I(\M / X).
	$$

	\noindent Therefore we get the other inclusion so it must be equal.
\end{proof}

\begin{cor}
	$X$ is separator of $\M$ $\iff$ $X$ is a separator of $\dual{\M}$. So also $\M$ is connected $\iff$ $\dual{\M}$ is connected.
\end{cor}

\begin{thm}
	$\M = \M_{1} \bigoplus \M_{2} \bigoplus \dots \bigoplus \M_{k}$, $\M_{i}$ is connected s.t. $\exists X_{1}, \dots, X_{k}$, $\M |_{X_{i}} = \M_{i}$ $\iff$ $X_{1}, \dots, X_{k}$ are connected components of $\M$.
\end{thm}

\begin{proof}
	We won't show the whole proof, but this follows from proposition on curcuits in directed sums and circuits in separators.
\end{proof}

\section{Higher connectivity}

For brewity and this section we will usually denote $X \subseteq E$ and then the component $E \setminus X$ as $Y$. If we look again at the connectivity define by separators we get that $X$ is separator \ifft $\rank(X) + \rank(Y) = \rank(E)$, note that $\rank(X) + \rank(Y) \geq \rank(E)$ always hold for arbitrary $X$. We may rewrite the equation like $\rank(X) + \rank(Y) - \rank(E) = 0$ and from this we will generalize to other types of connectivity.

\begin{defn}
	Partition of $E = X \cup Y$ is $k$-separation if

	$$
	\text{(\textcolor{orange}{\faCarrot})} \quad \rank(X) + \rank(Y) - \rank(E) \leq k-1.
	$$
\end{defn}

\begin{defn}
	Now we will define more types of separations:

	\begin{itemize}
		\item Tutte $k$-separation -- If (\textcolor{orange}{\faCarrot}) holds and $|X|,|Y| \geq k$.
		\item Essential $k$-separation -- If (\textcolor{orange}{\faCarrot}) holds and $\rank(X), \rank(Y) \geq k$.
		\item Cyclic $k$-separation -- If (\textcolor{orange}{\faCarrot}) holds and both $X$ and $Y$ contains circuit ($\rank(X) < |X|, \rank(Y) < |Y|$).
	\end{itemize}
\end{defn}

\begin{defn}
	From the separations we define connectivities.

	\begin{itemize}
		\item (Tutte) connectivity $\lambda(\M) = \min k$ s.t. $\M$ has Tutte $k$-separation, or $+\infty$ if it does not have one.
		\item Essential connectivity $\kappa(\M) = \min k$ s.t. $\M$ has essential $k$-separation, or $\rank(\M)$ if it does not have one.
		\item Cyclic connectivity $\dual{\kappa}(\M) = \min k$ s.t. $\M$ has cyclic $k$-separtaion, or $\dual{\rank}(\M)$ if it does not have one.
	\end{itemize}
\end{defn}

\begin{lemma}
	$(X,Y)$ is $k$-separation of $\M$ \ifft it is a $k$-separation of $\dual{\M}$.
\end{lemma}

\begin{proof}
	$\rank(X) + \rank(E \setminus X) - \rank(E) = \rank(X) + \dual{\rank}(X) - |X| = \dual{\rank}(X) + \dual{\rank}(E \setminus X) - \dual{\rank}(E)$.
\end{proof}

\begin{prop}
	$\lambda(\M) = \lambda(\dual{\M})$.
\end{prop}

\begin{proof}
	For proof see the previous lemma and also the fact that $|X|, |Y| \geq k$ is not affected by duality.
\end{proof}

\begin{prop}
	$\kappa (\M) = \dual{\kappa}(\dual{\M})$.
\end{prop}

\begin{proof}
	Let $(X,Y)$ be essential $k$-separation of $\M$ and assume, that it defines $\kappa(\M)$. We have that $\rank(X) \geq k, \rank(Y) \leq \rank(E)$ so then by $\rank(X) + \rank(Y) - \rank(E) \leq k-1$ implies that $\rank(Y) < \rank(E)$, therefore there $\exists H$ hyperplane of $\M$ s.t. $Y \subseteq H$. Hence $(E \setminus H)$ is co-circuit, $E \setminus H \subseteq X$. This can also be proven for $Y$ instead of $X$. Thus $(X,Y)$ is also a cyclic $k$-separation of $\dual{\M}$.

	For the other way $X$ contains a circuit $\dual{C}$ of $\dual{\M}$ which implies that $E \setminus \dual{C}$ is hyperplane of $\M$ and $Y \subseteq E \setminus \dual{C}$ and therefore $\rank(Y) < \rank(E)$ so $\rank(Y) - \rank(E) \leq -1$ and hence $\rank(X) \geq k$.
\end{proof}

\subsection{Tutte connectivity of uniform matroids}

\begin{lemma}
	$\M$ is Tutte $k$-connected ($\lambda(\M) \geq k$), $|E| \geq 2 (k-1)$. Then $\forall C \in \C(\M), |C| \geq k$. (And $\forall \dual{C} \in \dual{\C}(\M), |\dual{C}| \geq k$.)
\end{lemma}

\begin{proof}
	Assume we have $C \in \C$, where $|C| = k' < k$. Then set $X = C, Y = E \setminus C$, $|Y| \geq k'$. See $\rank(C) + \rank(E \setminus C) - \rank(E) \leq k'-1$ because $\rank(C) = k' -1$ and $\rank(E \setminus C) - \rank(E) \leq 0$. All these implies that $(X,Y)$ is Tutte $k'$-separation which is contradiction since it implies $\lambda(\M) \leq k'$.
\end{proof}

\begin{lemma}
	$\M$ is Tutte $k$-connected, $|E| \geq 2k-1$, then $\M$ has no $k$-element set $C$ s.t. $C \in \C \cap \dual{\C}$.
\end{lemma}

\begin{proof}
	Assume it is not satisfied, thus $\lambda(\M) \geq k, |E| \geq 2k-1$ and we have $\exists C \in \C \cap \dual{\C}, |C| = k$. Then $\rank(C) = k-1$ and $\rank(E \setminus C) = \rank(E) - 1$ since $C$ is circuit and co-circuit. Next $\rank(C) + \rank(E \setminus C) - \rank(E) = k-1 + \rank(E) - 1 - \rank(E) = k-2$. Lastly $|C| = k \geq k-1$ and $|E \setminus C| = |E| - k \geq 2k-1 -k = k-1$. Thus altogether $C$ and $E \setminus C$ is $(k-1)$-separation and hence $\lambda(\M) \leq k-1$, which is a contradiction.
\end{proof}

\begin{prop}
	Tutte connectivity can be prescribed as

	$$
	\uniform{r}{n} = \left\{ \begin{array}{l l}
		r + 1 & \text{if } n \geq 2r + 2 \\
		n - r + 1 & \text{if } n \leq 2 r - 2 \\
		\infty & \text{otherwise.}
	\end{array}
	\right.
	$$
\end{prop}

\begin{proof}
	Firslty note that $\dual{\uniform{r}{n}} \cong \uniform{n-r}{n}$, and with $\lambda{\M} = \lambda(\dual{\M})$ we can assume \wlogt that $r \geq n /2$. Which means that $n \geq 2r$, this way we got rid of the middle option, since we just switch to the dual.

	Now let $(X,Y)$ be a Tutte $k$-separation, \wlogt $k \leq |X| \leq |Y|$ so $|Y| \geq n/2 \geq r$ which also implies that $\rank(Y) = \rank(E) = r$. Because $k - 1 \geq \rank(X) + \rank(Y) - \rank(E) = \rank(X)$ we get that $X \notin \I$ and from that follows that $|X| > r$, therefore $\rank(X) = r \leq k-1$.

	In the first case set $|X| = r+1, \rank(X) = r$ and $|Y| \geq r+1$ and $\rank(Y) = r$. Get that $\rank(X) + \rank(Y) - \rank(E) = \rank(X) = r = (r+1)-1$ so it is Tutte $(r+1)$-separation.

	From the first case and duality we also obtain the second case.

	For the last case assume that either $n = 2r$ or $n = 2r+1$ (otherwise use duality). From the part $k - 1 \geq \rank(X) + \rank(Y) - \rank(E) = \rank(X)$ and that $\rank(X) = |X| \geq k$ we get a contradiction, because we got $k- 1 \geq k$.
\end{proof}

We mention a note on that, which says the following. If $\M$ has $\lambda(\M) = \infty$ then $\M$ is uniform ($\M \cong \uniform{r}{n}$).

\begin{prop}
	Let $\lambda(\M) = k$ finite. Then $\kappa(\M) \geq k$ and $\dual{\kappa} (\M) \geq k$.
\end{prop}

This will be without proof. Also we will state another theorem and again witout a proof, but before we do so we must define some terms. Let \textbf{girth} of matroid $\M$ be denoted and defined as $g(\M) = \min \{|C|, C \in \C(\M)\}$. Also it is good to mention that in general girth is upper bound to $\dual{\kappa}(\M)$.

\begin{thm}
	Let $\M$ be not isomorphic to a uniform matroid $\uniform{r}{n}$.

	$$
	\lambda(\M) = \min(\kappa(\M), g(\M)).
	$$
\end{thm}
