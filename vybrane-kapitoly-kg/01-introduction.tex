\chapter{Introduction to Steiner systems}

\section{Simple hypergraphs}

\begin{defn}
	Hypergraph is a tuple $(X, \M)$ where $\M \subseteq \mathcal{P}(X)$. Or generally just a set system.
\end{defn}

\begin{defn}
	A simple hypergraph (linear, $k$-graph) is $(X,\M)$ if $M_1 \neq M_2 \in \M \Rightarrow |M_1 \cap M_2| \leq 1$.
\end{defn}

\begin{example}
	Lets see some of the easier examples of simple hypergraphs.
	
	\begin{itemize}
		\item Graphs themselves are simple hypergraphs. Where $(X, \M)$ and $\M \subseteq \binom{X}{2}$.
		\item Or generally $k$-graphs, where $\M \subseteq \binom{X}{k}$.
		\item A well known Fano plane, see picture \ref{fano-plane}.
		\item Lets have a set of points $A$ and define $X = \binom{A}{2}$; which are edges in $A$ and $\M = \{\binom{T}{2} | |T| = 3, T \subseteq A\}$; which are triangles in $A$. This is also simple.
	\end{itemize}
\end{example}

Lets also define a chromatic number of such hypergraphs as:

$$
\chi (X,\M) := \min \{k | \exists \bigcup_{i=1}^k X_i = X \text{ and no } X_i \text{ contains } M \in \M\}.
$$

\noindent In other words: At least two "colors" for each $M \in \M$. And by Ramsey theory we may state that $\forall k \ \exists X: \chi(X,\M) > k$.

\begin{example}
	For fixed $k \in \N$ we have $k$ committees, each of them has $k$ members and they are meeting in a room with $k$ seats. Any two committees are disjoint. Can someone sit at the same place? And how many of them? -- This was stated by Erdos, Faber and Lovász in 1972.
\end{example}

\begin{thm}[Kuhn, Osthus, Kang, Kelly, Methuku, 2023]
	Showed that the previous example is true for large $k$.
\end{thm}

And some different formulation is by using simple hypergraphs. Lets have simple hypergraf $(X,\M)$ where $|\M| = k$ and line chromatic number $\leq k$. That is coloring the edges instead. If they meet they have to be distinct.

\begin{prop}
	$\chi_l (K_{2k}) = 2k-1$ for $k \in \N$.
\end{prop}

\begin{proof}[Sketch of proof]
	Lets draw the graph, so the vertices are on a circle. Then take the edges across in the same direction and one from the inside of the circle to the boundary and color them. Then rotate and color once again, until colored.
\end{proof}

\section{Dual hypergraphs}

Lets now define a dual hypergraphs, which may not be so intuitive at a first glance. Lets see a picture \ref{dual-hg} showing the incidence graph for $(X,\M)$. Then the dual is obtained by switching the parts of $(X, \M)$ and $(\M',X')$. Lets denote the dual of $(X,\M)$ as $d(X,\M)$.

\begin{figure}[!ht]\centering
	\begin{tikzpicture}
		\draw (0,0) to (6,0);
		\draw (0,3) to (6,3);
		\node (X) at (6.5,0) {$\M$};
		\node (M) at (6.5,3) {$X$};
		\draw (2,0) -- (1,3) node[midway, below, left] {$\in$};
		\draw (2,0) -- (2,3);
		\draw (2,0) -- (3,3);
		\draw (2,0) -- (4,3);
		\node (M) at (2,-.2) {$M$};
		\node (x1) at (1,3.2) {$x_1$};
		\node (x2) at (2,3.2) {$x_2$};
		\node (x3) at (3,3.2) {$x_3$};
		\node (x4) at (4,3.2) {$x_4$};
	\end{tikzpicture}
	\caption{Diagram for the dual hypergraph.}
	\label{dual-hg}
\end{figure}

\begin{lemma}
	$(X,\M)$ is simple \ifft $d(X,\M)$ is simple.
\end{lemma}

\begin{proof}[Proof by picture]
	For the proof see the picture \ref{simple-dual-hg}. This $C_4$ like structure happens if it is not simple and hence when we flip the diagram, obtaining the dual, the diagram does not change.
	
	\begin{figure}[!ht]\centering
		\begin{tikzpicture}
			\draw (0,0) to (6,0);
			\draw (0,3) to (6,3);
			\node (X) at (6.5,0) {$\M$};
			\node (M) at (6.5,3) {$X$};
			\draw (2,0) -- (2,3);
			\draw (4,0) -- (4,3);
			\draw (2,0) -- (4,3);
			\draw (4,0) -- (2,3);
			\node (M1) at (2,-.2) {$M_1$};
			\node (M2) at (4,-.2) {$M_2$};
			\node (x1) at (2,3.2) {$x_1$};
			\node (x2) at (4,3.2) {$x_2$};
		\end{tikzpicture}
		\caption{Simple dual graphs proof.}
		\label{simple-dual-hg}
	\end{figure}
\end{proof}

Now lets denote $A(X,\M)$ as an incidence matrix of a given hypergraph, then the dual has incidence matrix $A(d(X,\M)) = A^T(X,\M)$. Lets consider $(X, \M)$ a $k$-uniform hypergraph. Can we somehow boudn the size of $|\M|$? We may establish trivial bounds as $0 \leq |\M| \leq \binom{X}{k}$. We will further make better bound. lets see the picture \ref{bounds}. We will be using double counting, for which we notice that for some $\binom{X}{2}$ we have at least one $M \in \M$.

\begin{figure}[!ht]\centering
	\begin{tikzpicture}
		\draw (0,0) to (6,0);
		\draw (0,3) to (6,3);
		\node (X) at (6.5,0) {$\M$};
		\node (M) at (6.5,3) {$\binom{X}{2}$};
		\draw (4,0) -- (4,3);
		\draw (4,0) -- (3,3);
		\draw (4,0) -- (5,3);
		\node (M) at (4,-.2) {$M$};
		\node (Xo2) at (1,3.3) {$\binom{X}{2}$};
		\draw (1,3) -- (2,0);
		
		\node (x2) at (4,3.3) {$\binom{k}{2}$};
	\end{tikzpicture}
	\caption{Providing better bound.}
	\label{bounds}
\end{figure}

\noindent Hence we may compute the following.

$$
|\M| \cdot \binom{k}{2} = \sum_{M \in \M} \binom{|M|}{2} \leq \binom{|X|}{2}
$$

\noindent Therefore we can obtain the bound.

$$
|\M| \leq \frac{\binom{|X|}{2}}{\binom{k}{2}}
$$

\noindent See that this bound is actually tight. For $k=2$ we can consider a graph $K_n$ and for $k=3$ we may look at Fano plane. If the equality hold we call it \textit{Steiner system}. Or in other words it is true if $\forall x \neq y \in X \ \exists! M \in \M$ such that $\{x,y\} \subseteq M$. For $k = 3$ we call this \textit{Steiner triple system} or STS for short. This is particularly used in experiments and mainly in agriculture. Usually this is then denoted as \textit{BIBD} or balanced incomplete block design.

We may say that for STS to exists it must hold that both $n-1$ and $\frac{\binom{n}{2}}{\binom{3}{2}}$ must be integers. So it only exists if $n$ is either $6k+1$ or $6k+3$.

\begin{thm}
	Steiner triple system exists \ifft $n$ is either $6k+1$ or $6k+3$.
\end{thm}

\begin{proof}
	We will be showing how it can be generated. That is from two STS we create a new one.
\end{proof}

\section{Introducing BIBD}

\begin{defn}[BIBD]
	Hypergraph $(X, \M)$ is BIBD $(n,k,\lambda,t)$ if $|X| = n$ $\M \subseteq \binom{X}{k}$ and $\forall x_1, \dots x_t \in \binom{X}{t}$ we have that $|\{M \in \M | \{x_1, \dots, x_t\} \subseteq M\}| = \lambda$. 
\end{defn}