\chapter{Stochastic processes}

Stochastic process is a sequence of random variables $X_{1}, X_{2}, X_{3}, \dots$. We will show that there exist many of them.

\begin{itemize}
	\item Markov chain (+ extra condition)
	\item Wiener process
	\begin{itemize}
		\item Browner motion
		\item Stock prices
		\item Limit version of RN
	\end{itemize}
	\item \textbf{Arrival times} or alternatively \textbf{waiting for success}.
\end{itemize}

We will be looking at the last type of the processes.

\section{Bernoulli process (denoted as $\text{Bp}(p)$)}

That is we have $X_{1}, X_{2}, \dots$ iid and each one of them is $X_{i} \sim \text{Ber}(p)$ so with probability $p$ it is $1$ and $0$ with probability $1-p$.

\begin{observ}
	\begin{itemize}
		\item $X_{n}, X_{n+1}, \dots$ is also $\text{Bp}(p)$
		\item $X_{N}, X_{N+1}, \dots$ is also $\text{Bp}(p)$, with $N$ a random variable dependent only on the past
	\end{itemize}
\end{observ}

Then we will define $T =\min \{t : X_{t} = 1\}$ or by words the time of the first success / arrival. And we can easily see that $T \sim \text{Geom}(p)$ so $\mathbb{E}[T] = \frac{1}{p}$ and $\text{var} [T] = \frac{1-p}{p^{2}}$.

Now we will try to generalize this by $T_{k}$ as the time of the $k$-th arrival. So $T_{1} = T$. Or written as $T_{k} = \min \{t : X_{1} + X_{2} + \dots + X_{t} = k\}$.

Other interesting variable is the $k$-th waiting time (inter arrival) and it will be denoted as $L_{k}$. To describe this variable it is the time between $k-1$ arrival and $k$-th arrival. Then it follows

$$
L_{k} = T_{k} - T_{k-1} \text{ when we put } T_{0} = 0
$$

$$
L_{k} \sim L_{1} = T \Rightarrow L_{k} \sim \text{Geom}(p)
$$

And all $L_{i}$ are independent. From the other way we can define $T_{k}$ as the sum $\sum_{i=1}^{k} = L_{i}$. So we can then get expected value and variance.

$$
\mathbb{E}[T_{k}] = \mathbb{E}[L_{1}] + \mathbb{E}[L_{2}] + \dots + \mathbb{E}[L_{k}] = \frac{k}{p}
$$

$$
\text{var}[T_{k}] = \text{var}[L_{1}] + \text{var}[L_{2}] + \dots + \text{var}[L_{k}] = k \cdot \frac{1-p}{p^{2}}
$$

How could we compute $\Pr[T_{k} = t] = ?$ Easily we can compute this by convolution formula $\binom{t-1}{k-1} p^{k}(1-p)^{t-k}$.

Lastly we define $N_{t}$ as the sum $X_{1} + X_{2} + \dots + X_{t}$ which is the number of successes till the time $t$. And $N_{t} \sim \text{Bin}(t,p)$. So $\mathbb{E}[N_{t}] = tp$ and $\text{var}[N_{t}] = tp(1-p)$.

\subsection{Alternative definition}

We can define Bernoulli process by different definition. First we will define $L_{1}, L_{2}, \dots$ as iid $\sim \text{Geom}(p)$ and then $T_{k} = \sum_{i=1}^{k} L_{i}$. And $X_{i}$ is $1$ if $T_{k} = i$ for some $k$ or $0$ otherwise. Then $(X_{i})_{i}$ is $\text{Bp}(p)$.

\subsection{Merging of Bernoulli process}

We will have two processes which are independent.

$$
\begin{array}{rl}
X_{1}, X_{2}, X_{3}, \dots & \text{Bp}(p) \\
Y_{1}, Y_{2}, Y_{3}, \dots & \text{Bp}(q)
\end{array}
$$

Then the merge is $Z_{i} = X_{i} \text{ or } Y_{i}$. Properly it is

$$
\begin{array}{rl}
Z_{1}, Z_{2}, Z_{3}, \dots & \text{Bp}(p + q - pq) = \text{Bp}(1 - (1-p)(1-q))
\end{array}
$$

\subsection{Splitting Bernoulli process}

We can also split one Bernoulli process. Firstly we got

$$
\begin{array}{rl}
Z_{1}, Z_{2}, Z_{3}\dots & \text{Bp}(r)
\end{array}
$$

If $Z_{i}=1$ then $X_{i} = 1$ with probability $\alpha$ and $0$ with probability $(1 - \alpha)$ and if $Z_{i} = 0$ then $X_{i}=0$. By this construction we get new Bernoulli process.

$$
\begin{array}{rl}
X_{1}, X_{2}, X_{3}, \dots & \text{Bp}(\alpha r)
\end{array}
$$

\section{Poisson process (denoted as $\text{Pp}(\lambda)$)}

As we defined Bernoulli process we also can define Poisson process which can be described as a continuous approximation of $\text{Bp}(p)$. Now the arrival times are real numbers.

\begin{defn}
	\begin{enumerate}
		\item For any interval of length $\tau$ probability of $k$ arrivals is the same. Denoted as $P(k, \tau)$.
		\item Number of arrivals in $[a,b]$ is independent of number in $[0, a)$.
		\item $P(0, \tau) = 1 - \lambda \tau + o(1)$, $P(1, \tau) = \lambda \tau + o(1)$, $P(k, \tau) = o(1)$. for $k \geq 2$ where $o(1)$ is something that goes to zero
	\end{enumerate}
\end{defn}

% TODO: What is the small o??

Then the sequence $T_{1}. T_{2}, T_{3}, \dots$ is $\text{Pp}(\lambda)$ where Ts are the arrival times.

% $Pr[N_t = k] = P(k, t) = $

As in Bernoulli process we have $T_{k}$ as the time of $k$-th arrival. Then $N_{T}$ is the number of arrivals in $[0,t]$ and $N_{T} \sim \text{Pois}(\lambda t)$ so $P(k, t) = e^{-\lambda t} \frac{(\lambda t)^{k}}{k!}$.

We can show that by the following approximation:

$$
\begin{array}{c}
	Pr[N_t = k] = P(k,t) \implies P(1, \frac{t}{l}) = \frac{\lambda t}{l} + o(1) \\
	Pr[N_t = k] = P(k, t) \approx P[\text{there are K small intervals that has 1 arrival}] = \\
	= Pr[Bin(l, \frac{\lambda t}{l}) = k] \newline \implies lim_{l \to \infty} Bin(l, \frac{\lambda t}{l}) \to Pois(\lambda  t)
\end{array}
$$

Then again $L_{k} = T_{k} - T_{k-1}$ so $\Pr[L_{k} \geq t] = \Pr[\text{no arrival in }[T_{k-1}, T_{k-1}+t]]$ and that is equal to $P(0,t) = e^{-\lambda t}$. Next $\Pr[L_{k} \leq t] = 1 - e^{-\lambda t} \Rightarrow L_{k} \sim \text{Exp}(\lambda)$.

\subsection{Alternative description}

As in Bernoulli process we can define Poisson process the other way around. We start with sequence of iid $L_{1}, L_{2}, \dots \sim \text{Exp}(\lambda)$. Then $T_{k}$ is the sum $T_{k} = \sum_{i=1}^{k} L_{i}$. And we also get the same $N_{t}$.

\begin{thm}
	This also defines $\text{Pp}(\lambda)$. In other words it satisfies all of the three properties.
\end{thm}

Again as in Bp we can see that expected value of $T_{k}$ and variance is the sum of expected values (resp. variances) of $L_{i}$ which are $\frac{1}{\lambda}$ (resp. $\frac{1}{\lambda^{2}}$). By convolution we get that

$$
f_{T_{k}} (t) = \frac{\lambda^{k} t^{k-1}e^{-\lambda t}}{(k-1)!}
$$

\subsection{Splitting of Pp}

We have a $\text{Pp}(\lambda)$ and each one is split (1 or 0) with probability $p$ (resp. $1-p$). And then we get two processes $\text{Pp}(p \lambda)$ and $\text{Pp}((1-p) \lambda)$ and these are independent. Two new processes have still the same properties but with new $\lambda'$. To properly show that this holds we need to show all the properties from the definition.

% TODO

$$
Pr[T_1 > t] = Pr[T>t \ \&  \ T' > t] =  \dots
$$

\begin{rem}
	Proving independence is quite cumbersome. The proof is based on an example from the lecture.
\end{rem}

\subsection{Merging of Pp}

If we have two processes $\text{Pp}(\lambda)$ and $\text{Pp}(\lambda')$ we can merge these to get $\text{Pp}(\lambda + \lambda')$. Again to properly show that this holds we must show that the $\min$ of two $\text{Exp}$ distributions is again $\text{Exp}$ distribution with the sum. Which is quite easy since they are independent, then we get the product of exponent functions which is the same as the sum of their exponents.

What if we look at the $\Pr[T - t > s \vert T > t]$ which by definition is $\frac{\Pr[T > s+t \land T > t]}{\Pr [T > t]}$ and that is equal to $\frac{e^{-\lambda (s+t)}}{e^{-\lambda t}} e^{-\lambda s}$ and we get the property that the Poisson process is \textbf{memory-less} so it doesn't matter when we will start measuring our data.
