\documentclass{book}

\usepackage[czech]{babel}

\author{Tomáš Turek}
\title{Povídky plné ptákovin}


\begin{document}
	\maketitle
	
	\section{První slovo}
	
	Celé je to sbírká krátkých povídek, kterě se mi buď zdáli anebo jsem si je vymyslel. Pak bývá mezislova o tomto principu.
	
	\part{Utlačená dimenze}


\chapter{Začátek}

Je rok 196 po narození Eduare Krista, nástupce Ježíše Krista II. Jmenuji se Wenclaw, ale kamarádi mi říkají Venculík, ty mě zatím oslovuj Wenclaw, pocházím z poklidného města Burriwuk. Možná jste mě už potkali, ano je to tak, pracuji v Místní restauraci, která se zde nachází. Bohužel makám tam pouze jako vedlejší číšník, který má svoji práci strašně rád. Nejlepší je, když po mně zákazníci něco chtějí. Pořád jen: pane ta polívka je studená, „panme yá bych ješťe pívo,“ kdyby aspoň dokázal stát rovně, když mě o to prosí. Asi tak nejhorší je, když se jim zdá, že je jejich jídlo špatně uvařené. Já jdu kolem a oni na mě mávají, sice se vždy snažím je nevnímat, ale málokdy se mi to povede. Tak k nim přijdu a ptám se s úsměvem ve tváři. “Co byste si přál, milostpane?” a oni odpoví: “Ty vepřová žebírka chutnají nějak divně.” a tak se zdvořile zeptám, zdali je budou ještě jíst. Oni odseknou ještě naštvanějším tónem: “To si děláte srandu, samozřejmě, že ne.” Tak jim jen povím: “Tak jste si je neměl objednávat.” a odnesu to. No nicméně už jsem byl tak pětkrát vyhozen, teda málem, a to jen díky tomu, že jsem šéfíkovi řekl, že za takový prachy by tu snad nikdo nepracoval. Naštěstí práce není vše a vždycky, když skončí, tak se těším až půjdu domů za svojí milovanou manželkou. Bohužel děti ještě nemáme, možná je to dobře, možná ani ne, těžko říct ještě jsem je neměl. To je možná i důvod proč toho lidi tak často litují, protože to ani nejde vyzkoušet. Myslím, že by to mohl být dobrý nápad: Půjčovna dětí. Šlo by pouze o to, že by sem přišel pár, řekl na jak dlouho a jaký věk by dítě mělo mít, pokud je dostatek zásob tak si dítě vypůjčí. Popravdě si myslím, že by se tím vyřešilo přelidnění na naší planetě Zemi, přece jenom deset miliard lidí už je celkem dost. Ale i tak je v dnešní době spousta problému, dochází všechny nerostné materiály. Ale někdy bych se rád podíval, jak vypadala ta rapa, nebo jak to bylo, prý se na to jezdilo v automobilech. Bohužel o tomhle se pouze dočteme v holografických knihách. Ostatní 3D knihy se téměř rozpadají. Zase by to mohlo být celkem hustý, používat staré věci, prý se něco podobného dělalo i v minulé době, teď si pouze nemůžu vzpomenout, jak se tomu říkalo, ale to je jedno. Nicméně si stejně myslím, že spousta vynálezů by nemělo ani existovat. Nemyslíte si to taky?

\section{Neúplný konec}

I přesto, že mám, no stabilní, to je tak jediný, co se dá říct, práci, nebudu si nalhávat, že mě to baví. Právě proto, ještě studuji, abych měl lepší práci, bohužel předtím jsem si to neuvědomoval. Popravdě už se těším, jak už budu mít dostudováno a dělat moji vysněnou práci. To bude aspoň, dělat kuchaře, no to vám povím, to je úplně něco jiného než v té blbý restauraci. Jak já to tam nesnáším. A je to dokonce ještě lepší, zítra mám poslední zkoušku. Už se jen vyspím. Mám moji zkoušku, o které jsem mluvil, já věřím, že si to pamatuješ. Nezaspal jsem a taky jsem se nezapomenul obléci, jak to často bývá. No dobře, uznávám, málokdy se to někomu stane, nyní to můžu říct i já, že se mi to taky nestalo, máma by byla na mě hrdá, vlastně asi i je, protože jsem jí to už říkal, to víte, jak se říká, hned z první ruky, ne? Jakmile jsem došel na zkoušku, však už víte, tak jsem byl hned na řadě, protože tam nikdo jiný nebyl. Tak vám teda nevím, jestli jsem to stihl včas. Došel jsem tam a podíval se na můj úkol. Tam bylo napsáno: Vepřová žebírka, už jsem viděl toho zákazníka, jak to komentuje, ale on tam naštěstí nebyl. Pustil jsem se do vaření, skoro hned to bylo, zhrubapůl hodiny. Zkoušející začali chutnat moji delikatsu. No ti se tvářili. Dost hrozně. Tak mě vyhodili. Už jsem tak doufal a zase nic. Příště Zkusím něco doopravdy jiného.

\section{Konec, vlastně nový začátek.}

Tak jsem se rozhodl. Půjdu studovat technologie. Přihlásil jsem se do rychlokurzu na 2 týdny, prý to bude určitě užitečný. Hned po prvním kurzu vám řeknu, že se asi stanu nějakým tím vynálezcem a pomůžu tak celému světu. Myslím to vážně, opravdu. Navíc to byla pouze první hodina, jaký to bude za 2 týdny. Po dvou dokonalých týdnech jsem konečně dokončil svůj kurz. Dokonce jsem dostal i svůj diplom, ten vypadal spíš jako kus toaletního papíru, ale i to se počítá. Ještě před tím, než jsem došel domů, tak jsem dal výpověď v restauraci a po cestě jsem si dal to pivo, to nesmělo chybět. Po třech hodinách jsem těžce držel krok, ale zvládnul jsem to, došel jsem domů. Tam jsem oznámil šťastnou novinu, bohužel moje manželka měla pro mě taky šťastnou novinu, byla konečně těhotná. To vám povím, chtěl jsem se radovat, ale jí vadilo, že jsem se u toho válel na zemi a smál se na celý kolo. Raději už na pivo nikdy nepůjdu, fakt nechápu, proč tam lidé chodí, když to má takový následky, ovšem sranda to byla, pouze ale pro mě, neb jsem se doma párkrát vyblib. Další den už jsem pádil si najít novou práci. Šel jsem na univerzitu, moc jsem to nečekal, ale oni mě fakt vzali. Takhle jsem započal svůj výzkum ohledně urychlení a zdokonalování dopravy. Popravdě mi to přišlo jako hodně blbý nápad, protože člověk snad ani nemůže vymyslet něco lepšího, než je Hyperloop.

\chapter{Svítání na lepší včerejšky.}

\section{Uvědomit si to, to je první krok.}

Začínám další den v mé nové práci. Nečekal jsem to, ale v porovnání s minulou prací je tady tahle naprosto zlatá, samozřejmě ne doslova, to by zřejmě nedávalo smysl. Kdybych pracoval ve zlatnictví, tak by se o tom možná dalo bavit, ale to pořád neznamená, že práce je ze zlata, to prostě nejde. Aspoň v to doufám, jinak bych ji bral, pokud by nebyla náročná. Už bych se měl vrátit ke své bohužel nezlaté práci. To vám povídám, celý dny nic nedělám, pouze každému kolemjdoucím a nadřízeným říkám, že přemýšlím nad něčím velkým a že se mají na co těšit. Říkali mi, že se už se nemůžou dočkat na mé výsledky, být nimi bych se moc netěšil. Jo vlastně, za dva dny je nějaká konference nebo co, Předpokládám, že to bude stejná pakárna jako vše tady. Tak po dvou dnech nejen houpáním nohami, se blíží naše konference, to bude k popukání. Hned co jsem přišel se mě ptali Jak to šlape, no popravdě jsem jím odpovídal, že už jsem párkrát měl lepší boty. Nevím proč se mě na to ptali, asi se jim líbili moje botičky. Začala konference, to jsem opravdu nečekal. Oni normálně říkali, na co přišli, nebo s jakou teorií přišli. Kdybych předtím radši něco dělal. To bylo strastiplných 5 hodin. Naštěstí moje chytrá kebule vymyslela naprosto dokonalý nápad. Už jsem přišel na řadu, tak jim samozřejmě sebevědomě, jak jinak, vykládám: „Mám zatím malou teorii, dalo by se říci pouze v zárodku,“ mezi tím se začínám celkem dosti potit, mělo by se asi vyvětrat, ale to teď neovlivním a raději jsem pokračoval se svým projevem, „jediným problémem dnešní dopravy jest všem známá odporová síla a tomu se pokusím co nejvíce zamezit.“ Všichni se na mě dívali udiveně, ale po pár sekundách mi začali tleskat asi tak nejvíc ze všech projevů. No dobře, před tím jsem nedával pozor, ale i tak jsem to doma říkal s velkým nadšením. V duchu jsem si říkal, jaký jsem génius. Ovšem nebe není nikdy bez mráčků. Začalo to už doma, když se mě manželka hned zeptala, jak to chci provést. Nebyla by to větší náhoda, samozřejmě se mě zeptali i v práci. Já jsem ovšem pokračoval v mém mlhavém popisu, který vlastně nic neříkal, už jsem si připadal jak politik. Pár dní pozdějijsem začal opravdu přemýšlet nad tímto tématem, zkrátka nešlo mi to z hlavy. Věděl jsem, že mě určitě něco chytrého napadne.

\section{Nápad může být někdy i hodně jednoduchý.}

Už uběhlo pár těch měsíců, no vlastně ne pár, to jsou totiž dva, popravdě jich bylo celkem šest, což každý vidí, že to nejsou dva, to je zřejmé. No jednoduše řečeno často mě napadá spoustu kravin jako třeba tahle, ale za boha jsem nemohl vymyslet, jak zlepšit dopravu. Možná si říkáte, že nějaký takový obor, kde pracuji neexistuje, ale to fakt není pravda. Existuje a jsem v tom nejlepší, protože zatím se situace v dopravě nezlepšila. No jo vše mi jde přímo od ruky. Moje manželka, už ale nedokáže trpět, jak nejsem ducha přítomný a jak se ji prý ani nevěnuji. Tak jsme se rozhodli jít na film do Old-school kina, které bylo pár minut od našeho domu. Co jsme tam dojeli, tak jsem samozřejmě koukal, co dávají, popravdě jsem koukal jen a pouze na ceník. To víte každý se vás snaží oškubat, není to tak lehké. Ale tohle mě opravdu trklo, jak je možné, že 2D film je asi o polovinu levnější než nejnovější filmy ve 3D, ba dokonce ty XYZ-D, to se jim nechtělo ani počítat. Na druhou stranu je chápu, taky bych se na to vykašlal. Film trval dvě hodiny, ve skutečnosti asi tak pět dní, to byla ale splácanina, to byste ani nechtěli číst na to, to vidět. Prostě děs, raději jsem už zapomněl, jak se to jmenovalo. Poté jsme šli navštívit Restauraci, dokonce tu stejnou, ve které jsem byl. Chtěl jsem si aspoň jednou užít být tím otravným zákazníkem. Můj bývalí šéfík, mě hodně rád viděl až samou radostí zlomil vařečku, prý jsem jim taky změnily výdělky od té doby, co tam nejsem. Zřejmě se mají dobře, ale i tak jim to nezávidím. Celý večer proběhl bez problému, samozřejmě nepočítám situaci v restauraci, to bylo ovšem naschvál, dlouho jsem se takhle nezasmál. Ale o to, to bylo lepší. Po těžkém večeru jsme zalehli do společné lože s dobrými zážitky. Oproti tomu byla celá noc naprosto příšerná. Nespal jsem ani minutu, pořád jsem musel přemýšlet nad mojí prací. No jo hotový workoholik, jako vždycky, ale zdálo se mi, jako bych měl samotnou odpověď na jazyku, jenom mě samotný jazyk neposlouchal, bylo to hrozné. Ale najednou z čista jasna, jako kdyby z nebe spadl zase Ježíš, to vám taky byla podívaná, ale to byla zase jiná příhoda. Spojil jsem si vše dohromady a nápad byl na světě. Už jsem se nemohl dočkat do práce, abych to vše dal do srozumitelné kopy. To, kdybych slyšel před pár lety, nějaký těšení na práci, to bych si hned neflákal, no asi bych ani nemusel, protože jsem se nikdy předtím netěšil. Už jsem usnul s poklidem. Další den jsem do práce naběhl s plnou radostí a skočil ke své tabuli. Vše jsem zapisoval, bylo to jako kdyby mě moje myšlenky doslova ovládali. Nápad vypadal možná až moc jednoduše, ale byl
dokonalý.

\section{Nečekané zásluhy.}

Nápadu už neschází, myslím, že už náš svět nebude takový, jaký vždycky byl. Už vidím, jak budu vydělávat miliardy a budu populární mezi všemi, dokonce i ten spolužák Dyrhuwk mi bude závidět. No jo já bych byl na něj možná i hodnější, ale když mi strkal hlavu do záchodu, ještě k tomu to bylo už ve školce. Na to nikdy nezapomenu, tak aspoň doufám, že bude dosti naštvaný. Mimochodem za pár hodin, teď to jsou opravdu dvě, máme moji oblíbenou konferenci. Tentokrát, tam budu vykládat o mém ještě málokomu známém nápadu s kterým udělám díru do světa, možná i doslova. Ještě doladím detaily abych byl na konferenci plně připraven. Dvě hodiny utekli jako voda, no voda sice nemá nohy, ale však znáte to, říká se to i přesto, že to nedává smysl. Už jsme se všichni shromáždili a započali tak naše jednotlivé přednášky. Já jsem se celou dobu usmíval, protože jsem věděl, že tohle nikdo nemá, ještěaby jo, když jsem sám v tomhle oddělení. Po několika neskonale nudných proslovech o naprostých nesmyslech, jsem přišel na řadu já. Na tento moment jsem čekal celý můj život, popravdě nejspíše tomu tak není, ale vyzní to hodně sebevědomě. Můj přednes byl tak dobrý. Člověk by slyšel i spadnout špendlík na zem, samozřejmě v moment, když jsem nemluvil. Ale na konci, byla jejich reakce naprosto jiná, než mé očekávání. Jedni kroutili hlavou a říkali: „To přece není možné,“ druzí se začali modlit a ti třetí, to bych nechal raději bez komentáře. Můj šéf si pak se mnou chtěl promluvit osobně. Čekal jsem spoustu věcí, medaili, ocenění nebo dokonce pořádně mastný grant, ale tohle. Říkal mi, že bych takhle mohl rozbít stabilitu vesmíru a že to ani není možné to udělat. Hlavně však chtěl vědět kde jsem na takovou blbost přišel a ke konci mi poradil abych to vyhodil. Blbý ale nejsem, moc mi to docházelo, že mi chtěl moji práci ukrást, to víte, jak jsem říkal, všude se vás snaží oškubat. Nevzdal jsem se a se vší silou jsem při odchodu třískl dveřmi, to byste měli slyšet, to byla ale šupa. Samozřejmě jsem to všechno začal díky němu brát ještě na větší váhu. První věc, kterou jsem hned udělal, tak jsem si koupil největší váhu, co jsem našel. Později už jsem doopravdy pracoval na svém projektu.

\chapter{Blbý nápad není vždycky blbý, ale většinou je.}

\section{Drumlanův pejsek}

Asi si říkáte, co jsem teda vymyslel. Zatím o tom pořád mluvím, ale prakticky jsem nic neřekl, už to je jak na té první poradě. Tak já vám teda povím, jak to funguje. Vzpomínáte si ještě na 2D kino. A taky na to, že největší problém je odporová síla. Jako první, mě napadlo, že by se udělali dráhy, ve kterém by bylo vakuum, ale to by bylo hodně energeticky náročné. Takže jsem od tohohle hned opustil, protože to byla naprostá blbost. Ale když jsem se nad tím více zamyslel, tak mě doslova praštilo do hlavy, bouli jsem měl ještě měsíc. Jednoduše řečeno, tak nejvíce odporové síly momentálně působí především na předek jakéhokoliv dopravního prostředku, a to včetně Hyperloopu. Co to nějak omezit, řekl jsem si. A teď se konečně vyplácí moje návštěva kina. Nesmím zapomenout na manželku, protože to byl hlavně její nápad. Sečteno podtrženo, dopravní prostředek se prostě zkrátí jen na 2 dimenze. Tím pádem bude vzduchem jezdit, či létat, jako když projedete nožem máslo, ale nesmí být z lednice, to už není tak lehké. Teď už mi zbývalo to nějak provést do reality. Oproti vakuu to bude ovšem hračka. Teda myslel jsem si to a nikdo mě nemohl přesvědčit, že to tak není. Po tomhle úsudku jsem pracoval celé dny a noci, doma a samozřejmě něco i v práci. Stejně jsem většinu dělal doma, aby mi nemohli můj nápad převzít. Po pár letech později byl prototyp na světě. To nebylo to jediné, co už bylo na světě. Jestli umíte počítat, tak si to dáte samozřejmě dohromady. Počkat já zas řekl jen pár, že. Tak to není bylo jich víc. Kašlu na to prostě se nám narodil náš syn. Jmenuje se Drumlan, celkem pěkné jméno, neříkáte si. Zpět k vynálezu. Tenhle popis bude hodně těžký a na dlouhou dobu. Doufám, že to aspoň pochopíte. Když nad tím přemýšlím, tak jsem to ještě nepatentoval, takže si představte divně svítící objekt. Jediné, co zbývalo byl testovací subjekt. Bohužel manželka byla proti Drumlanovi. Já jsem říkal, vždyť bude průkopníkem nové éry. Takže z toho sešlo. Já jsem to stejně musel vyzkoušet. Tak jsem si koupil psa, prodejce se mě ptal na rasu, já jen odpověděl, že je pro mě důležité jen to, kolik stojí. Při odchodu se na mě pejsek usmíval a já jsem si říkal pro sebe, že se snad bude takhle smát pořád. Doma se pejsek líbil, já jsem však varoval. Aby si na něj nezvykali. Hned jsem spěchal do dílny. Se vším nadšením, jsem ho hodil do daného objektu, však si to představujete. Čekal jsem všechno možné, ale tohle snad nikdy.

\section{Rychlý sestup.}

Takže jak už očekáváte, pejsek se už neukázal. Byl jsem úplně na dně, nevěděl jsem, co jsem měl v tu chvíli dělat. Takhle jsem se ještě nikdy necítil. Nejhorší moment mého života, aspoň jsem si to myslel. Rodina byla nadšená, co se stalo, jakmile viděli můj výraz tak si mysleli, že si z nich dělám srandu, chvilku jim to trvalo, ale došlo jim to, že se stalo nejhorší. Následující vteřiny, minuty, hodiny ba dokonce dny, jsem ležel v posteli a jen přemýšlel. Co a jak se to stalo, vždyť to je nemožné, aby něco jen tak zmizelo. Asi to tak mělo být. Chtěl jsem vše schovat zajel jsem daleko do lesů, tam jsem můj vynález zahrabal, aby ho už nikdo neviděl, a to hlavně já. Během měsíce to všechno šlo do háje. Ztratil jsem práci, taky manželku, a to i syna. Dostal jsem se ještě hlouběji, než jen na dno. Stal jsem se alkoholikem, spal jsem na ulici a vydělával jsem si žebráním. Párkrát jsem i potkával mého syna i manželku. Ale při poslední návštěvě se dělo něco divného. Celý dům strašně zapáchal, jako kdyby se někdo dlouho nemyl. Dokonce jsem to nebyl já. To vám povím taková rychna to byla, málem jsem se z toho zeblil. Manželka, teda už to nebyla manželka, hned zavolala nějaký odborníky, přes smrad nebo co, fakt nevím. Prý doma říkali, že už něco cítili, před několika dny, ale to si jen mysleli, že to byl smrad po mně. Hned jsem jím to vysvětlil, že nepáchnu, sice tamti chlápci měli jiný názor, ale to jen kvůli tomu, že je určitě podplatila. Po tomhle jsem rychle odešel. Ale poté jsem stejně musel přemýšlet, co by to mohlo být. Nápad se zrodil. Dosti zamlžený, ale byl a viděl jsem ho dobře.

\section{Jako Holmes.}

Tak jsem se vrátil domů, sice jsem tam nebydlel, ale abyste to pochopili. Řekl jsem, že mám tušení, čím by to mohlo být způsobeno. Jen jsem je poprosil, aby mě pustili domů, že se na to podívám. Naštěstí s tím souhlasila, řekla taky, že raději bude čuchat i mě než ten divný pach. Málem jsem se z toho začervenal. Zamířil jsem do dílny a rozsvítil jsem si. Nic. Prostě tam nic nebylo, ale na druhou stranu ten smrad byl ještě více intenzivní. Nevěděl jsem, co by. Vypadalo to dost trapně, tak jsem radši odešel nenápadně, aby si nemysleli, že jsem jen kecal. Ale cestou mě viděli. Manželka se rozhodla, že si zatím pronajme byt a slitovala se na de mnou, že prý můžu jít taky spát pod střechu. Sice to tam pořád smrdělo, ale aspoň nepršelo. Pár dní později, jsem si zapnul televizi. Díval jsem se na spongeboba, když v tom to přerušili důležitými zprávami. Tentokrát byli ale fakt důležitý. Tam říkali, že neviditelný muž už vykradl třetí banku po sobě. Poslouchal jsem zřetelně, sice jsem chtěl akorát vědět, jestli se spongebob vrátí domů, nebo ne, ale to vypadalo dost záhadně. Říkali, že tento neviditelný muž vykrádá banky naprosto divným způsobem, nepřišlo mi to tak divné, akorát nevěděli, jak to dělá. Taky jsem to nechápal, ten chlápek tam prostě přišel vzal a odešel. Jediný, co bylo vidět, tak nebylo nic. Už podruhé jsem začal přemýšlet nad vším a spojovat si vše, co vím. Co když můj vynález nebyl propadák, co když jsem to prostě jen přepískl. To není možné přece, nebo jo. Že bych fakt zkrouhl vše pouze na jednu dimenzi. To by pak znamenalo, že se pohybuje pouze v pomyslné čáře, která nelze vidět. Pokud je to pravda, tak ten smrad musí být, no ovšem, ten pejsek. Protože prostě umřel, a to smrdí jak zdechlina. Dobré zprávy jsem hned pověděl bývalý manželce. Ta se ptala, co s tím má dělat, tak to jsem bohužel fakt nevěděl. Vlastně jsem to věděl. Potřeboval jsem pouze najít můj vynález a taky jsem věděl, kde hledat. Neviditelný muž byl můj nový cíl.

\chapter{Peníze neznamenají všechno}

\section{Miliónový pes}

Musel jsem se do toho vrhnout po hlavě. Cíl byl stanoven a já jsem se toho měl jenom držet. To se ale řekne najít člověka. Normálně by to bylo celkem lehké, ale když nevíte, jak vůbec vypadá a co nosí za čepici, popravdě to není zas tak podstatné. Přece jsem to nemohl vzdát, nějak jsem to udělat musel. Ze začátku jsem se bál dát informace policii, co kdyby mi sebrali můj nápad. Po pěti minutách mi došlo, jak je to absurdní. Zamířil jsem tedy k policii. Nejdřív si mysleli, že jsem nějaký bezdomovec, moc jsem se ani nedivil. Celkově jsem tam strávil deset dní, samozřejmě, že to bylo v rámci pouhého dne, ale oni byli úplně vygumovaní. Fakt mi nedošlo, proč nemůžou pochopit můj vynález a jak funguje. Vy jste to určitě pochopili, vlastně jsem vám to ani neřekl, ale řekněme, že máte tušení. Poté co to pochopili, tak mě poslali zpátky kam patřím a řekli mi ať už to neberu. Nevím, co přesně jsem neměl brát, zřejmě jen chtěli, abych veškeré vyšetřování nechal pouze na nich. To určitě, to by to dopadlo. Už jsme viděli jejich práci, přesněji tři vykradené banky. Musel jsem to vzít do vlastních rukou a taky jsem tak učinil. Tak začalo moje pátrání a moje plány, kde začnu hledat. Bohužel přes ten nehorázný smrad se nedalo přemýšlet. Tak jsem si řekl, nejdřív musím vyhodit toho psa. No jo, ale kde je. Asi si říkáte, přece v dílně, tak lehký to nebylo. Byl čas na oběd. Zašel jsem si do restaurace Místní, kam jinam. Cestou jsem potkal na ulici hovnocuc, víte takovou tu věc, kterou strčí do kanálu, a to vycucne splašky ven. To mi vniklo nápad, co kdybych toho pejska odfoukl, teda vyfoukl a zahrabal. Po dlouhém přemlouvání mi místní pracovníci jejich stroj stejně nepůjčili. Tak jsem si musel koupit vysavač, ale fakt hodně silný. Učinil jsem a zamířil hned domů. Sakra, zapomněl jsem na můj oběd. Musím holt počkat. Celou dílnu jsem vysál, vše radši dvakrát ne-li třikrát, když v tom něco škublo. Určitě to byl můj hafík. To by bylo pěkné, konečně bych se zbavil smradu a manželka by mi snad už i odpustila. Jenže to jsem takhle nemohl nechat být, přeci musí existovat nějaký způsob, jak ho vrátit do zpáteční velikosti a tvaru. Teď jsem hodně litoval vyhozeného vynálezu a samotných plánů. Neviditelný muž se asi má dost dobře, když jsem to vyhodil dohromady. Ale tak když to je můj vynález, tak to snad zvládnu znovu. To byste ani neřekli, ale neměl jsem ponětí, jak jsem to udělal. Naštěstí po pár hodinách, mě navštívil sám Bůh, tak jsem mu pak poděkoval za jeho pomoc. V tenhle moment jsem měl nápady zpět v hlavě. Hned jsem začal skládat, zřejmě si toho už i policisté všimli, dokonce třikrát během celé noci. Pokaždé jsem jim vysvětlil, že už na tom makám, oni se jen usmáli a řekli mi ať už hlavně nedělám takový hluk. Nechť byl můj vynález už podruhé zhotoven.

\section{Ve vatě}

Jakmile jsem to dodělal, tak jsem musel jen vymyslet, jak to udělat, aby to jelo na zpětný chod. To víte neviditelný chlápek to už určitě měl, jinak by neriskoval svůj život s takovou hračkou. Teď byla řada na mně. Samozřejmě já jako workoholik jsem neznal přestávek a na vlastní riziko jsem začal zkoušet. Proběhlo nespočet špatných pokusů, z pejska se stal třeba i dinosaurus a podobné, no raději byste to nechtěli vidět. Přeci jen jsem se nevzdával, záležel na tom celý svět, ale taky jsem myslel i na sebe samozřejmě. Mezitím totiž náš chlapík vykradl další dvě banky, takže vlastně pár. To jsem si nenechal líbit, ba dokonce mě to nutilo pracovat ještě tvrději. Samozřejmě, věřím, že jste v to věřili i vy, tak jsem to vymyslel. Bylo to celkem jednoduché, když jsem se na to díval. Pamatujete, jak je tam modrý a ten žlutý kabel, jak jsem říkal. Tak je stačilo jen přehodit. Nyní jsem neudělal stejnou chybu. Schoval jsem to do mého baťůžku a běžel k soudu, abych si to nechal patentovat. Proběhlo to hodně hladce. Rychlým obratem jsem to pustil i na velkovýrobu, ale mělo to jednu podmínky, jen aby to bylo bytelné, protože jinak by to mohl každý zneužít, topřece nechceme, sice to v tomhle případě stálo v čínský výrobě o trochu více, ale jak je to v staroangličtině: safety first. V tenhle moment jsem chodil na všechny banky, že pro ně mám vynález, který zabrání dalším krádežím. Všechny banky to ode mě koupili. Začalo se mi doopravdy dařit. To byste měli vidět, jak všichni chtěli zpátky ke mně. Doma už náš pejsek neležel, místo toho tam se mnou ležela jak manželka, tak i Drumlan. Už jsem byl šťastný, nic nemohlo být lepší.

\section{Nedotknutelný}

Hned po pár dnech byl náš pán chycen. No jo, nemá se měřit s nejlepším vynálezcem všech dob. Stejně to bylo divné on to ani nebyl neviditelný muž, jak se říkalo, protože to byla žena. To nikdo nečekal a zvlášť ona to nečekala. Peníze byli z části vráceny a žena zatčena a souzena. I přesto, že to bylo jednoznačné tak ji nedali do vězení, prý že nemají dostatek důkazů. Všechno to bylo jednoznačný a oni to pokazili, tupci. Musel jsem jim to hned vysvětlit, že si veškerý zisky takzvaně zneviditelnila a nechala část doma. Však pozdě, oni ji už pustili na svobodu a to znamená, že veškeré důkazy schovala. Tahle země je řízená snad úplnými blbci. Nešlo to jinak, už jsem nadále nad tím nepřemýšlel a raději jsem si počítal mé zisky, které bylo enormní a taky poctivě vydělaný, to se moc dneska nevidí, spolupráce s různými firmami se hrnuly ze všech stran. Opravdu, dokonce i ze spodu. Víc peněz jsem snad nikdy neměl, ale co bylo podstatnější, tak to bylo to, že jsem měl spoustu přátel, aspoň se tak tvářili. Byl jsem v euforii, a to i bez pomocných látek. Když v tom zčistajasna se ukázal pád. Přeci jsem vše nechal dělat z lepších materiálů, ale i přesto se někdo dostal dovnitř a změnil funkčnost zařízení. Na jednu stranu jsem byl celkem rád, protože to konečně zvládl za mě a vylepšil to pouze na dvě dimenze čili jednu přidal. Sice to vypadá, že to nic neměnilo, protože spousta lidí měla moje stroje, ale ty fungují trochu jinak. Mohlo to dopadnout ještě bez větších problémů, ale naše rychlá média vše vyžvanila a všichni o tom věděli. Kdybyste to jen viděli najednou všechno šlo do háje. Kradlo se všude kde to jenom šlo. Tohle byla hrůza. Skoro jako ta apokalypsa. Stejně jsem, jako vždy, vše chtěl vrátit na správnou míru, ale někdo mi ukradli všechny mé poznatky, a v ten moment už to vypadalo jako naprostý konec i pro mě.

\chapter{Normálně by to byli nemrtví, ale tohle.}

\section{Morálky konec}

Celé se to začalo hroutit čím dále, tím více. Veškeré země se s tím snažili něco udělat, ale víte, jak je to se státními pracovníky. Neznají ani pracovní čas, když se jedná o pauzy tak to ví přesně. Na druhou stranu to nebylo zas tak špatné. Nikdo neměl nic a zároveň všichni měli všechno. Zní to jako blbost, ale tohle byla realita, celkem krutá realita. Pořád to mělo spíše více nevýhod než výhod. Už jen to že každý měl svůj majetek tak maximálně den, pak si to vzal někdo jiný. Dokonce i mně to vadilo. Ani jsem nezískával peníze, takhle já je dostával, avšak víte, jak jsem to říkal. Po tom, co jsem si říkal, že to nemůže být horší, tak se to trochu zlepšilo a pak tak desetkrát zhoršilo. Už nikdo necítil vinu za své činy. Ze začátku to byli krádeže, už se ale zabíjí, a to celkem ve velkým. Všechny státy mění svoje předsedy a taky jestli to jsou diktatury či demokracie, popravdě mi přišlo zbytečné, stejně jsem volil demokracii. Smůla, nikdy nevyhráli. Byl čas si říct: A dost! V hlavě se mi začali mihnout myšlenky. Myslíte si, že to celé byla moje vina, nebo ne. Asi si říkáte, že ne, teda jo, to já si říkám ne. Mám pro to jednoduché odůvodnění, přece můj vynález byl smýšlen jako pomoc lidem, zřejmě to vypadá, že lidstvo si nic nezaslouží, ba dokonce zahubit.Najednou jsem byl spokojený, nevím, jestli to bylo tím, že jsem se zprostil viny, anebo zjistil, čeho jsou lidi schopni. Jak je to v starém přísloví: kdo chce kam, pomozme mu tam, nebo nevím, nějak tak. Moc jsem nedával pozor, když jsme se to učili. Pointa je však jasná. Jestli to celé nebylo předurčeno, abych já, Wenclaw, osvobodil Zemi od znečištěný, kterému se říká člověk.

\section{Záchrana}

Cíl jsem měl už daný. Zachránit Zemi. Nebo aspoň mě. Stroj by na to byl. Tím myslím, hned ten první, takže bych vše dal pouze do jedné dimenze a nikdo by to už neřešil. Ovšem tohle byl blbý plán, protože bych ani Zemi nepomohl. Stejně jsem nemohl nic dělat a vymýšlet, když mi to pořád někdo kradl. Takže jsem změnil taktiku. Přesunout se na jiné odlehlejší místo a tam to pořádně promyslet. Hned jsme sedli, tím myslím mě manželku a synka, na vlak takřka bez odporu, díky mně. Dojeli jsem na poslední zastávku. Moc lidí tam už nebylo. Bohužel pořád to nebylo nejideálnější místo, avšak nic lepšího nebylo možné. Tohle byl pouze takový mezikrok, mezi zemskou očistou od našeho nezkrotitelného viru. Nevěděl jsem co dál. Sice zde bylo lidí málo, ale o to hůře se chovali. Jakmile málem zabili Drumlana, špatná muška no, tak jsem se na to vykašlal. Žádná očista ještě nebude, nejdřív musíme zmizet, a to hodně daleko a tím nemyslím za devatero hor a řek, protože to je jen nějaká blbost. Když jsem se snažil odpočívat, tak si Drumlan hrál s kusem dřeva, jako kdyby to bylo letadlo. V ten moment to bylo více než zřejmé, že je to můj syn, takový dobrý nápad. Odletět na jiné místo by stále nestačilo, to znamená, že jsem jeho nápad jen trochu zlepšil a místo letadla dáme raketu a místo Země dáme Měsíc. V tu noc, když jsem pracoval na raketě se naštěstí přidala i manželka. Práce nám šla od ruky, jak jinak, když jsme nic jiného prakticky neměli. Během pár hodin to bylo připravené, a to jen díky tomu, že jsem se jako malý díval na poučné dokumenty, kde říkaly, jak si můžeme raketu klidně postavit i doma. Vzdělání na prvním místě. Noc byla hodně poklidná a tmavá. Neváhali jsme a nabrali směr Měsíc, samozřejmě ne s prázdnými kapsami. Měly jsme toho nepočítaje. Stále tohle byl ten lehčí krok z celé výpravy. Při nekonečné cestě jsme sledovali naši planetu. Ta už nebyla ani modrá ani zelená, spíše byla zbarvená do šeda, skoro až do černa. Nikdy bych to asi neříkal, ale i ten Měsíc vypadal více barevně. Už jsme se nemohli dočkat. Během této zdlouhavé cesty jsem zapojil moje mozkové závity a uvažoval, jak se usídlit na Měsíci, a hlavně jak přežít noc. Nebylo to moc složité pro mě. Prostě uděláme takovou báň, ve které se bude udržovat dobrá teplota.

\section{Osídlování na Měsíci}

Jakmile jsme se začali blížit k Měsíci, tak nám teprve začalo docházet, co jsme udělali, a hlavně, co chceme udělat. Už se nedalo uhnout, lidstvo se jen tak nenapraví. Každý má aspoň trochu špatné myšlení a nic jiného už nezbývá. Teď jsem si vzpomněl, že jsem vám zapomněl říct o tom, že na naší posádce necestujeme úplně sami, ale vzali jsem ještě našeho milého a taky drahého doktora, kdyby se něco stalo. Však už víte safety first. Vždycky jsem přemýšlel dopředu, než něco udělám, teď se to moc nevydařilo. Znovu to raději zopakuji, moje chyba to není. Teď už to bude všechno na nás, budeme muset začít všechno znova. To jen protože to Bůh neudělá. Nemám mu to za zlé, přeci je jenom trochu stár. Museli jsem diskutovat, jak a co uděláme, abychom vše stihli ještě za světla. Mezitím co jsem to už dopracovávali, tak jsem se blížili k přistání. Výhoda byla, že tu už jisté zázemí bylo, my jsme to museli jen trochu dodělat. Víte, kdysi dávno tu létali časté spoje, ale teď užto je nezajímavá turistická oblast. Hned co jsme přistáli, začali jsem makat jako nikdy. To byl fofr, vám povídám. V jeden moment jsem zaslechl nějaké zašustění. Ignoroval jsem to, zřejmě někomu něco spadlo. Za chvíli přišel synek, že někoho viděl. Dal jsem mu facku a řekl, že je to doktor. Nevychovaný zřejmě byl. Já jsem ho zas tak moc nevychovával, takže tohle taky nebyla moje chyba. Do třetice všeho dobrého i zlého, teď jsem ho viděl i já. Nějaký další člověk. Zřejmě byl vystrašený, my samozřejmě ne. Tak jsme si promluvili a zjistili, že je to Pepik ze školky. Toho jsem už dlouhou dobu neviděl a ani mi to moc nevadilo. Vždycky mi kradl moje hračky, ale já jsem to učitelkám říkal a oni, že nic neviděli. Momentálně to muselo jít stranou a aspoň nám mohl pomoct. Taky nám vlastně řekl, že se sem přemístil, aby šel pryč od lidí, protože co se tam děje je nehorázné. S tím jsem souhlasil, to jo. Díky dalšímu členovi našeho měsíčního výletu šlo všechno ještě rychleji. Samozřejmě to nebyl výlet na měsíc, ale na Měsíc. Předpokládám, aspoň doufám, že to chápete. Bylo ještě spoustu času, ale my jsme už dokončili naši báň. Čekal jsem, že to bude mnohem horší, ale nikomu to nevadilo. Teď zbývalo jenom domyslet můj plán na zničení lidstva, ale mohou si za to sami.

\chapter{Změnit či nezměnit názor}

\section{Změna je život}

Nejdříve jsme si museli rozmyslet, jestli tohle fakt chceme udělat, přeci jenom likvidace lidstva není jen taková sranda. Máme tam spousta známých i kamarádů, ale na druhou stranu i nepřátel. Mohlo by to být zajímavé, začít osídlovat Měsíc, jenom naší pomocí. Tady je hned další důvod, proč jsme vzali doktora. Mohli jsme vzít i porodní bábu, ale to by byla vhodná jen k jedné věci, doktor nás může ještě ošetřit. Co se pravděpodobně i stane, že se něco někomu stane. Sama věta zní, že to je prakticky nutnost než náhoda. Začali jsme debatovat ohledně tématu zničení lidstva. Sepisovali jsme kladné a záporné důsledky tohoto činu. Ze začátku to vypadá, jako hodně špatný nápad. Tolik let evoluce, sice si myslím, že to nebylo k ničemu, ale to je jen můj názor. A celý lidstvo to je celkem dost. Byla by to taková druhá potopa a my ten Noe, který zachránil zvířata, což jsme neudělali. To je fuk, přeci jen zvířata jsou na tom snad ještě hůř než lidstvo. Ale musíme dostatečně zdůraznit, že celé lidstvo se zbláznilo a je naprosto nekontrolovatelné. Co značilo jen jedno, Zničit lidstvo opravdu musíme. Je to naše nutnost a dobrý čin vůči samotnému lidstvu. Zní to jako velká blbost, což možná i je, ale snad to chápete. Sice se už chýlilo k našemu výsledku, který byl stejný jako na začátku, ale nesměli jsme nic uspěchat a ani jsme nemuseli. My jsme už v bezpečí byli, to ti na zemi ne, ale to nám bylo fuk. Už se nás to netýkalo, jediný, čeho jsem se bál byl Pepik, co když náš celý plán zničí. Moc ho už neznám a celkově se choval nějak divně a netradičně. To si představte, vždycky, když jsme měli oběd, který byl naprosto dokonalý, dalo by se říct gastronomický zážitek, tak si Pepik ani neříhnul. Jeno mi pak řekl, že je to nevkusné. Vám povídám, fakt divný to člověk. Nakonec jsme se rozhodli, nechat této otázce o pozemšťanech na později. Momentálně jsme měli v plánu prozkoumat okolí na Měsíci. Znáte to, vypadá to tady trochu jinak a člověk nechtěl přijít o místní, jak se vypráví, faunu a floru, která tu prakticky ani není.

\section{Wendrlach}

V následujícím týdnu se v ovzduší nacházelo takové nejasno. My jsme si ale řekli, že tomu ten týden dáme a potom každý řekne svůj názor, čili buď ano nebo ne. Jiná možnost se bohužel nevykytuje. Zato jsme měli spousta času v soukromí, kterého jsme následně dostivyužili, na Měsíc se přeci dostaneš jen párkrát za měsíc. To vám ale musím říci. Při jedné vycházce jsem tak skákal, jak je to vždycky v tom filmu, přesněji ve zpomaleném filmu. Ze začátku to byla celkem sranda, ale potom to akorát zdržovalo. Zpátky k příběhu. Takhle jsem se přibližoval ke kráteru, ale ten byl fakt obří. Jak jsem se, tak blížil, tak jsem to nějak nedopočítal a jak jsem byl málem u kraje, tak jsem ten kraj přeskočil, kdyby mě tak můj tělocvikář viděl, jak jsem daleko doskočil. Následně jsem padal. To taky trvalo, pokusím se vám to nějak zprostředkovat abyste to dostatečně pochopili. Tak jsem padal, padal, padal, padal, pak jsem usnul a pak zase probudil. Raději to zkrátím. No tak jsem dopadl, to byla rána jako z děla. Hned jsem upadl do bezvědomí, nikdo by to nezvládl takovou ránu. Musel jsem tam ležet ještě nějakou řádku hodin. Byla to celkem nuda, to vlastně asi nevím co. Dobře předpokládám, že to byla nuda. To pořád nebylo to nejzajímavější, což ani být nemohlo, ale spíš to, co následovalo. Normálně jsem se mezitím, no, jak bych to řekl, hm prostě vykonal potřebu. Tohle bylo spíše nechutné než zajímavé. Přesto to bylo možná i více výhodné. Můj trus přilákal jistého tvora. Těžko ho lehce popsat. Byl takový barevný, velký a malý zároveň a vypadal jako pes zkombinovaný s leguánem. Vypadal vystrašeně a šťastně zároveň. Rychle jsem si ho ochočil. Prakticky se mě nemohl pustit. Dovedl jsem ho domů. Tak jsme mu vymysleli jméno Wendrlach. Můžete zapřemýšlet proč zrovna tak, opravdu to má své opodstatnění. Zajímavější bylo, jak se rychle přizpůsobil našemu ovzduší. Rychle jsme si na něj zvykli a díky němu jsme taky přestali tak moc přemýšlet nad nejtěžší otázku našeho života.

\section{Den D, nebo B, těžko říct}

Díky Wendrlachovi týden uplynul jako voda, sice nevím, jak voda plyne, když to není plyn, ale kapalina. Nedá se svítit, prostě se to tak říká. Ještě by bylo vhodné říci, že to neuplynulo tak rychle jen díky Wendrlachovi, ale taky díky tomu, že jsem ho potkal v pondělí, což byl poslední den, před naším rozhodnutí. Otázku už znáte, zní relativně jednoduše, ale moc není, schválně se vás zeptám, co byste udělali na mém místě, kdybyste věděli, že celé lidstvo trpí, nikdo není v bezpečí a jediný dobrý nápad bylo celé lidstvo prostě zrušit, až tedy na nás, takovou elitu. Už to zní náročně, ne? Proto jsme tomu dali celý týden. Ne jenom víkend. Takže kdo všechno hlasoval: já, moje žena, doktor a Pepik. Ze začátku nám to nedošlo, ale byli jsme čtyři, což naznačovalo tomu, že to bude tzv. plichta. Taky že byla. Naštěstí, když jsme pokládali možnost zničení lidstva, tak už bylo jasné, že to nikdo nevyhraje. V ten moment začal ale Wendrlach hlasitě vyluzovat zvuky radosti. Takže bylo rozhodnuto. No není to vtipné, že celý osud zůstal pouze na zvuku, co udělal divný pesleguán. Mě to přišlo celkem dost komické. Tak jsem se smál. Možná až moc. Sice Pepik a doktor chtěli znovu hlasovat, ale to jsem zamítl. Nyní to bylo zřejmé musíme osvobodit pozemšťany, a tak je zahubíme. Dobře to pokračuje, bohužel nyní se muselo vymýšlet jak. Normálně za poslední dobu mé existence jsem přemýšlel více jak za celou dobu, co jsem byl ve škole. To je to moderní vyučování, naprosto k ničemu. Proto jsem byl spíše takový samouk. Vše, co jsem potřeboval, tak jsem se naučil, co mi přišlo nepodstatné tak ne. To je ovšem logické a nemusel jsem to tím pádem zdůrazňovat. Začalo se tedy uvažovat nad možnostmi. Každý pondělím jsme měli něco jako briefing. Tam pokládali naše návrhy.

\chapter{Bum}

\section{Prásk}

Po několika týdnech diskuzí ohledně způsobu zahubení lidstva, které byli někdy až hodně divné, jsme se rozhodl, ne fakt tam má být rozhodl, že zničíme lidstvo raketou. Řekl bych, že je prostě odpojíme od jejich životů. Proč by měli dál žít, když takhle dehonestují naše jména, vlastně naše ne a vlastně tu nikdo jiný než lidstvo a Wendrlach není, kašlu na to, zpět k tématu. Kde jsem to byl. Jo. Takže je zahubíme pomalým a bolestivým způsobem. Nebo ne. Počkat raketa, tak to spíše půjde rychle. Plán byl na stole, po kterým běhal Wendrlach, jak je roztomilý. A my jsme měli v plánu vytvořit a poslat na planetu Zemi několik jaderných raket. Přesný počet ještě nevíme. Něco jako šest, spíš sto, to je lepší. Jako vždycky, pustili jsme se do práce. Ovšem během noci jsem nemohl usnout. A když už jsem usnul, tak jsem měl divný sny. Takový, který už jsem někdy zažil. Ale bylo to celé jako v mlze. Všude něco pípalo a byla hrozná tma. Občas na mě někdo sahal a já s tím nemohl nic udělat. To bylo fakt nechutný. Ostatním jsem to raději neříkal, ještě by měli blbý otázky. Lepší bude si to nechat jenom pro sebe a zůstat jen se svými myšlenkami. Ty mě přece nemohou zradit, he. Zpět k ději. Už jsme pár, zase to nebyly dvě, ale víc, raket měli. Blížili jsme se ke konci. Už jsme prakticky viděli až na konec. K lepším zítřkům a k zapomenutým včerejškům. Chtěli jsme to odpálit. Já jsem byl v čele a chtěl to růžové tlačítko zmáčknout jako první. Opravdu bylo růžové, červenou jsme nenašli, no a co, máte s tím snad problém. Ale co to najednou to nešlo, moje ruka se zastavila. Nešlo s ní pohnout ani dopředu, ani dozadu. Všechno se začalo zasekávat a co je horší, kde je druhá a třetí část.

	
	\section{Zpět k tématu}
	
	
	Teď mi to už pomalu začíná docházet. To byl jen hloupý sen. I když nevím jestli bych hned měl říct, že byl hloupý. Dobře zápletka moc nedávala smysl a celý děj nebyl nějak zajímavý. No a co. Na to, že to vymyslel můj mozek během spánku je vskutku zajímavé. A kdybyste byli v mé kůži, tak to v tom snu i prožíváte přímo jako ve virtuální realitě. Poslední dobou mi přijde, že to co dokážu vymyslet za blbosti a to počítám i to co se odehrává v mých snech je docela dost fascinující a taky někdy hodně divné. Tohle je takové mezislovo a pak už dále budu pokračovat s jiným příběhem, který se mi buď zdál anebo jsem si ho kompletně vymyslel. Je ještě nutné dodat, že je snad všm jasné, že si několik věcí musím domyslet během psaní, ale to už nechám na čtenáři aby zjistil co to jsou za blbosti.
	
	\part{Kulka}

\section{Pandemie}

Je rok 2022 a zrovna začíná školní rok. To se mě už naštěstí netýká, protože už nějaký ten pátek (a podotkl bych, že nejenom pátek) nejsem ve škole. A raději se k tomuto systému už nikdy nepřiblížím. Možná si říkáte proč máš takový názor Emanueli. Tak zaprvé k tomu se dostanu a za druhé nejmenuji se Emanuel ale jen Em. Proč tak blbé jméno, no to je otázka ne úplně mířená na mě nemyslíte si? Už jsem se zase ztratil. Tedy proč nemám rád školu je celkem jednoduché. Ve škole jsem nikdy neměl problémy co se týče prospěchu, akorát moji spolužáci stáli za velké kulové. O tom bych raději už nechtěl nikdy mluvit. Doufám, že to dokážete pochopit. Jo a taky to je tím, že už jsem na škole strávil tolik času. To víte když studujete jako medik, tak na škole strávíte hodně let. Aspoň teď už to nemusím řešit. Ale jako medik jsem se pustil do zkoumání virů a hledání léků a vakcín. Bohužel poslední roky to bylo příšerné, jakmile z Číny přišel virus do celého světa. A pak to bylo samé “Rychle potřebujeme vakcínu!!” a takové kecy. Jakmile se nám to povedlo, tak to zase bylo zleva a zprava “To bylo moc rychlé. Já vám nevěřím!”. Tak víte co. Příště se na vás mohu vykašlat. No jo no. Ale pak mě to napadlo. Co takhle to pořádně vyzkoušet. A udělat nějaký pokus. Ale musím to pořádně promyslet aby to stálo za to. A taky abych neublížil moc lidem. Ale nemusíte se bát tohle je taková “hra”. I když asi to fakt nedopadne moc dobře. No snad lidi zjistí jak jsou hloupí ve správný čas. Jinak to bude ehm tragédie. Tedy teď si říkáte Eme proč chceš dělat pokusy na lidech? To se musím ohradit. Já dělám spíše pokus na společnost a komunitní hloupost než na samotných lidech.

\section{Plán}

Teď jak na to? No je to prosté. Vezmeme si nějaký virus, který má fatální následky a předem najdeme vakcínu. Protože chceme udělat takový experiment, tak vakcínu pak pojmenujeme “Kulka”, aby to znělo strašidelně. Abych se pochlubil, tak na tomhle projektu pracujeme už delší dobu a dost věcí už máme předpřipravené. Jedná se o virus, který dokáže převzít moc nad člověkem. Je to vlastně takový parazit, který převezme celé tělo, respektive je to trochu komplikovanější, ale opustím od doktorské hantýrky, abyste tomu rozuměli. Už jsem slyšel spoustu divných názoru a ne není to žádné zombie. Jsme snad v nějakém béčkovém filmu. Na to nikdo nevěří. Souběžně jsme pracovali na vakcíně. Co je hodně zajímavé je, že vakcína funguje i jako lék a tedy co je člověk napaden, tak může být vyléčen i když byl dost hloupý na to nevzít si vakcínu. To teprve bude správná lekce. Nemyslíte si? Vše máme plně pod kontrolou a nemůže se určitě nic stát, no vlastně jenom lid nemáme pod kontrolou, ale ty chceme zkoumat. Teď je třeba najít pacienta nula u kterého začneme. Pak následně musíme rychle izolovat danou osobu a “hledat vakcínu”. Nicméně už bude vše hotové a v přímém přenose ukážeme jak funguje vakcína a vysvětlíme jak všechno funguje. Následně po tom co by se všichni naočkovali, nebo aspoň měli na to čas, tak virus kontrolovaně vypustíme a uvidíme co se následně bude dít. Jakmile vše v pořádku přejde, tak lidé začnou konečně důvěřovat ve věci, které jsou podpořené vědeckými podklady. Myslím, že to je dokonalý plán. Teď byste možná ještě chtěli vědět, jak nemoc probíhá. Ze začátku člověk začne mít bolesti hlavy a následně pak i zažívací problémy. Celá se většinou projeví až tak za měsíc až dva. Potom kognitivní část mozku odumírá a virus převládá moc nad tělem. Jo a málem bych zapomněl. To, že vakcína může být použita jako lék znamená, že to musí být dříve než se projeví poslední část nemoci, tedy převzetí kontroly nad tělem.

\section{Začátek plánu}

Jak najít nějakého dobrovolníka? To byla naše otázka, na kterou jsme nemohli najít odpověď. Už jsme vyvěsili spousty plakátů a inzerátů, ale nikdo se nehlásí. Možná to je tím, jak jsme to tam napsali, ale proč by to snad někomu mělo vadit. Stojí tam "Cítíte se býti morčetem ve svém životě, tak pojďte aspoň podpořit vědu, aby na Vás byl veden důležitý výzkum." Pěkné, ne? Ale jak říkala moje sestra Karla, je to takové děsivé a možná i přidává na nedůvěře medicíně. To ale vlastně chceme vyzkoušet, takže to je vlastně dobře. Tak už vás nebudu trápit naším vlastně primitivním problémem. Jediné co stačilo přidat je, že je každý člověk získá peněžitou odměnu, ale nemyslete si, že to byla nějaká velká odměna spíše takové drobné na autobus, aby se sem vůbec dostali. Přeci jen ten lidský mozek často bývá dosti primitivní a můžeme očekávat, jak se budu v daných situacích chovat a tedy pokud nabídnete něco zadarmo, tak hned máte spoustu zákazníků. Tedy už nemáme žádného pacienta nula, ale rovnou celou skupinu zájemců. Je tedy pravdou, že půlka z nich vypadá jako lidé bez domova, kteří sem možná zabloudili, ale na to už se nemůžeme dívat a řešit to. Teď se vlastně nemoc bude šířit po různých *hordách* občanů a tak to můžeme hlídat. Pravděpodobně nechápete, proč to vím. Je to prosté, pořád jsem nevysvětlil každý aspekt nemoci a taky ani dané vakcíny. Nemoc se šíří relativně dost pomalu a pokud má člověk už nemoc, ale i vakcínu, tak je nadále šiřitelem a už se mu nic nemůže stát. Tak náš ďábelský plán může začít "Muhahaha", tedy já samozřejmě nejsem zlý, to byl jen vtip, opravdu. Začněme tedy náš experiment. Všechny občany z *hordy* nula naočkujeme a za měsíc jim dáme nákazu. Díky tomuto postupu se jim nic nemůže stát, ale budeme mít nemoc rozšiřující se společností.

\chapter{Vakcinace}

\section{Světoví medici}

Jméno? "Buč, Buč Teodor". Datum narození? "14.05.1984" Tak tady vám předávám nemoc a následně i daný peněžitý obnos. "Děkuji, nashle." Neshledanou. Marku prosím tě, to už byl poslední? "Jo." Tak vidíte sami už první část proběhla. Teď si dáme týden pauzu a pak dáme rychle vědět všem osobnostem a to od politiků až po známé herce a novináře. Nemůžeme přece říct, že jsme našli virus hned, když se ani nerozvinul mezi občany. V tomhle týdnu aspoň budeme mít čas dále vytvářet naši vakcínu. Protože mezi první hordou byli i lidé majetní, kteří nejspíše budou cestovat, tak je třeba upozornit i další státy. No ale nemůžeme jim to říci dříve než nám, takže musíme nejdříve obvolat všechny lékařské asociace a předat jim dané informace o našem průzkumu a protože našim primárním plánem není zbohatnout, tak jim zašleme také jak vyrábět naši vakcínu, aby se také připravili na možnou pandemii. Již máme připraven seznam, komu všemu se ozveme, ale protože je nudné věše opakovat, tak spíše sjednáme hromadný hovor. Samozřejmě neleníme a už všichni mají elektronickou pozvánku k danému termínu sněmu, teda jestli to mohu označit sněmem, ale zní to dobře a profesionálně a to já samozřejmě taky jsem, takže je to sněm. Proč mi někdo klepe na dveře. Dále. "Eme, myslím, že tu máme menší problém." Jo jasně ty tvoje vtípky jsou fakt dobré Marku. Vždycky se rád zasměji, ale nemohlo by to počkat na apríla. "Bohužel Eme, ale ani to není vtip." Co? To mě velice znepokojuješ. Snad to není vážné, tak rychle, co to je? "Spousta lidí odmítlo pozvánku, protože nemají čas." Jak jako nemají čas, vždyť je to vážné! Přece jsem na tom dostatečně pracovali, aby pochopili, že to není sranda. Do prdele! Musíme jim to rychle oznámit. Tak se všemi, kdo může zařiďte hovor a musíme sepsat i návod na to, co se děje. Co když pak nebudou vakcíny a žádné informace. Celé se to může pokazit. Tak co tu furt stojíš Marku, dělej! "Jasný, už běžím."

\section{Sněm}
	
	\part{Pavouci}

\section{Není}

Končí v šichtě kopáčů. Pak už sen s pavouky a stolním fotbálku.
\end{document}