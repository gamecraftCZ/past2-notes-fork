% Parameters of the paper.
\documentclass[12pt,a4paper]{report}
\setlength\textwidth{160mm}
\setlength\textheight{247mm}
\setlength\oddsidemargin{0mm}
\setlength\evensidemargin{0mm}
\setlength\topmargin{0mm}
\setlength\headsep{0mm}
\setlength\headheight{0mm}
\let\openright=\clearpage

\usepackage[czech]{babel}  % Czech
%\usepackage{babel}          % English
\usepackage{lmodern}        %
\usepackage[T1]{fontenc}    % Change the font
\usepackage{textcomp}       %

\usepackage[utf8]{inputenc} % Coding.

%%% Další užitečné balíčky (jsou součástí běžných distribucí LaTeXu)
\usepackage{amsmath}        % rozšíření pro sazbu matematiky
\usepackage{amsfonts}       % matematické fonty
\usepackage{amsthm}         % sazba vět, definic apod.
\usepackage{bbding}         % balíček s~nejrůznějšími symboly
\usepackage{bm}             % tučné symboly (příkaz \bm)
\usepackage{graphicx}       % vkládání obrázků
\usepackage{fancyvrb}       % vylepšené prostředí pro strojové písmo
\usepackage{indentfirst}    % zavede odsazení 1. odstavce kapitoly
\usepackage{natbib}         % zajištuje možnost odkazovat na~literaturu
\usepackage[nottoc]{tocbibind} % zajistí přidání seznamu literatury,
                            % obrázků a tabulek do~obsahu
\usepackage{icomma}         % inteligetní čárka v~matematickém módu
\usepackage{dcolumn}        % lepší zarovnání sloupců v~tabulkách
\usepackage{booktabs}       % lepší vodorovné linky v~tabulkách
\usepackage{paralist}       % lepší enumerate a itemize
\usepackage{xcolor}         % barevná sazba
                            %
\usepackage{caption}        % Packages pro subfigures.
\usepackage{subcaption}     %
\usepackage{pdfpages}       % Vkladani pdfka.
\usepackage{hyperref}       % Odkazy.
\usepackage{tikz}           % Grafy a obrázky.
\usepackage{algpseudocode}  % Pseudokód
\usepackage{algorithm}
\usepackage{titling}
\usepackage{amssymb}

%%% Tento soubor obsahuje definice různých užitečných maker a prostředí %%%
%%% Další makra připisujte sem, ať nepřekáží v ostatních souborech.     %%%

%%% Drobné úpravy stylu

% Tato makra přesvědčují mírně ošklivým trikem LaTeX, aby hlavičky kapitol
% sázel příčetněji a nevynechával nad nimi spoustu místa. Směle ignorujte.
\makeatletter
\def\@makechapterhead#1{
  {\parindent \z@ \raggedright \normalfont
   \Huge\bfseries \thechapter. #1
   \par\nobreak
   \vskip 20\p@
}}
\def\@makeschapterhead#1{
  {\parindent \z@ \raggedright \normalfont
   \Huge\bfseries #1
   \par\nobreak
   \vskip 20\p@
}}
\makeatother

% Toto makro definuje kapitolu, která není očíslovaná, ale je uvedena v obsahu.
\def\chapwithtoc#1{
\chapter*{#1}
\addcontentsline{toc}{chapter}{#1}
}

% Trochu volnější nastavení dělení slov, než je default.
\lefthyphenmin=2
\righthyphenmin=2

% Zapne černé "slimáky" na koncích řádků, které přetekly, abychom si
% jich lépe všimli.
\overfullrule=1mm

%%% Makra pro definice, věty, tvrzení, příklady, ... (vyžaduje baliček amsthm)

\theoremstyle{plain}
\newtheorem{veta}{Věta}
\newtheorem{lemma}[veta]{Lemma}
\newtheorem{tvrz}[veta]{Tvrzení}

\theoremstyle{plain}
\newtheorem{definice}{Definice}
\newtheorem*{pozor}{Pozorování}
\newtheorem*{cvic}{Cvičení}
\newtheorem*{fakt}{Fakt}

\theoremstyle{remark}
\newtheorem*{dusl}{Důsledek}
\newtheorem*{pozn}{Poznámka}
\newtheorem*{prikl}{Příklad}

\theoremstyle{plain}
\newtheorem{thm}{Theorem}
%\newtheorem{lemma}[thm]{Lemma}
\newtheorem{claim}[thm]{Claim}

\theoremstyle{plain}
\newtheorem{defn}{Definition}
\newtheorem*{observ}{Observation}
\newtheorem*{exerc}{Exercise}
\newtheorem*{fact}{Fact}

\theoremstyle{remark}
\newtheorem*{cor}{Corollary}
\newtheorem*{rem}{Remark}
\newtheorem*{example}{Example}


%%% Prostředí pro důkazy

\newenvironment{dukaz}{
  \par\medskip\noindent
  \textit{Důkaz}.
}{
\newline
\rightline{$\qedsymbol$}
}

\newenvironment{myproof}{
	\par\medskip\noindent
	\textit{Proof}.
}{
	\newline
	\rightline{$\qedsymbol$}
}


%%% Prostředí pro sazbu kódu, případně vstupu/výstupu počítačových
%%% programů. (Vyžaduje balíček fancyvrb -- fancy verbatim.)

\DefineVerbatimEnvironment{code}{Verbatim}{fontsize=\small, frame=single}

%%% Prostor reálných, resp. přirozených čísel
\newcommand{\R}{\mathbb{R}}
\newcommand{\N}{\mathbb{N}}
\newcommand{\Z}{\mathbb{Z}}

%%% Užitečné operátory pro statistiku a pravděpodobnost
\DeclareMathOperator{\pr}{\textsf{P}}
\DeclareMathOperator{\E}{\textsf{E}\,}
\DeclareMathOperator{\var}{\textrm{var}}
\DeclareMathOperator{\sd}{\textrm{sd}}

%%% Příkaz pro transpozici vektoru/matice
\newcommand{\T}[1]{#1^\top}

%%% Vychytávky pro matematiku
\newcommand{\goto}{\rightarrow}
\newcommand{\gotop}{\stackrel{P}{\longrightarrow}}
\newcommand{\maon}[1]{o(n^{#1})}
\newcommand{\abs}[1]{\left|{#1}\right|}
\newcommand{\dint}{\int_0^\tau\!\!\int_0^\tau}
\newcommand{\isqr}[1]{\frac{1}{\sqrt{#1}}}

%%% Vychytávky pro tabulky
\newcommand{\pulrad}[1]{\raisebox{1.5ex}[0pt]{#1}}
\newcommand{\mc}[1]{\multicolumn{1}{c}{#1}}


% set up \maketitle to accept a new item
\predate{\begin{center}\placetitlepicture\large}
	\postdate{\par\end{center}}

% commands for including the picture
\newcommand{\titlepicture}[2][]{%
	\renewcommand\placetitlepicture{%
		\includegraphics[#1]{#2}\par\medskip
	}%
}
\newcommand{\placetitlepicture}{} % initialization

 % Use global macros.

\title{Teorie množin}
\author{Tomáš Turek}
\titlepicture[width=5in]{res/venn}
%\date{}

\begin{document}
	\maketitle
	
	\textit{Poznámka: Následující text jsou moje osobní zápisky z Teorie množin z roku 2021-2022. V textu se můžou vyskytovat jak gramatické chyby, tak i technicé chyby (jako ne zcela správný důkaz apod.), tím pádem berte text jako doplňek přednášky.}
	
	\tableofcontents
	
	\chapter{Úvod}

\section{Jazyk teorie množin}

Jazyk teorie $x \in Y$. Také se bude používat *metajazyk* jako například: “definovat”, “formule” a “třída”.

\subsection{Symboly}

\begin{itemize}
	\item Proměnné pro množiny $X,Y,Z,x_{1},x_{2}, \dots$.
	\item Binární predikátový (relační) symbol $=$ a taky $\in$ (náležení).
	\item Dále také logické spojky: $\neg ,\land ,\lor ,\rightarrow , \leftarrow (\Leftarrow , \Rightarrow)$.
	\item Také kvantifikátory: $\forall \text{ a } \exists$.
	\item Samozřejmě i závorky $(), []$.
\end{itemize}

\subsection{Formule}

Atomické formule $x = y \text{ a } x \in y$.

\begin{enumerate}
	\item Jsou-li $\varphi, \psi$ formule, pak $\neg \varphi , \varphi \lor \psi , \varphi \land \psi , \varphi \rightarrow \psi , \varphi \leftrightarrow \psi$ jsou také formule (popřípadě i uzávorkované).
	\item Je-li $\varphi$ formule, pak $(\forall x) \varphi \text{ a } (\exists x)\varphi$ jsou také formule.
\end{enumerate}

Každá formule pak lze dostat z atomických formulí konečně mnoha pravidly 1 a 2.

\subsection{Rozšíření jazyka (zkratky)}

\begin{itemize}
	\item $x \neq y$ je pro $\neg (x = y)$.
	\item $x \notin y$ je pro $\neg (x \in y)$.
	\item $x \subseteq y$ je pro "$x$ je podmnožina y" $(\forall u)(u \in x \rightarrow u \in y)$.
	\item $x \subset y$ je pro "$x$ je vlastní podmnožina" $(x \subseteq y \land x \neq y)$.
\end{itemize}

\textit{Cvičeni: Napište formulí “množina $x$ je prázdná”.}

\section{Axiomy logiky (“jak se chovají logické symboly”)}

Axiomy výrokové logiky např.: schéma axiomů: Jsou-li $\varphi , \psi$ formule, pak

$$
\varphi \rightarrow (\psi \rightarrow \varphi)
$$

je **axiom**.

Axiomy predikátové logiky např.: Schéma axiomů: Jsou-li $\varphi, \psi$ formule, $x$ proměnná, která není volná ve $\varphi$, pak

$$
(\forall x) (\varphi \rightarrow \psi) \rightarrow (\varphi \rightarrow (\forall x)\psi)
$$

je axiom.

Axiomy pro rovnost:

\begin{itemize}
	\item $x$ je proměnná, pak $x=x$ je axiom.
	\item $x,y,z$ jsou proměnné, $R$ je relační symbol, pak
\end{itemize}

$$
(x=y) \rightarrow (\forall z)(R(x,z) \leftrightarrow R(y,z))
$$

$$
(x=y) \rightarrow (\forall z)(x \in z \leftrightarrow y \in z)
$$

$$
(x=y) \rightarrow (\forall z)(z \in x \leftrightarrow z \in y)
$$

Odvozovací pravidla:

\begin{itemize}
	\item Z $\varphi, \varphi \rightarrow \psi$ odvoď $\psi$.
	\item Z $\varphi'$ odvoď $(\forall x)\varphi$.
\end{itemize}
	\chapter{Axiomy teorie množin}

“Jak se chová $\in$.” “Jaké množiny existují.”

\textit{Zermelo-Fraenkelova teorie}, zkráceně \textbf{ZF} má celkem 9 axiomů (resp. 7 axiomů a 2 schémata). Pak je ještě 10.axiom výběru (\textbf{AC}) to pak je \textbf{ZF+AC=ZFC}.

\section{1.Axiom existence množin}

“Existuje množina.”

$$
(\forall x)(x=x)
$$

\section{2.Axiom extenzionality}

Udává souvislost mezi $\in \text{ a } =$. “Množina je určena svými prvky.”

$$
(\forall z)(z \in x \leftrightarrow z \in y) \rightarrow x = y
$$

\textit{Cvičeni: Dokažte $((x \subseteq y) \land (y \subseteq z)) \rightarrow x \subset z$.}

\section{3.Schéma axiomu vydělení}

Je-li $\varphi(x)$ formule, která neobsahuje volnou proměnnou $z$. Pak:

$$
(\forall a)(\forall x)(\exists z)(x \in z \leftrightarrow (x \in a \land \varphi(x)) 
$$

je axiom.

“Z množiny $a$ vybereme prvky s vlastností $\varphi(x)$ a ty vytvoří novou množinu $z$.” Díky axiomu extenzionality je taková $z$ právě jedna.

\subsection{Značení:}

\begin{itemize}
	\item $\{x; x \in a \land \varphi(x)\}$ je zkrácení.
	\item $\{x \in a; \varphi(x)\}$ "Množina všech prvků $a$ splňující $\varphi(x).$"
\end{itemize}

\begin{definice}
	\begin{itemize}
		\item Průnik: $a \cap b$ je $\{x, x \in a \land x \in b\}$.
		\item Rozdíl: $a \setminus b$ je $\{x, x \in a \land x \notin b\}$
	\end{itemize}
\end{definice}

\textit{Cvičení:}

\begin{itemize}
	\item \textit{Napište formulí “množina $a$ je jednoprvková”.}
	\item \textit{Dokažte, že množina všech množin neexistuje.}
\end{itemize}

\section{4.Axiom dvojice}

$$
(\forall a)(\forall b)(\exists z)(\forall x)(x \in z \leftrightarrow (x=a \lor x=b))
$$

“(Ne)každým dvěma množinám $a,b$ existuje množina $z$, která má za prvky právě $a,b$.”

\begin{definice}
	\begin{itemize}
		\item $\{a,b\}$ je \textbf{neuspořádaná dvojice} množin $a,b$, jakožto dvouprvková množina s prvky $a,b$ (pokud $a \neq b$).
		\item $\{a\}$ znamená $\{a,a\}$, nebo-li jednoprvková množina s prvkem $a$.
	\end{itemize}
\end{definice}

\begin{prikl}
	Můžeme vytvořit $\{\emptyset\},\{\{\emptyset\}\}, \{\emptyset,\{\emptyset\}\}, \dots$.
\end{prikl}

\textit{Cvičení: Dokažte $(\forall z)(x \in z \leftrightarrow y \in z) \rightarrow x = y$.}

\begin{definice}
	$(a,b)$ je \textbf{uspořádaná dvojice} množin $a,b$. To je pak množina $\{\{a\},\{a,b\}\}$
\end{definice}

\begin{pozn}
	Pro $a = b$ je $(a,b) = \{\{a\},\{a,a\}\} = \{\{a\},\{a\}\} = \{\{a\}\}$.
\end{pozn}

\begin{lemma}
	$$
	(x,y) = (u,v) \leftrightarrow (x = u \land y = v)
	$$
\end{lemma}

\begin{proof}
	\begin{itemize}
		\item $\leftarrow$
		\item $\{x\} = \{u\}$ plyne z axiomu extenzionality.
		\item $\{x,y\} = \{u,v\}; \{\{x\},\{x,y\}\} = \{\{u\},\{u,v\}\}$
		\item $\rightarrow$
		\item $\{\{x\},\{x,y\}\} = \{\{u\},\{u,v\}\}$ to pak znamená, že $\{x\} = \{u\} \lor \{x\} = \{u,v\}$ kde v obou případech $x=u$.
		\item $\{u,v\} = \{x\} \lor \{u,v\} = \{x,y\}$ tedy $v = x \lor v = y$
		\item Pokud $v=x$ pak z $x = u$ plyne, že $v=u=x$.
	\end{itemize}
\end{proof}

\begin{definice}
	Jsou-li $a_{1},a_{2},a_{3}, \dots ,a_{n}$ množiny, definujeme \textbf{uspořádanou $n$-tici} \newline $(a_{1},a_{2},a_{3}, \dots ,a_{n})$. Následně $(a_{1})$ znamená $a_{1}$ a je-li definována $(a_{1}, \dots ,a_{k})$ pak \newline $(a_{1}, \dots ,a_{k}, a_{k+1})$ je $((a_{1}, \dots ,a_{k}), a_{k+1})$.
\end{definice}

\begin{lemma}
	$$
	(a_{1},a_{2},a_{3}, \dots ,a_{n}) = (b_{1},b_{2},b_{3}, \dots ,b_{n}) \leftrightarrow (a_{1} = b_{1} \land \dots a_{n} = b_{n})
	$$
\end{lemma}

\begin{proof}
	Jako cvičení.
\end{proof}

\section{5.Axiom sumy (axiom of the union)}

$$
(\forall a)(\exists z)(\forall x)(x \in z \leftrightarrow (\exists y)(x \in y \land y \in a))
$$

\begin{definice}
	$\bigcup a$ je \textbf{suma} množiny $a$. Tzn “$\{x, (\exists y)(x \in y \land y \in a)\}$”.
\end{definice}

Pozorování: Pokud $a = \{b,c\}$, pak $\bigcup \{b,c\} = \{x, x \in b \lor x \in c\}$.

\begin{definice}
	$b \cup c$ je $\bigcup \{b,c\}$ sjednocení množin $b,c$.
\end{definice}

\begin{definice}
	Jsou-li $a_{1}, \dots a_{n}$ množiny, definujeme \textbf{neuspořádanou $n$-tici} $\{a_{1}, \dots a_{n}\}$ ($n$-prvkovou množinu, pokud každé $a_{i}$ je různé) rekurzivně. Je-li definovaná $\{ a_{1}, \dots a_{k}\}$ pro $k \geq 2$, pak $\{ a_{1}, \dots a_{k}, a_{k+1}\}$ je $\{ a_{1}, \dots a_{k}\} \cup \{a_{k+1}\}$.
\end{definice}

\section{6.Axiom potence (power set, potenční množina)}

$$
(\forall a)(\exists z)(\forall x)(x \in z \leftrightarrow x \subseteq a)
$$

“Existuje množina $z$ jejichž prvky jsou právě podmnožiny množiny $a$.”

\begin{definice}
	$\mathcal{P}(a)$ je “$\{x; x \subseteq a\}$” potenční množina $[2^{a}]$ množiny $a$ (potence $a$).
\end{definice}

\begin{prikl}
	$\mathcal{P}(\emptyset) = \{\emptyset\}$ a $\mathcal{P}(\{\emptyset\}) = \{\emptyset, \{\emptyset\}\}$.
\end{prikl}

\textit{Cvičení: Co je $\mathcal{P}(\bigcup a)$ a jestli $\bigcup (\mathcal{P}(a)) = a$?}

\section{7.Schéma axiomu nahrazení}

“Obraz množiny funkcí je množina.”

Je-li $\psi(u,v)$ formule, která neobsahuje volné proměnné $w,z$, pak

$$
(\forall u)(\forall v)(\forall w)((\psi(u,v) \land \psi (u,w)) \rightarrow v = w) \rightarrow (\forall a)(\forall z)(\forall v)(v \in z \leftrightarrow (\exists u)(u \in a \land \psi(u,v)))
$$

je axiom.

\begin{itemize}
	\item “Je-li $\psi$ funkce (částečná) určená formulí: $\psi (u,v) \text{ je } f(u)=v$, pak obrazem $a$ touto funkcí je opět množina (z).”
	\item Také implikuje schéma vydělení: $\varphi (u) \land u = v$.
	\item Poznámka: \textit{transfinitní rekurze, konstrukce $\omega + \omega$, Zornovo lemma, věta o dobrém uspořádání.}
\end{itemize}

\section{8.Axiom fundovanosti (foundation, regularity)}

$$
(\forall a)(a \neq \emptyset \rightarrow (\exists x)(x \in a \land x \cap a = \emptyset))
$$

“Každá množina má prvek, který je s ní disjunktní.”

\textit{Cvičení: Ukažte, že Axiom fundovanosti zakazuje existenci konečných cyklů relace $\in$. Tedy množiny $y$ takové, že $y \in y$, ale i $y_{1},y_{2}, \dots ,y_{n}$ takové, že $y_{1} \in y_{2} \in \dots \in y_{n} \in y_{1}$.}

Díky axiomu fundovanosti lze všechny množiny vygenerovat z prázdné množiny operacemi $\mathcal{P}, \bigcup$.
	\chapter{Třídy}

\begin{definice}
	$\varphi(x)$ je formule a $\{x; \varphi(x)\}$ označuje “seskupení” množin, pro které platí $\varphi(x)$.
\end{definice}

\begin{itemize}
	\item Pokud $\varphi(x)$ je tvaru $x \in a \land \psi(x)$, pak je to množina (axiom vydělení).
	\item $\{x; \varphi(x) \}$ je třídový term, soubor které označuje je \textbf{třída} určená formulí $\varphi(x)$.
	\item “Definovatelný soubor množin.”
	\item Je-li $y$ množina, pak $y = \{x; x \in y \land x = x\}$ je třída.
	\item Tedy každá množina je i třída.
	\item \textbf{Vlastní třída} je třída, která není množinou.
\end{itemize}

\section{Rozšíření jazyka:}

\begin{itemize}
	\item Ve formulích na místě volných proměnných připustíme třídové termy.
	\item Navíc proměnné pro třídy jsou $X,Y, \dots$ (nebude možné je kvantifikovat).
\end{itemize}

\section{Atomické proměnné}

\begin{itemize}
	\item $x = y, x \in y, x = X, x \in X, X \in x, X=Y, X \in Y$
	\item Plus ještě výrazy vzniklé nahrazením $\{x, \varphi(x)\}$ za $x$ a $\{y, \varphi(y)\}$ za $y$.
	\item Ostatní formule rozšířeného jazyka vznikají pomocí logických spojek $(\neg, \lor, \land, \leftarrow, \rightarrow, \leftrightarrow)$ a kvantifikací množinových proměnných $((\forall x)\dots(\exists y)\dots)$.
	\item Formule s třídovými termy bez třídových proměnných označován jako “zkrácený zápis” formule základního jazyka.
	\item Formule s třídovými proměnnými označované jako “schéma formulí” základního (popř. rozšířeného) jazyka.
\end{itemize}

\section{Eliminace třídových termů}

$x,y,z,X,Y$ jsou proměnné a $\varphi(x), \psi(x)$ formule základního jazyka. $X$ zastupuje \newline $\{x, \varphi(x)\}$ a $Y$ zastupuje $\{y, \varphi(y)\}$.

\begin{enumerate}
	\item $z \in X$ zastupuje $z \in \{x, \varphi(x)\}$.
	\begin{itemize}
		\item "$z$ je prvkem třídy všech množin, splňující $\varphi(x)$."
		\item Nahradíme: $\varphi (z)$.
	\end{itemize}
	\item $z = X$ zastupuje $z = \{x, \varphi(x)\}$.
	\begin{itemize}
		\item “Množina $z$ se rovná třídě $X$.”
		\item Nahradíme: $(\forall u)( u \in z \leftrightarrow \varphi(u))$.
	\end{itemize}
	\item $X \in Y$ zastupuje $\{x, \varphi(x)\} \in \{y, \psi(y)\}$.
	\begin{itemize}
		\item Nahradíme: $(\exists u)(\forall v)((v \in u \leftrightarrow \varphi (v)) \land \psi(u))$.
	\end{itemize}
	\item $X \in y$ zastupuje $\{x, \varphi(x)\} \in y$.
	\begin{itemize}
		\item Nahradíme: $(\exists u)(\forall v)((v \in u \leftrightarrow \varphi (v)) \land u \in y)$.
	\end{itemize}
	\item $X = Y$ zastupuje $\{x, \varphi(x)\} = \{y, \psi(y)\}$.
	\begin{itemize}
		\item Nahradíme: $(\forall u)(\varphi(u) \leftrightarrow \psi(v))$
	\end{itemize}
\end{enumerate}

Meta pozorování: Formule rozšířeného jazyka určují stejné třídy jako formule základního jazyka. Příklad $\{x; x \notin \{z, \psi(z)\}\} \rightarrow \{x; \neg \psi(x)\}$.

\section{Třídové operace}

\begin{definice}
	\begin{itemize}
		\item $A \cap B$ je $\{x, x \in A \land x \in B\}$.
		\item $A \cup B$ je $\{x, x \in A \lor x \in B\}$.
		\item $A \setminus B$ je $\{x, x \in A \land x \notin B\}$.
		\item Pokud $A = \{x, \varphi(x)\}$ a $B = \{y, \psi(y)\}$, pak $A \cap B = \{z, \varphi(z) \land \psi(z)\}$.
	\end{itemize}
\end{definice}

\begin{definice}
	$\{x; x = x\}$ je \textbf{univerzální třída}, která se značí jako $V$.
\end{definice}

\begin{itemize}
	\item $A$ je třída, (absolutní) doplněk $A$ je $V \setminus A$, který se značí jako $-A$.
	\item $A \subseteq B, A \subset B$ značí, že $A$ je podtřídou $B$ (popř. vlastní podtřídou).
\end{itemize}

\textit{Cvičení: Rozepište v základním jazyce teorie množin.}

\begin{enumerate}
	\item \textit{$\bigcup A$ nebo-li suma třídy $A$ je $\{x, (\exists a)(a \in A \land x = a)\}$}
	\item \textit{$\bigcap A$ nebo-li průnik třídy $A$ je $\{x, (\forall a)(a \in A \rightarrow x = a)\}$}
	\item \textit{$\mathcal{P}(A)$ nebo-li potenciál třídy $A$ je $\{a, a \subseteq A\}$.}
\end{enumerate}

\textit{$\bigcap \emptyset = V$, protože $\{x, (\forall a)(a \in \emptyset \rightarrow x \in a)\}.$}

\textit{Cvičení: $a \neq \emptyset$, je $\bigcap a$ množina?}

\textit{Cvičení: Je $\mathcal{P}(V) = V^{2}$?}

\begin{lemma}
	Univerzální třída $V$ není množina.
\end{lemma}

\begin{proof}
	\textit{Cvičení.}
\end{proof}

\begin{lemma}
	Je-li $A$ třída $a$ množina, průnik $A \cap a$ je množina.
\end{lemma}

\begin{proof}
	Schéma axiomu vydělení $A = \{x, \varphi(x)\}, a \cap A = \{x, x \in a \land \varphi(x)\}$.
\end{proof}

\begin{definice}
	\textbf{Kartézský součin tříd} $A,B$ značen $A \times B$ je $\{(a,b), a \in A \land b \in B\}$ což je zkrácený zápis pro $\{x, (\exists a)(\exists b)(x = (a,b) \land a \in A \land b \in B)\}$.
\end{definice}

\begin{lemma}
	Jsou-li $a,b$ množiny pak i $a \times b$ je množina.
\end{lemma}

\begin{proof}
	\begin{itemize}
		\item Platí $a \times b \subseteq \mathcal{P}(\mathcal{P}( a \cup b))$.
		\item Vpravo je množina axiomu dvojice , sumy, dvakrát potence.
		\item Pak podle lemma (axiomu vydělení) $A = a \times b, a = \mathcal{P}(\mathcal{P}(a \cup b))$ tedy $a \times b$ je množina.
		\item Pokud $u \in a, v \in b$, pak $\{u\},\{u,v\} \subseteq a \cup b$ tedy $\{u\},\{u,v\} \in \mathcal{P}(a \cup b)$, stejně pak $\{\{u\}, \{u,v\}\} \subseteq \mathcal{P}(a \cup b)$ a $\{\{u\}, \{u,v\}\} \in \mathcal{P}(\mathcal{P}(a \cup b))$.
	\end{itemize}
\end{proof}

\begin{definice}
	$X$ je třída, pak $X^{1} = X$, induktivně pak $X^{n} = X^{n-1} \times X$.
\end{definice}

$X^{n}$ je třída všech uspořádaných $n$-tic prvků $X$.

Pozorování: $V^{n} \subseteq V^{n-1} \subseteq \dots \subseteq V^{1} = V$

\textit{Cvičení: Ukažte, že obecně neplatí $X \times X^{2} = X^{3}$. Například pro $X = \{\emptyset\}$.}
	\chapter{Relace}

\begin{definice}
	\begin{itemize}
		\item Třída $R$ je (binární) \textbf{relace}, pokud $R \subseteq V \times V$.
		\item $x R y$ zkratka za $(x,y) \in R$.
		\item $n$-ární relace je $R \subseteq V^{n}$.
	\end{itemize}
\end{definice}

\begin{prikl}
	\begin{itemize}
		\item \textit{Relace náležení $E$ je $\{(x,y), x \in y \}$.}
		\item \textit{Relace identity $Id$ je $\{(x,y),x = y \}$.}
	\end{itemize}
\end{prikl}

\begin{definice}
	Je-li $X$ relace (libovolná třída), pak:
	
	\begin{itemize}
		\item $Dom (X)$ je $\{u,(\exists v)(u,v) \in X\}$
		\item $Rng (X)$ je $\{v, (\exists u)(u,v) \in X\}$
		\item Je-li $Y$ třída, pak $X \shortparallel Y (X [Y])$ je $\{z, (\exists y)(y \in Y \land (y,z) \in X\}$.
		\item Nebo-li obraz třídy $Y$ třídou $X$.
		\item $X \upharpoonright Y$ je $\{(y,z), y \in Y \land (y,z) \in X\}$.
		\item Zúžení třídy $X$ na třídu $Y$. (restrikce, parcelizace)
	\end{itemize}
\end{definice}

\begin{lemma}
	Je-li $x$ množina, $Y$ třída, pak $Dom(x), Rng(x), x \upharpoonright Y, x \shortparallel Y$ jsou množiny.
\end{lemma}

\begin{proof}
	\begin{itemize}
		\item Vnoříme do větší množiny.
		\item Platí $Dom(x) \subseteq \bigcup( \bigcup(x))$.
		\item Když $u \in Dom(x)$ pak $(\exists v)(u,v) \in x$ a $u \in \{u\} \in (u,v) \in x$. Tedy $\{u\} \in \bigcup (x)$, tedy $u \in \bigcup(\bigcup(x))$.
		\item Podobně i pro $Rng(x) \subseteq \bigcup(\bigcup(x))$.
		\item $v \in Rng(x): (\exists u)(u,v) \in x$
		\item $v \in \{u,v\} \in (u,v) \in x$ tedy $v \in \bigcup(\bigcup(x))$.
		\item Pak už jenom $x \upharpoonright Y \subseteq x; x \shortparallel Y \subseteq Rng(x)$
	\end{itemize}
\end{proof}

\begin{definice}
	\begin{itemize}
		\item $R,S$ jsou relace. Pak $R^{-1}$ je $\{(u,v), (v,u) \in R\}$.
		\item Nebo-li relace \textbf{inverzní} k $R$.
		\item $R \circ S$ je $\{(u,v); (\exists w)((u,w)\in R \land (w,v) \in S)\}$.
		\item Nebo-li složení relací $R$ a $S$.
	\end{itemize}
\end{definice}

\begin{pozn}
	$(f \circ g)(x) = g(f(x))$
\end{pozn}

\textit{Cvičení}

\begin{itemize}
	\item \textit{Ověřte, že pro libovolnou relaci $R$ je $Id \circ R = R = R \circ Id$.}
	\item \textit{$(x,y) \in E \circ E \leftrightarrow x \in \bigcup y$}
\end{itemize}

\begin{definice}
	Relace $F$ je \textbf{zobrazení (funkce)} pokud:
	
	$$
	(\forall u)(\forall v)(\forall w)(((u,v) \in F \land (u,w) \in F) \rightarrow v = w)
	$$
\end{definice}

“Pro každé $v \in Dom(F)$ existuje právě jedna množina $v$ taková, že $(u,v) \in F$.” Píšeme $F(u) = v$.

\begin{definice}
	\begin{itemize}
		\item $F$ je zobrazení třídy $X$ \textbf{do} třídy $Y$; $F: X \to Y$, pokud $Dom(F) = X$ a $Rng(F) \subseteq Y$.
		\item $F$ je zobrazení třídy $X$ \textbf{na} třídu $Y$; pokud navíc platí $Rng(F) = Y$.
		\item $F$ je \textbf{prosté} zobrazení pokud $F^{-1}$ je zobrazení.
		\item Pokud $(\forall v)(\forall u)(\forall w)((F(u) = w \land F(v) = w) \rightarrow u = v)$.
		\item “Každý prvek $Rng(F)$ má právě jeden vzor.”
	\end{itemize}
\end{definice}

Pozorování: Pokud $F$ je prosté zobrazení, pak $F^{-1}$ je také prosté zobrazení.

\begin{definice}
	$A$ je třída, $\varphi$ je formule pak:
	
	\begin{itemize}
		\item $(\exists x \in A) \varphi$ je zkratka za $(\exists x)(x \in A \land \varphi)$.
		\item $(\forall x \in A) \varphi$ je zkratka za $(\forall x)(x \in A \rightarrow \varphi)$.
	\end{itemize}	
\end{definice}

\section{Značení:}

\textbf{Obraz / vzor} třídy $X$ zobrazením $F$.

\begin{itemize}
	\item $F[X]$ místo $F \shortparallel X$ : $F[X] = \{y, (\exists x \in X) y = F(x)\}$
	\item $F^{-1}[X]$ místo $F^{-1} \shortparallel X$ : $F^{-1}[X] = \{y, (\exists x \in X) x = F(y)\}$
\end{itemize}

\begin{definice}
	$A$ je třída, $a$ je množina, pak $^{a}A$ je $\{f; f: a \to A\}$, třída všech zobrazení z $a$ do $A$.
\end{definice}

\begin{pozn}
	\begin{itemize}
		\item Z axiomu nahrazení $Rng(f)$ je množina, $f \subseteq a \times Rng(f)$, tedy $f$ je množina.
		\item Nelze definovat $^{B}A$ pokud $B$ je vlastní třída a $A \neq \emptyset$, protože je-li $Dom(f)$ vlastní třída, pak je i $f$.
		\item $^{\emptyset}A = \{\emptyset\}$
		\item $^{x}\emptyset = \emptyset$
	\end{itemize}
\end{pozn}

\begin{lemma}
	\begin{enumerate}
		\item Pro libovolné množiny $x,y$ je $^{x}y$ množina.
		\item Je-li $x \neq \emptyset, Y$ je vlastní třída, pak $^{x}Y$ je vlastní třída.
	\end{enumerate}
\end{lemma}

\begin{proof}
	\begin{enumerate}
		\item Pokud $f: x \to y$. pak $f \subseteq x \times y$, tedy $f \in \mathcal{P}(x \times y)$. Tedy $^{x}y \subseteq \mathcal{P}(x \times y)$.
		\item Pro $y \in Y$ definujeme konstantní zobrazení $K_{y}: x \to Y$ tak, že $(\forall u \in x)(K_{y}(u) = y)$. $K_{y} = x \times {y}$, protože $x \neq \emptyset$, pro $y \neq y'$ platí $K_{y} \neq K_{y'}$. $K = \{K_{y}, y \in Y\}$ máme $K \subseteq ^{x}Y$.
		\begin{itemize}
			\item Teď sporem: Pokud $^{x}Y$ je množina, pak $K$ je množina. Definujeme $F: K \to Y$ jako $F(K_{y})=y$. Z axiomu nahrazení $Y$ je množina a to je spor.
		\end{itemize}
	\end{enumerate}
\end{proof}

\section{Uspořádání}

\begin{definice}
	Relace $R (\subseteq V \times V)$ je na třídě $A$:
	
	Reflexivní:
	
	$$
	(\forall x \in A)((x,x) \in R)
	$$
	
	Antireflexivní:
	
	$$
	(\forall x \in A)((x,x) \notin R)
	$$
	
	Symetrická:
	
	$$
	(\forall x, y \in A)((x,y) \in R \leftrightarrow (y,x) \in R)
	$$
	
	Slabě antisymetrická:
	
	$$
	(\forall x, y \in A)(((x,y) \in R \land (y,x) \in R) \rightarrow y = x)
	$$
	
	Antisymetrická
	
	$$
	(\forall x \in A)(\forall y \in A)(x R y \rightarrow \neg (y R x))
	$$
	
	Trichotomická:
	
	$$
	(\forall x \in A)( \forall y \in A)(xRy \lor yRx \lor x = y)
	$$
	
	Tranzitivní:
	
	$$
	(\forall x,y,z \in A)((xRy \land yRz) \rightarrow xRz)
	$$
\end{definice}

Pozorování: Tyto vlastnosti jsou \textbf{dědičné}, to znamená, že platí na každé podtřídě $B \subseteq A$.

\begin{definice}
	\begin{itemize}
		\item Relace $R$ je \textbf{uspořádání na třídě $A$}, pokud $R$ je reflexivní, slabě antisymetrická a tranzitivní.
		\item $x,y \in A$ jsou \textbf{porovnatelné (srovnatelné)} relací $R$ pokud $xRy \lor yRx$.
	\end{itemize}
\end{definice}

\subsection{Značení:}

$x \leq_{R} y$ znamená $xRy$, neboli "$x$ je menší nebo rovno $y$ vzhledem k $R$."

\begin{definice}
	\begin{itemize}
		\item Uspořádání $R$ je \textbf{lineární} pokud $R$ je trichotomické.
		\item $R'$ je \textbf{ostré} uspořádání pokud je tvaru $R \setminus Id$ (je antireflexivní, antisymetricá a tranzitivní).
		\item $x <_{R} y$ značí $x R' y$
	\end{itemize}
\end{definice}

\textit{Cvičení: Doplňte tabulku ANO/NE.}

\begin{center}
	\begin{tabular}{c | c | c}
		\centering
		Relace & Uspořádání? & Ostré? \\ \hline
		$E$    &             &        \\
		$Id$   &             &
	\end{tabular}
\end{center}

\begin{definice}
	Nechť $R$ je uspořádání na třídě $A$ a nechť $X \subseteq A$. Řekněme, že $a \in A$ je (vzhledem k $R$ a $A$):
	
	\begin{itemize}
		\item \textbf{Majorita (horní mez)} třídy $X$, pokud $(\forall x \in X)(x \leq_{R} a)$.
		\item \textit{Minoranta (dolní mez)} třídy $X$, pokud $(\forall x \in X)(a \leq_{R} x)$.
		\item \textbf{Maximální prvek} třídy $X$, pokud $a \in X \land (\forall x \in X)(\neg (a <_{R} x))$.
		\item \textit{Minimální prvek} třídy $X$, pokud $a \in X \land (\forall x \in X)(\neg (x <_{R} a))$.
		\item \textbf{Největší prvek} třídy $X$, pokud $a \in X$ a $a$ je majoranta $X$.
		\item \textit{Největší prvek} třídy $X$, pokud $a \in X$ a $a$ je minoranta $X$.
		\item \textbf{Supremum} třídy $X$, pokud $a$ je nejmenší prvek třídy všech majorant $X$.
		\item \textit{Infimum} třídy $X$, pokud $a$ je největší prvek třídy všech minorant $X$.
	\end{itemize}
\end{definice}

Pozorování: Největší implikuje maximální, pokud $R$ je lineární, tak platí i opačná implikace. Také největší a supremum je vždy nejvýše 1. Lze značit jako $a = \max_{R}(X)$ a $a = \sup_{R}(X)$.

\begin{definice}
	\begin{itemize}
		\item $X$ je \textbf{shora omezená}, pokud existuje majoranta $X$ v $A$.
		\item $X$ je \textit{zdola omezená}, pokud existuje minoranta $X$ v $A$.
		\item $X$ je \textbf{dolní množina}, pokud $(\forall x \in X)(\forall y \in A)(y \leq_{R} x \rightarrow y \in X)$.
		\item Analogicky i \textit{horní množina}.
		\item $x \in A$, pak $| \leftarrow, x]$ je $\{y, y \in A \land y \leq_{R} x\}$. Nebo-li horní ideál omezená $x$.
	\end{itemize}
\end{definice}

Pozorování: $R$ uspořádání na $A$, pak pro libovolné $x,y \in A$ platí $x \leq_{R} y \leftrightarrow |\leftarrow,x] \subseteq |\leftarrow,y]$.

\begin{pozn}
	\begin{itemize}
		\item Konstrukce $\mathbb{R}$ z $\mathbb{Q}$: \textbf{Dedekindovy řezy}.
		\item $X \subseteq \mathbb{Q}, X$ je dolní množina (vzhledem k $\subseteq$) a navíc existuje-li $\sup X$, pak $\sup X \subseteq X$.
	\end{itemize}
\end{pozn}

\begin{definice}
	Uspořádání $R$ na třídě $A$ je \textbf{dobré}, pokud každá neprázdná podmnožina $A: (u \subseteq A)$ má nejmenší prvek vzhledem k $R$.
\end{definice}

\textit{Cvičení: Napsat definice pomocí logických formulí.}

Pozorování: “Dobré” je dědičná vlastnost. Dobré implikuje lineární.

\textit{Cvičení: Najděte nějaké množiny, na nichž je $E$ dobré ostré uspořádání.}

\begin{definice}
	\textbf{Ekvivalence} je pokud je reflexivní, symetrická a tranzitivní.
\end{definice}
	\chapter{Srovnávání mohutností}

\begin{definice}
	\begin{itemize}
		\item Množiny $x,y$ mají \textbf{stejnou mohutnost} (psáno $x \approx y$) pokud existuje prosté zobrazení $x$ na $y$ (nebo-li bijekce). Někdy označováno jako $x$ je \textit{ekvivalentní} $y$.
		\item Množina $x$ má \textbf{mohutnost menší nebo rovnou} mohutnosti $y$ (psáno $x \preceq y$) pokud existuje prosté zobrazení $x$ do $y$. Někdy označováno jako $x$ je \textit{subvalentní} $y$.
		\item $x$ má \textbf{menší mohutnost} než $y$ (psáno $x \prec y$) pokud platí $x \preceq y \land \neg (x \approx y))$.
	\end{itemize}
\end{definice}

Pozorování: $x \subseteq y \rightarrow x \preceq y$ (identita),  $x \subset y \rightarrow x \preceq y$ (ne $x \prec y$, například $\mathbb{N} \approx \mathbb{N}\setminus\{1\})$.

\begin{pozn}
	To jestli $\preceq$ je trichotomická v **ZF** nelze rozhodnout. Přidáím axiomu výběru už ale ano.
\end{pozn}

\begin{lemma}
	Jsou-li $x,y,z$ množiny, potom:
	
	\begin{enumerate}
		\item $x \approx x$
		\item $x \approx y \rightarrow y \approx x$
		\item $((x \approx y) \land (y \approx z)) \rightarrow x \approx z$, tedy $\approx$ je ekvivalence.
		\item $x \preceq x$
		\item $x \preceq y \land y \preceq z \rightarrow x \preceq z$
	\end{enumerate}
\end{lemma}

\begin{proof}
	Prakticky jen triviální, stačí najít dané zobrazení.
	
	\begin{itemize}
		\item $Id$
		\item $F \rightarrow F^{-1}$
		\item $F \land G \rightarrow F \circ G$
		\item $Id$
		\item $F \land G \rightarrow F \circ G$
	\end{itemize}  
\end{proof}

Pozorování: $x \approx y \rightarrow (x \preceq y \land y \preceq x)$

\begin{thm}[Cantor-Bernstein]
	$$
	(x \preceq y \land y \preceq x) \rightarrow x \approx y
	$$
\end{thm}

\begin{proof}
	Důkaz se provede pomocí grafů. Také bude potřeba dodatečné lemma, které bude později. Jako graf si představíme bipartitní, kde jedna partita je $x$ a druhá $y$. Následně přidáme orientované hrany jakožto funkce $f$ a $g$, kde $f: x \to y, g: y \to x$ jsou prosté zobrazení. Teď se podíváme na komponenty grafu.
	
	\begin{enumerate}
		\item Buď může být kružnice sudé délky.
		\item Nebo cesta s počátkem.
		\item Anebo cesty obousměrné.
	\end{enumerate}
	
	Nyní uvažme “indukovaná” zobrazení: $(\hat{f}): \mathcal{P}(x) \to \mathcal{P}(y)$. Tahle funkce je monotónní vzhledem k inkluzi. Definujeme $H: \mathcal{P}(x) \to \mathcal{P}(x)$ takto: Pro $u \subseteq x$ nechť $H(u) = x - g[y - f[u]]$. $H$ je monotónní vzhledem k inkluzi. $u_{1} \subseteq u_{2} \Rightarrow f[u_{1}] \subseteq f[u_{2}] \Rightarrow y - f[u_{1}] \supseteq y - f[u_{2}] \Rightarrow$, $\Rightarrow g[y - f[u_{1}] \supseteq g[y - f[u_{2}] \Rightarrow H(u_{1}) \subseteq H(u_{2})$. Podle lemma o pevném bodě $(\exists c)(H(c) = c)$, tedy $x - g[y - f[c]] = c \Rightarrow x - c = g[y - f[c]]$. Tedy $g^{-1}$ je prosté zobrazení $x\setminus c$ na $y \setminus f[c]$. Stačí definovat $h: x \to y$ jako:
	
	$$
	h(u) =
	\left\{
	\begin{array}{ll}
		f(u) & \text{pokud } u = c \\
		g^{-1}(u) & \text{jinak}
	\end{array}
	\right.
	$$
	
	$h$ je prosté zobrazení $x$ na $y$.
\end{proof}

\begin{definice}
	Zobrazení $H: \mathcal{P}(x) \to \mathcal{P}(x)$ je \textbf{monotónní} (vzhledem k inkluzi) pokud pro každé dvě množiny $u,v \subseteq x$ platí $u \subseteq v \rightarrow H(u) \subseteq H(v)$.
\end{definice}

\begin{lemma}
	Je-li $H: \mathcal{P}(x) \to \mathcal{P}(x)$ zobrazení monotónní vzhledem k inkluzi, pak existuje podmnožina $c \subseteq x$ taková, že $H(c) = c$. Též označován jako \textbf{pevný bod}.
\end{lemma}

\begin{proof}
	$A = \{u, u \subseteq x \land u \subseteq H(u)\}$, $c = \bigcup A$ neboli supremum. $u \in A$ pak dostanu dvě možnosti:
	
	\begin{enumerate}
		\item $u \subseteq c$
		\item $u \subseteq H(u) \subseteq H(c)$ (díky tomu, že $H$ je monotónní)
	\end{enumerate}
	
	Z toho pak plyne, že $H(c)$ je majoranta a tedy $c \subseteq H(c)$. Pak z monotonie platí $H(c) \subseteq H(H(c))$, tedy $H(c) \in A$, takže $H(c) \subseteq c$, nebo-li $c$ je majoranta. Z obou inkluzí pak plyne, že $c = H(c)$.
\end{proof}

\textit{Cvičení: Ilustrace monotńní funkce $h: [0,1] \to [0,1]$.}

\textit{Cvičení: $A \subseteq \mathcal{P}(x)$ a uspořádání $\subseteq$, pak $\sup_{\subseteq} A = \bigcup A$ a $\inf_{\subseteq} A = \bigcap A$.}

\begin{prikl}
	\begin{itemize}
		\item $\omega = \mathbb{N}_{0}$ pak $\omega \approx \omega \times \omega$
		\item $f: \omega \to \omega \times \omega$ jako $f(n) = (0,n)$
		\item $g: \omega\times\omega \to \omega$ jako $g((m,n)) = 2^{m}3^{n}$
		\item Podle Věty platí $\omega \approx \omega \times \omega$.
		item $h: \omega\to\omega\times\omega$ jako $h((m,n)) = 2^{m}(2n+1)-1$
	\end{itemize}
\end{prikl}

\textit{Cvičení: Ověřte, že $g$ je prosté a $h$ je bijekce.}

\textit{Cvičení: $\mathbb{N} \approx \mathbb{Q}$}

\textit{Cvičení: $[0,1] \approx [0,1] \times [0,1]$}

\begin{lemma}
	Nechť $x,y,z,x_{1},y_{1}$ jsou množiny, pak:
	
	\begin{enumerate}
		\item $x \times y \approx y \times x$
		\item $x \times (y \times z) \approx (x \times y) \times z$
		\item $(x \approx x_{1} \land y \approx y_{1}) \rightarrow (x \times y \approx x_{1} \times y_{1})$
		\item $x \approx y \rightarrow \mathcal{P}(x) \approx \mathcal{P}(y)$
		\item $\mathcal{P}(X) \approx ^{x}2$, kde $2 = \{\emptyset,\{\emptyset\}\}$
	\end{enumerate}
\end{lemma}
 
\begin{proof}
	Vždy jde o to najít vhodné funkce.
	
	\begin{enumerate}
		\item $(u,v) \to (v,u)$
		\item $(u,(b,c)) \to ((u,b),c)$
		\item $f: x \to x_{1}, g: y \to y_{1}: (a,b) \to (f(a),g(b))$
		\item $f:x \to y, u \to f[u]$ (izomorfismus vzhledem k inkluzi)
		\item Pro $u \subseteq x$ definujeme charakteristickou funkci $\chi_{a}:x \to 2$, kde;
	\end{enumerate}
	
	$$
	\chi_{a}(v) =
	\left\{
	\begin{array}{ll}
		1 & v \in a \\
		0 & v \notin a
	\end{array}
	\right.
	$$
	
	Zobrazení $\{(a, \chi_{a}); a \subseteq x\}$ je prosté a zobrazuje $\mathcal{P}(x)$ na $^{x}2$.
\end{proof}

\section{Konečné množiny}

\begin{definice}[Tarski]
	Množina $x$ je \textbf{konečná}, označíme $Fin(x)$, pokud každá neprázdná podmnožina $\mathcal{P}(x)$ má \textbf{maximální} prvek vzhledem k inkluzi.
\end{definice}

\textit{Cvičení: Napište definici pomocí formule.}

Pozorování: $x$ je konečná právě tehdy, když každá neprázdná podmnožina $\mathcal{P}(x)$ má minimální prvek vzhledem k inkluzi.

\begin{proof}
	Uvažme $d: \mathcal{P}(x) \to \mathcal{P}(x)$ jako $d(u) = x \setminus u$. $u \subseteq v \Leftrightarrow d(u) \supseteq d(v)$
\end{proof}

\begin{definice}
	Množina $a$ je \textbf{Dedekindovsky konečná} pokud má větší mohutnost než každá vlastní podmnožina $b \subset a$. (Nebo-li neexistuje prosté zobrazení $a$ na $b$.)
\end{definice}

\begin{lemma}
	Je-li množina $a$ konečná tak je i Dedekindovsky konečná.
\end{lemma}

\begin{proof}
	Nutno dokázat, že pokud $b \subset a$ pak $b \preceq a$. Sporem: $b \approx a$. Nechť $y = \{b, b \subset a \land b \approx a\}, y \neq \emptyset, y \in \mathcal{P}(a)$. Nechť $c \in y$ je minimální prvek $y$ vzhledem k $\subseteq$. Nechť $f: a \to a$ je prosté zobrazení $a$ na $c$. $d = f[c]$. $f \upharpoonright c$ je prosté zobrazení $c$ na $d$. Tedy $c \approx d$, tedy $d \in y$. $d \subseteq c: (\exists x)( x \in a \setminus c)$ pak $f(x) \in c \setminus d$. Spor s minimalitou volby $c$.
\end{proof}

\begin{pozn}
	Opačná implikace v \textbf{ZF} není dokazatelná.
\end{pozn}


\begin{itemize}
	\item Existuje lineární uspořádání $\leq$, které je dobré, pak i $\geq$ je dobré.
	\item Existuje lineární uspořádání a každá 2 lineární uspořádání jsou izomorfní.
	\item $x$ je konečná $\Leftrightarrow \mathcal{P}(\mathcal{P}(x))$ je dedekindovsky konečná.
\end{itemize}

\begin{thm}
	\begin{enumerate}
		\item Je-li $a$ konečná uspořádaná množina (relací $\leq$) pak každá její neprázdná podmnožina $b \subseteq a$ má maximální prvek.
		\item Každé lineární uspořádání na konečné množině je dobré.
	\end{enumerate}
\end{thm}

\begin{proof}
	\begin{enumerate}
		\item Pro každé $x \in a$ uvažme $| \leftarrow , x] = \{y, y \in a \land y \leq x\}$.
		\begin{itemize}
			\item $u = \{|\leftarrow , x], x \in b\}, u \subseteq \mathcal{P}(a), u \neq \emptyset$
			\item Z konečnosti $a$ existuje $m \in b$ takové, že $| \leftarrow ,m]$ je maximální prvek vzhledem k $\subseteq$.
			\item $x \leq y \Leftrightarrow | \leftarrow , x] < | \leftarrow , y]$
			\item Tedy $m$ je maximální prvek $b$ vzhledem k $\subseteq$.
			\item Minimální prvek se najde podobně, akorát to bude horní množina a minimální prvek.
		\end{itemize}
		\item Minimální prvek v lineárním uspořádání je už nejmenší.
	\end{enumerate}
\end{proof}

\begin{definice}
	$F$ je zobrazení $A_{1}$ do $A_{2}$, $R_{1},R_{2}$ jsou relace. $F$ je \textbf{izomorfismus} tříd $A_{1},A_{2}$ vzhledem k $R_{1},R_{2}$ pokud $F$ je prosté zobrazení $A_{1}$ na $A_{2}$ a $(\forall x \in A_{1})(\forall y \in A_{2})(x,y) \in R_{1} \leftrightarrow (F(x),F(y)) \in R_{2}$.
\end{definice}

\begin{definice}
	$A$ je mmožina uspořádaná relací $R$. $B$ je mmožina uspořádaná relací $S$. Zobrazení $F$ je \textbf{počátkové vnoření} $A$ do $B$, pokud $A_{1} = Dom(F)$ je dolní podmnožina $A$ a $B_{1} = Rng(F)$ je dolní podmnožina $B$. A $F$ je izomorfismus $A_{1}$ a $B_{1}$ vzhledem k $R,S$.
\end{definice}

\begin{lemma}
	Nechť $F,G$ jsou počátkové vnoření dobře uspořádané množiny $A$ do dobře uspořádané množiny $B$. Potom $F \subseteq G$ nebo $G \subseteq F$.
\end{lemma}

\begin{proof}
	Nechť $R$ je dobré uspořádání množiny $A$. Nechť $S$ je dobré uspořádání množiny $B$. $Dom(F), Dom(G)$ jsou dolní podmnožiny $A$. $R$ je lineární, tedy $Dom(F) \leq Dom(G) \lor Dom(G) \leq Dom(F)$. (BÚNO: $Dom(F) \leq Dom(G)$, jinak přejmenuji množiny). Dokážeme $(\forall x \in Dom(F)) F(x) = G(x)$. Sporem Nechť $x$ je nejmenší (vzhledem k $R$) prvek množiny $\{z, z \in A \land G(z) \neq F(z)\}$. Tedy $\forall y <_{R} x : F(y) = G(y)$. Z linearity $S$ je $F(x) <_{S} G(x) \lor G(x) <_{S} F(x)$ (BÚNO: $F(x) <_{S} G(x)$). Nechť $b = F(x)$. Je-li $z \in Dom(G)$ pak buď: $z <_{R} x$ $G(z) = F(z)$, $z \geq_{R} x$ $F(x) = b$. Pak $G(z) \geq_{S} G(x) >_{S} F(x) = b$. V obou případech $b \notin Rng(G)$ a tedy $Rng(G)$ není dolní množina a to je spor.
\end{proof}

\textit{Cvičení: Lineární uspořádání jsou každé dvě dolní množiny porovnatelné inkluzí.}

\textit{Cvičení: Co když místo dobrého uspořádání bude jen lineární uspořádání.}

\begin{thm}[O porovnávání dobrých uspořádání.]
	$A$ je množina dobře uspořádaná relací $R$.	$B$ je množina dobře uspořádaná relací $S$. Pak existuje právě jedno zobrazení $F$, které je izomorfismus $A$ a dolní množiny $B$, nebo $B$ a dolní množiny $A$.
\end{thm}

\begin{proof}
	$P$ je množina všech počátečních vnoření $A$ do $B$. Nechť $F = \bigcup P$. $F$ je zobrazení: Když $(x,y_{1})(x,y_{2}) \in F$ existuje počáteční vnoření $F_{1}, F_{2}$, že $(x,y_{1}) \in F_{1}, (x,y_{2}) \in F_{2}$. Podle lemma $F_{1} \subseteq F_{2}$ nebo naopak. Předpokládejme, že nastala tato situace. Tedy $(x,y_{1}) \in F_{2}; F_{2}$ je zobrazení, tedy $y_{1} = y_{2}$. $F$ je počáteční vnoření: Když $x_{1} <_{R} x_{2} \in Dom(F)$ tak existuje počáteční vnoření $F'$ že $x_{2} \in Dom(F')$. Tedy $x_{1} \in Dom(F') \subseteq Dom(F)$. Podobně pro $Rng(F) = \bigcup Rng(F')$ je dolní. $F(x_{1}) = F'(x_{1}) <_{S} F'(x_{2}) = F(x_{2})$, $Dom (F) = A \lor Rng(F) = B$. Sporem: $A \setminus Dom(F), B \setminus Rng(F)$ jsou neprázdné, mající nejmenší prvky $a,b$. Definujeme $F'= F \cup \{(a,b)\}$ je počáteční vnoření $F' \in P, F' \subseteq F$ a to je spor.
\end{proof}

\textit{Cvičení: Jednoznačnost $F$.}

\textit{Cvičení: Sjednocení dolních množin je dolní množina.}

\begin{thm}
	$a$ je konečná množina, pak každé lineární uspořádání na $a$ jsou izomorfní.
\end{thm}

\begin{proof}
	$R,S$ jsou dvě lineární uspořádání a také dobrá uspořádání. $(a,R)$ je izomorfní dolní množině $(a,S)$ nebo dolní množina $(a,R)$ je izomorfní $(a,S)$. Dolní množina $b, b \approx a$, z Dedekindovy konečnosti platí, že $a = b$.
\end{proof}

\begin{lemma}[Zachovávání konečnosti.]
	\begin{enumerate}
		\item $(Fin(x) \land y \subseteq x) \rightarrow Fin(y)$
		\item $(Fin(x) \land y \approx x) \rightarrow Fin(y)$
		\item $(Fin(x) \land y \preceq x) \rightarrow Fin(y)$
	\end{enumerate}
\end{lemma}

\begin{proof}
	\begin{enumerate}
		\item $w \subseteq \mathcal{P}(y) \subseteq \mathcal{P}(x)$
		\item $\mathcal{P}(y)$ je izomorfní $\mathcal{P}(x)$
		\item Plyne z 1 a 2.
	\end{enumerate}
\end{proof}

\begin{lemma}[sjednocení konečných množin]
	\begin{enumerate}
		\item $Fin(x) \land Fin(y) \rightarrow Fin(x \cup y)$
		\item $Fin(x) \rightarrow (\forall y) Fin(x \cup \{y\})$
	\end{enumerate}
\end{lemma}

\begin{proof}
	$w \subseteq \mathcal{P}(x \cup y)$ neprázdná. $w_{1} = \{ u, (\exists t \in w)( u = t \cap x)\} \subseteq \mathcal{P}(x)$. Má maximální prvek $v_{1}$. $w_{2} = \{u, (\exists t \in w)( t \cap x = v_{1} \land t \cap y = u)\} \subseteq \mathcal{P}(y)$. Má maximální prvek $v_{2}$. $v_{1} \cup v_{2}$ je maximální prvek $w$.
\end{proof} 

\begin{definice}
	Třída všech konečných množin $Fin = \{x, Fin(x)\}$.
\end{definice}

\begin{thm}[Princip indukce pro konečné množiny]
	Je-li $X$ třída, pro kterou platí:
	
	\begin{enumerate}
		\item $\emptyset \in X$,
		\item $x \in X \rightarrow (\forall y)(x \cup \{y\} \in X)$, pak $Fin \subseteq X$.
	\end{enumerate}
\end{thm}

\begin{proof}
	Sporem: Pokud $x \in Fin \setminus X$. nechť $w = \{v, v \subseteq x \land v \in X\}$. Podle 1: $\emptyset \in w$. $w \subseteq \mathcal{P}(x)$, neprázdná. $w$ má maximální prvek $v_{0}$. $v_{0} \subseteq x$. $v_{0} \in X$, tedy $v_{0} \neq x$ a $v_{0} \subset X$. Tedy existuje $y \in x \setminus v_{0}$. Nechť $v_{1} = v_{0} \cup \{y\}$. Podle 2: $v_{1} \in X$. Tedy $v_{1} \in w$, spor s maximalitou $v_{0}$.
\end{proof}

\begin{lemma}
	$Fin (x) \rightarrow Fin(\mathcal{P}(x))$
\end{lemma}

\begin{proof}
	Indukcí: Nechť $X = \{x, Fin(\mathcal{P}(x))\}$. $\emptyset \in X$, protože $\mathcal{P}(\emptyset) = \{\emptyset\}$ je konečná. Nechť $x \in X, y$ je množina. Chceme aby $x \cup \{y\} \in X$. BÚNO: $y \notin x$ (jinak triviální). Rozdělíme $\mathcal{P}(x \cup \{y\})$ na dvě části: $\mathcal{P}(x \cup \{y\}) = \mathcal{P}(x) \cup (\mathcal{P}(x \cup \{y\}) \setminus \mathcal{P}(x))$. Platí $\mathcal{P}(x) \approx z$, kde $z$ se rovná předchozímu druhému prvku v sjednocení. Pro $u \in \mathcal{P}(x)$ definujeme $f(u) = u \cup \{y\}$. $f$ je prosté zobrazení $\mathcal{P}(x)$ na $z$. Podle předpokladu $Fin(\mathcal{P}(x))$. Podle lemma $Fin(z)$. Podle lemma o sjednocení $Fin(\mathcal{P}(x) \cup z)$. Podle principu indukce $Fin \subseteq X$.
\end{proof}

\begin{dusl}
	$Fin(x) \cap Fin(y) \rightarrow Fin(x \times y)$
\end{dusl}

\begin{proof}
	Nechť $z = x \cup y$, víme $Fin(z)$. $x \times y \subseteq z \times z \subseteq \mathcal{P}(\mathcal{P}(z))$.
\end{proof}

\begin{lemma}[sjednocení konečně mnoha konečných množin je konečná množina]
	Je-li $Fin(a)$ a $(\forall b \in a) Fin(b)$, pak $Fin(\bigcup a)$.
\end{lemma}

\begin{proof}
	Indukcí: $X = \{x, x \subseteq Fin \rightarrow Fin(\bigcup x)\}$.
	
	\begin{enumerate}
		\item $\emptyset \in X$, protože $\bigcup \emptyset = \emptyset$.
		\item Nechť $x \in X, y$ množina. Chceme aby $x \cup \{y\} \in X$.
	\end{enumerate}
	
	Předpokládejme, že $x \cup \{y\} \subseteq Fin$. Speciálně $x \subseteq Fin$. $\bigcup (x \cup \{y\}) = \bigcup x \cup y$. Obě dvě jsou konečné a sjednocení tím pádem je také konečné. Tedy $x \cup \{y\} \in X$. Podle principu indukce $Fin \subseteq X$.
\end{proof}

\begin{dusl}[Dirichletův princip pro konečné množiny.]
	Je-li nekonečná množina sjednocení konečně mnoha množin, pak jedna z nich musí být nekonečná.
\end{dusl}

\begin{lemma}[Každá konečná množina je srovnatelná se všemi množinami.]
	$Fin(x) \rightarrow (\forall y)( y \preceq x \lor x \preceq y)$
\end{lemma}

\begin{proof}
	Indukcí: $x = \{x, (\forall y)(y \preceq x \lor x \preceq y)\}$.
	
	\begin{enumerate}
		\item $\emptyset \in X$, protože $(\forall y) \emptyset \subseteq y$ tedy $\emptyset \preceq y$.
		\item Nechť $x \in X, u$ je množina. BÚNO: $u \notin X$. Chceme $x \cup \{u\} \in X$, nechť $X$ je množina.
	\end{enumerate}
	
	Když $y \preceq x$, pak $x \preceq x \cup \{u\}$ z tranzitivity $y \preceq x \cup \{u\}$. Nechť $x \prec y$. $g$ je prosté zobrazení $x$ do $y$. Nechť $v \in X \setminus Rng(g)$. Definujeme $h = g \cup \{(u,v)\}, h$ je prosté zobrazení $x \cup \{u\}$ do $y$. Tedy $x \cup \{u\} \preceq y$. Z principu indukce $Fin \subseteq X$.
\end{proof}

\textit{Cvičení: $Fin(x)$ a $f: x \to y$, pak $Rng (f) \preceq x$ (pomocí indukce).}

\textit{Cvičení: $(\forall x) Fin(x)$ lze dobře uspořádat (indukcí).}
	\chapter{Přirozená čísla}

\begin{definice}[von Neumann]
	$0 = \emptyset; 1 = \{0\} = \{\emptyset\}; 2 = \{0,1\} = \{\emptyset, \{\emptyset \}\}; 3 = \{0,1,2\} = \dots$, Myšlenka: “Přirozené číslo je množina všech menších přirozených čísel.”
\end{definice}

\begin{definice}
	$w$ je \textbf{induktivní množina}, pokud $\emptyset \in w \land (\forall v \in w)(v \cup \{v\} \in w)$.
\end{definice}

\section{9.Axiom nekonečna (“Existuje induktivní množina.”)}

$$
(\exists z)(0 \in z \land (\forall x)(x \in z \rightarrow x \cup \{x\} \in z))
$$

\begin{definice}
	\textbf{Množina všech přirozených čísel} $\omega$ je $\bigcap\{w, w \text{ je induktivní množina}\}$.
\end{definice}

\begin{lemma}
	$\omega$ je nejmenší induktivní množina.
\end{lemma}

\begin{proof}
	$0 \in \omega$, $x \in \omega, x$ patří do každé induktivní množiny. $x \cup \{x\}$ patří do každé induktivní množiny. $x \cup \{x\} \in \omega$.
\end{proof}

Prvky $\omega$ jsou \textbf{přirozená čísla} v teorii množin.

\begin{definice}
	Funkce následník $S: \omega\to\omega$. Pro $v \in \omega: S(v) = v \cup \{v\}$. “Následník čísla $v$.”
\end{definice}

\begin{thm}[Princip (slabé) indukce pro přirozená čísla.]
	Je-li $X \subseteq \omega$ taková, že platí:
	
	\begin{enumerate}
		\item $0 \in X$,
		\item $x \in X \rightarrow S(x) \in X$. Pak $X = \omega$.
	\end{enumerate}
\end{thm}

\begin{proof}
	1 a 2 dohromady říká, že $X$ je induktivní, tedy $\omega \subseteq X$.
\end{proof}

\begin{prikl}
	Důkaz indukcí: Chceme dokázat: $(\forall n \in \omega)(\varphi(n))$. Dokazujeme: 1. $\varphi(0)$ a 2. $(\forall n \in \omega)(\varphi(n) \rightarrow \varphi(S(n)))$.
\end{prikl}

\begin{pozn}
	Princip silné indukce: 2: $((\forall m \in \omega) m \in X) \rightarrow n \in X$.
\end{pozn}

\begin{lemma}[$\in$ je ostré uspořádání]
	Pro libovolné $m,n \in \omega$ platí:
	
	\begin{enumerate}
		\item $n \in \omega \rightarrow n \subseteq \omega$
		\begin{itemize}
			\item “Prvky přirozených čísel jsou přirozená čísla.”
		\end{itemize}
		\item $m \in n \rightarrow m \subseteq n$
		\begin{itemize}
			\item “Náležení je tranzitivní na $\omega$.”
		\end{itemize}
		\item $n \nsubseteq n$
		\begin{itemize}
			\item "$\in$ je antireflexivní na $\omega$."
		\end{itemize}
	\end{enumerate}
	
	Z toho všeho plyne, že se jedná o ostré uspořádání.
\end{lemma}

\begin{proof}
	Indukcí:
	
	\begin{enumerate}
		\item $0 \subseteq \omega$, a indukční krok $n \in \omega$, předpokládáme, že $n \subseteq \omega$. Pak $\{n\} \subseteq \omega$ tedy $n \cup \{n\} \subseteq \omega$.
		\item Indukcí podle $n$:
		\begin{itemize}
			\item 1. Krok: $m \notin 0$ tím pádem implikace splněna.
			\item 2.  Krok $X = \{n, n \in \omega \land (\forall m)(m \in n \rightarrow m \subseteq n)\}$.
			\item Víme $0 \in X$.
			\item Nechť $n \in X$, víme $S(n) \in \omega$.
			\item Nechť $m \in S(n) = n \cup \{n\}$. Pak buď $m \in n$ a z IP pak $m \subseteq n$ anebo $m = n$ tím pádem také $m \subseteq n \subseteq S(n)$.
		\end{itemize}
		\item $0 \nsubseteq 0$ platí, nechť $n \in \omega$ a $n \nsubseteq n$.
		\begin{itemize}
			\item Sporem $S(n) \subseteq S(n) = n \cup \{n\}$. Z toho pak plyne, že buď $S(n) \subseteq \{n\}$ anebo $S(n) \subseteq n$. V obou případech je $S(n) \subseteq n$, ale to pak znamená, že $n \in S(n) \subseteq n$ což je spor s předpokladem.
		\end{itemize}
	\end{enumerate}
\end{proof}

\begin{lemma}
	Každé přirozené číslo je konečná množina.
\end{lemma}

\begin{proof}
	Indukcí: $Fin(\emptyset)$ víme. Podle lemma $Fin(x) \rightarrow (\forall y)Fin(x \cup \{y\})$, speciálně pro $Fin(x \cup \{x\})$ a to je následník.
\end{proof}

\begin{thm}
	Množina $x$ je konečná právě tehdy, když $(\exists n \in \omega) x \approx n$.
\end{thm}

\begin{proof}
	$\Leftarrow Fin(n)$ tedy $Fin(x)$. $\Rightarrow$ indukcí: $X = \{x; (\exists n \in \omega) x \approx n\}$. Víme, že $0 \in X$ protože $0 \approx 0$. Nechť $x \in X, y$ množina. Víme, že $(\exists n \in \omega) n \approx x$. $y \in x$ pak $x \cup \{y\} = x \approx n$, $y \notin x$ pak $x \cup \{y\} \approx S(n) = n \cup \{n\}$. K bijekci $x$ a $n$ přidáme $(y,n)$. Tedy $Fin \subseteq X$.
\end{proof}

\begin{lemma}
	Množina $\omega$ i každá induktivní množina je nekonečná.
\end{lemma}

\begin{proof}
	Podle lemma: 1 $n \in \omega \rightarrow n \subseteq \omega$, tedy $n \in \mathcal{P}(n)$ tedy $\omega \subseteq \mathcal{P}(n), \omega$ je neprázdná ale nemá maximální prvek vzhledem k inkluzi. Když $n \subseteq \omega$ pak podle lemma 3. $n \nsubseteq n$ a tedy $n \subset n \cup \{n\} = S(n)$. $\omega \subseteq W$ tedy i induktivní množiny.
\end{proof}

\textit{Cvičení: $\omega$ je Dedekindovsky nekonečná.}

\begin{lemma}[Linearita $\in$ na $\omega$.]
	$m,n \in \omega$, platí:
	
	\begin{enumerate}
		\item $m \in n \leftrightarrow m \subset n$
		\item $m \in n \lor m = n \lor n \in m$ (\textit{trichotomie})
	\end{enumerate}
\end{lemma}

\begin{proof}
	\begin{enumerate}
		\item $\rightarrow$ plyne z lemma 2 $m \in n \rightarrow m \subset n \land n \nsubseteq n$
		\begin{itemize}
			\item $\leftarrow$ indukcí podle $n$; $n = 0$ nelze splnit.
			\item Indukční krok. Nechť platí pro nějaké $n$ a $\forall m$.
			\item Nechť $m \subset S(n) = n \cup \{n\}$ a $m \subseteq n$, kdyby ne pak $n \in m$ tedy $n \subseteq m$ tedy $S(n) = n \cup \{n\} \subseteq m$ a to je spor.
			\item $m \subset n$ z IP $m \in n \subseteq S(n)$ tedy $m \in S(n)$
			\item $m = n$ pak $n \in S(n)$
		\end{itemize} 
		\item Pro $n \in \omega$ nechť $A(n) = \{m \in \omega, m \in n \lor m =n \lor n \in m\}$.
		\begin{itemize}
			\item Dokážeme, že $A(n)$ je induktivní, indukcí podle $m$.
			\item $n = 0: 0 \in A(0)$, protože $0 = 0$
			\item Je-li $m \in A(0)$, pak: $m = 0: 0 \in \{m\}$ anebo $0 \in m$ a z obou plyne $0 \in m \cup \{m\} = S(n)$.
			\item Tedy $S(n) \in A(0)$.
			\item Tedy $A(0) = \omega$.
			\item Tedy také $(\forall n \in \omega) 0 \in A(n)$.
			\item $n \in \omega, m \in \omega$, předpokládejme, že $m \in A(n)$. Ukážeme, že $S(m) \in A(n)$.
			\item $m \in n \rightarrow m \subset n; \{m\} \subseteq n$ tedy $S(m) \subseteq n$ z toho plyne, že $S(m) = n \lor S(m) \in n$.
			\item $m = n \lor n \in m$ potom $n \in m \cup \{m\} = S(m)$
			\item Ve všech případech ke $S(m) \in A(n)$.
		\end{itemize}
	\end{enumerate}
\end{proof}

\begin{thm}
	Množina $\omega$ je dobře (ostře) uspořádaná relací $\in$.
\end{thm}

\begin{proof}
	Nechť $a \subseteq \omega, a \neq \emptyset$. Zvolme $n \in a$. Není-li $n$ nejmenší (minimální), tak definuji $b = n \cap a$. $n$ je konečná, tak i $b$ je konečná a neprázdná. $b \subseteq \omega$ tedy $b$ má minimální prvek $m$ vzhledem k náležení. $m$ je minimální i v množině $a$: kdyby $(\exists x \in a) x \in m$, tak víme, že $m \in n$, tedy $m \subseteq n$, tedy $x \in n$, tedy $x \in b$. To je spor s minimalitou $m$ v $b$. $\in$ je lineární na $\omega$, tedy $m$ je nejmenší prvek v $a$. Tedy $\in$ je dobré uspořádání.
\end{proof}

\begin{pozn}
	Nekonečná množina $A$ s lineárním (ostrým) uspořádáním $<$ pro každé $a \in A: |\leftarrow, a]$ je konečná. Pak $<$ je dobré a $(A,<)$ je izomorfní $(\omega, \in)$.
\end{pozn}

\begin{thm}[Charakterizace uspořádání $\in$ na $\omega$]
	Nechť $A$ je nekonečná množina, lineárně uspořádaná (ostře) relací $<$ tak, že pro každé $a \in A$ je dolní množina $|\leftarrow , a]$ konečná. Pak $<$ je dobré a množiny $A, \omega$ jsou izomorfní vzhledem k $<, \in$.
\end{thm}

\begin{proof}
	$<$ je dobré: $\emptyset \neq c \in A$. Nechť $a \in c$, předpokládejme, že $a$ není minimální v $c$, pak definujeme $b = c \cap |\leftarrow, a]$. $b$ je konečná. Tedy má minimální prvek $m, m$ je minimální i v $c$. Protože $m \leq a$, pak $x \leq a$ tedy $x \in |\leftarrow, a]$ tedy $x \in b$ a to je spor. Izomorfismus: podle věty o porovnávání dobrých uspořádání jsou 2 možnosti:
	
	\begin{enumerate}
		\item $A$ je izomorfní s dolní podmnožinou $B \subseteq \omega$, pak $B$ není shora omezená. Neexistuje $n \in \omega (\forall b \in B) b \in n$. Sporem $B \subseteq S(n)$ tedy $B$ by byla konečná a to je spor.
		\begin{itemize}
			\item To znamená, že $(\forall n \in \omega)$ je menší než nějaký prvek $b \in B$. $B$ je dolní množina, tedy $n \in B \rightarrow \omega \subseteq B \rightarrow \omega = B$.
		\end{itemize}
		\item $\omega$ je izomorfní dolní podmnožině $C \subseteq A$. $C$ není shora omezená, kdyby ano, tak $\exists a \in A : C \subseteq |\leftarrow, a], C$ by byla konečná, spor. $(\forall a \in A, \exists c \in C: a \subseteq c, C$ je dolní, tedy $C= A$.
	\end{enumerate}
\end{proof}

\section{Spočetné množiny}

\begin{definice}
	Množina $x$ je \textbf{spočetná}, pokud $x \approx \omega$. Množina $x$ je \textbf{nejvýše spočetná}, pokud je konečná nebo spočetná. Jinak je množina \textbf{nespočetná}.
\end{definice}

\begin{thm}
	\begin{enumerate}
		\item Každá shora omezená množina $A \subseteq \omega$ je konečná, každá shora neomezená $A \subseteq \omega$ je spočetná.
		\item Každá podmnožina spočetné množiny je nejvýše spočetná.
	\end{enumerate}
\end{thm}

\begin{proof}
	\begin{enumerate}
		\item $A$ omezená, to znamená, že $\exists n: A \subseteq S(n)$. Takže $Fin(S(n)) \rightarrow Fin(A)$.
		\begin{itemize}
			\item Pokud je $A$ neomezená, pak je nekonečná. To lze dokázat sporem, že kdyby byla konečná, pak má $A$ maximální prvek $m$, tedy je shora omezená $m$, to je spor.
			\item $A$ je lineárně uspořádaná $\in$. Pro každé $n \in A$ je $|\leftarrow ,n] \subseteq S(n)$, tedy $|\leftarrow ,n ]$ je konečná. Podle charakterizační věty $A$ je izomorfní $\omega$. Takže $A \approx \omega$.
		\end{itemize}
		\item $A$ je spočetná $f: A \to \omega$ (bijekce). $B \subseteq A$, pak $B \approx f[B] \subseteq \omega$. Podle 1) je $f[B]$ spočetná anebo konečná.
	\end{enumerate}
\end{proof}

\begin{prikl}
	\textbf{Lexikografické uspořádání} na $\omega \times \omega$.
	
	$$
	(m_{1},n_{1}) <_{L} (m_{2},n_{2}) \leftrightarrow (m_{1} \in m_{2} \lor ((m_{1} = m_{2}) \land (n_{1} \in n_{2})))
	$$
\end{prikl}

\textit{Cvičení: Ověřte, že $<_{L}$ je dobré uspořádání na $\omega \times \omega$.}

\textit{Cvičení: Ověřte, že $<_{L}$ na $\omega \times 2$ je izomorfní s $(\omega, \in)$.}

\textit{Cvičení: Ověřte, že $<_{L}$ na $2 \times \omega$ není izomorfní s $(\omega, \in)$.}

\begin{definice}
	\textbf{Maximo-lexikografické uspořádání} na $\omega \times \omega$ je:
	
	$$
	\max(m,n) =
	\left\{
	\begin{array}{ll}
		m & n \in m \\
		n & \text{ jinak}
	\end{array}
	\right.
	$$
	
	$$
	\begin{array}{c}
		(m_{1},n_{1}) <_{ML} (m_{2},n_{2}) \\
		\updownarrow \\
		((\max(m_{1},n_{1}) \in \max(m_{2},n_{2})) \lor ((\max(m_{1},n_{1}) = \\
		= \max(m_{2},n_{2})) \land ((m_{1},n_{1}) <_{L} (m_{2},n_{2}))))
	\end{array}
	$$
\end{definice}

\textit{Cvičení: Ověřte, že $\omega \times \omega <_{ML}$ je izomorfní $(\omega, \in)$.}

\begin{thm}
	Jsou-li $A,B$ spočetné množiny, pak $A \cup B$ a $A \times B$ jsou spočetné.
\end{thm}

\begin{proof}
	$f: A \to \omega$ a $g: B \to \omega$ jsou bijekce. Definujeme $h: A \cup B \to \omega \times 2 \approx \omega$ jako:
	
	$$
	h(x) =
	\left\{
	\begin{array}{ll}
		(f(x),0) & x \in A \\
		(g(x),1) & x \in B \setminus A
	\end{array}
	\right.
	$$
	
	$h$ je prosté. Tedy $A \cup B \subseteq \omega \times 2 \approx \omega \land \omega \preceq A \preceq A \cup B$ a z Cantor-Bernsteinovy věty implikuje, že $\omega \approx A \cup B$. $A \times B$ definujeme $k: A \times B \to \omega \times \omega$ jako $k((a,b)) = (f(a),g(b)), k$ je bijekce. Opět mám $A \times B \approx \omega \times \omega \approx \omega$.
\end{proof}

\begin{dusl}
	$\mathbb{Z}, \mathbb{Q}$ jsou spočetné. Kde $\mathbb{Z}$ lze modelovat jako množinu dvojic, kde první je číslo a druhé bool jestli je kladné nebo ne. A $\mathbb{Q}$ jako množinu dvojic $(m,n)$ kde je číslo nejmenší společný dělitel $(m,n) = 1$ a číslo je $\frac{m}{n}$.	
\end{dusl}

\begin{dusl}
	Konečná sjednocení, konečné součiny jsou spočetné. \textbf{Dirichletův princip}: je-li $A$ nespočetná, $A = A_{1} \cup A_{2} \cup \dots \cup A_{n}$, potom aspoň jedna množina $A_{i}$ je nespočetná. Konečná podmnožina $[A]^{< \omega}$ konečné posloupnosti jsou spočetné.
\end{dusl}

\textit{Cvičení: Je-li $A$ nespočetné, $B$ spočetná, $C$ konečná, potom $A \cup C, A \setminus C$ jsou nespočetné a $B \cup C, B \setminus C$ jsou spočetné, $A \cup B, A \setminus B$ jsou nespočetné.}

\begin{pozn}
	Spočetné sjednocení spočetně mnoha množin $\bigcup A$, kde $A$ je spočetná a $(\forall a \in A)$ jsou spočetné.
\end{pozn}

\begin{thm}[Cantor]
	$$
	x \prec \mathcal{P}(x)
	$$
\end{thm}

\begin{proof}
	Pomocí diagonální metody. $\preceq : f(y) = \{y\}, f: x \to \mathcal{P}(x)$ je prosté. Definujme $y = \{t, t \in x \land t \notin f(t)\}$. Potom $y \subseteq \mathcal{P}(x)$ nemá vzor při $f$. Kdyby
	
	$$
	f(v) = y:
	\left\{
	\begin{array}{llr}
		v \in y & \text{ pak } v \notin f(v) = y & \text{ SPOR} \\
		v \notin y = f(v) & \text{ tedy } v \in y & \text{ SPOR}
	\end{array}
	\right.
	$$
\end{proof}

\begin{dusl}
	$\mathcal{P}(\omega)$ je nespočetná.
\end{dusl}

\begin{dusl}
	$V$ není množina: $\mathcal{P}(V) \subseteq V$, kdyby byla množina, pak by musela platit Cantorova věta.
\end{dusl}

\begin{thm}
	$$
	\mathcal{P}(\omega) \approx \mathbb{R} \approx [0,1]
	$$
\end{thm}

\begin{proof}
	Víme $\mathcal{P}(\omega) \approx ^{\omega}2$ podmnožiny $\leftrightarrow$ charakteristická funkce $\leftrightarrow$ posloupnosti \newline $(a_{0},a_{1},a_{2},\dots)$, kde $a_{i} \in \{0,1\}$. $[0,1] \approx ^{\omega}2: a \in [0,1]$ zapíšu v binární soustavě tak, že pokud je to nula, tak je to nekonečně nul a jinak vždy tak, aby obsahovalo nekonečno jedniček. $\leftarrow$ použijeme trojkovou soustavu. $(a_{0},a_{1},a_{2},\dots) \to a = \sum_{n = 0}^{\infty} \frac{a_{n}}{3^{n+1}}$. Cantor-Bernstein $\rightarrow [0,1] \approx ^{\omega}2$. (pozn.: Cantorovo diskontinuum). $[0,1] \subseteq \mathbb{R}$, $\mathbb{E} \to [0,1]$ nějakou vhodnou funkci např. $\frac{\pi / 2 - \arctan(x)}{\pi}$.
\end{proof}

\begin{pozn}
	Množina algebraických čísel (tj. kořeny polynomů s racionálními koeficienty) je spočetná.
\end{pozn}

\textit{Cvičení: Pokrytí $N$ intervaly.}

\begin{enumerate}
	\item \textit{Konečně.}
	\begin{itemize}
		\item $A \subseteq I_{1} \cup I_{2} \cup \dots \cup I_{n}$ pak $\sum (b_{i} - a_{i} \geq 1$
	\end{itemize}
	\item \textit{Nekonečně.}
	\begin{itemize}
		\item $\forall \epsilon > 0: \exists I_{1},I_{2}, \dots, A \subseteq \bigcup I_{i}; \sum (b_{i} - a_{i}) < \epsilon$
	\end{itemize}
\end{enumerate}

\begin{pozn}
	\textbf{Hypotéza kontinua} je, že každá nekonečná podmnožina $\mathbb{R}$ je buď spočetná anebo ekvivalentní s $\mathbb{R}$.
\end{pozn}

\section{Axiom výběru}

\subsection{Princip výběru}

Pro každý rozklad $r$ množiny $x$ existuje \textbf{výběrová množina}. To jest $v \subseteq x$, pro kterou platí $(\forall u \in r)(\exists x)( v \cap u = \{x\})$.

\begin{definice}
	Je-li $X$ množina, pak funkce $f$ definovaná na $X$ splňující $(y \in X \land y \neq \emptyset) \rightarrow f(y) \in y$ se nazývá **selektor** na množině $X$.
\end{definice}

\subsection{10.Axiom výběru (AC - axiom of choice)}

Na každé množině existuje selektor.

\subsubsection{Ekvivalentně}

Každou množinu lze dobře uspořádat. $\leq$ je trichotomická. Zornovo lemma.

\begin{dusl}
	\begin{itemize}
		\item Každý vektorový prostor má bázi.
		\item Součin kompaktních topologických prostorů je kompaktní.
		\item Hahn-Banachova věta.
		\item Princip kompaktnosti.
		\item Banach Tarski (rozdělení koule na malé části a vytvoření dvou stejně velkých koulí).
	\end{itemize}
\end{dusl}

\begin{definice}
	(Indexový) soubor množin $<F_{j}; j \in J>$. Kde $F$ je zobrazení s definovaným obrazem $J$. Pro $j \in J: F_{j} = F(j)$. $J$ je \textbf{indexová třída} a jeho prvky jsou \textbf{indexy}.
\end{definice}

Lze definovat:

$$
\left\{
\begin{array}{l}
	\bigcup_{j \in J} F_{j} \text{ jako } \{x, (\exists j \in J) x \in F_{j})\} \\
	\bigcup_{j \in J} F_{j} = \bigcup Rng(F)
\end{array}
\right.
$$

$$
\left\{
\begin{array}{l}
	\bigcap_{j \in J} F_{j} \text{ jako } \{x, (\forall j \in J) x \in F_{j})\} \\
	\bigcap_{j \in J} F_{j} = \bigcap Rng(F)
\end{array}
\right.
$$

Kartézský součin souboru množin indexovaného množinou $J$ je $X_{j \in J} F_{j} : \{f, f: J \to \bigcup_{j \in J} F_{j} \land (\forall j \in J)f(j) \in F_{j}\}$.

\begin{lemma}
	Je-li $J$ množina, pak $XF_{j}$ je množina. Je-li $(\forall j \in J) F_{j} = Y$, pak $X_{j \in J}F_{j} = ^{J}Y$.
\end{lemma}

\begin{proof}
	Axiom nahrazení. $Rng(F)$ je množina, $\bigcup Rng(F)$ je množina. $^{J}\bigcup_{j\in J}F_{j}$ je množina. $XF_{j} \subseteq ^{J}\bigcup_{j \in J} F_{j}$.
\end{proof}

\begin{lemma}
	NTJE: (Následující tvrzení jsou si ekvivalentní.)
	
	\begin{enumerate}
		\item Axiom výběru.
		\item Princip výběru.
		\item Pro každou množinovou relaci $s$ existuje funkce $f \subseteq s$ taková, že $Dom(f) = Dom(s)$.
		\item Kartézský součin $X_{i \in x} a_{i}$ neprázdného souboru neprázdných množin je neprázdný.
	\end{enumerate}
\end{lemma}

\begin{proof}
	$1 \Rightarrow 2:$ $r$ rozklad $X$, podle 1 existuje selektor $f$ na $r$. Pak $Rng(f)$ je výběrová množina. $2 \Rightarrow 3:$ BÚNO: $s \neq \emptyset$. Vytvoříme rozklad $s$. $n = \{\{i\}\times s \shortparallel\{i\}; i \in Dom(s)\} = \{\{(i,x),(i,x) \in s\}, i \in Dom(s)\}$. Výběrová množina $n$ je funkce, která je podmnožina $s$ a má stejný definiční obor. $3 \Rightarrow 4:$ Máme soubor množin $<a_{i}, i \in x>$. Vytvoříme relaci $s = \{(i,y), i \in x \land y \in a_{i}\}$. Funkce $f \subseteq s: Dom(f) = Dom(s) = x$ je prvkem $X_{i \in x}a_{i}$. $4 \Rightarrow 1:$ $x$ množina. BÚNO: $x \neq \emptyset, \emptyset \in X$. $ID \upharpoonright x$ určuje soubor $<y;y \in x>$. Každý prvek $X_{y \in x}y$ je selektor na $x$.
\end{proof}

\begin{lemma}
	Sjednocení spočetného souboru spočetných množin je spočetné. (Popřípadě je všude místo spočetné nejvýše spočetné.)
\end{lemma}

\begin{proof}
	Soubor $<B_{j};j \in J>$. BŮNO: $I = \omega$. Najděme prosté zobrazení $\bigcup_{j \in \omega} B_{j}$ do $\omega \times \omega$. Uvažujme soubor $<E_{j}; j \in \omega>$ kde $E_{j}$ je množina všech prostých zobrazení $B_{j}$ do $\omega$. Podle lemma 4) je $X_{j \in \omega}E_{j}$ neprázdný, tedy existuje soubor $<f_{j}; j \in \omega>$, kde $f_{j} \in F_{j}$. Definujme $h; \bigcup_{j \in \omega}B_{j} \to \omega\times\omega$ jako $h(x) = (j, f_{j}(x))$. Kde $j$ je nejmenší prvek $\omega$ pro který $x \in B_{j}$.
\end{proof}

\begin{pozn}
	Bez AC je bezesporné ZF a to, že "$\mathbb{R}$ jsou spočetným sjednocením spočetných množin".
\end{pozn}

\section{Princip maximality (PM)}

\begin{itemize}
	\item AC $\leftrightarrow$ PM
	\item Je-li $A$ množina uspořádaná relací $\leq$ tak, že každý řetězec má horní mez.
	\item Pak pro každé $a \in A$ existuje maximální prvek $b \in A$ takový, že $a \leq b$.
\end{itemize}

\begin{definice}
	$B \subseteq A$ je \textbf{řetězec} pokud $B$ je lineárně uspořádaná $\leq$.
\end{definice}

\begin{pozn}
	V aplikacích často pro $(A, \subseteq); A \subseteq \mathcal{P}(x)$ stačí ověřit, že $\bigcup B \in A$.
\end{pozn}

\textit{Cvičení: Ukažte pomocí PM: Je-li $(A, \leq)$ uspořádaná množina, pak pro každý řetězec $B \subseteq A$ existuje maximální řetězec $C$ splňující $B \subseteq C \subseteq A$.}

\subsection{Princip maximality II (PMS)}

Je-li $(A, \leq)$ uspořádaná množina, kde každý řetězec má suprémum, pak pro každé $a \in A$ existuje $b \in A$ maximální prvek splňující $a \leq b$.

\textit{Cvičení: Dokažte: PM$\leftrightarrow$PMS.}

\section{Princip trichotomie $\preceq$ (PT)}

Pro každé dvě množiny $x,y$ platí $x \preceq y$ nebo $y \preceq x$.

\begin{lemma}
	PM $\rightarrow$ PT.
\end{lemma}

\begin{proof}
	Definuji množinu $D = \{f, f \text{ prosté zobrazení } \land Dom(f) \subseteq x \land Rng(f) \subseteq y \}$. $(D, \subseteq)$ splňuje předpoklady PM. Tedy má maximální prvek $g$. Kdyby $x \setminus Dom(f) \neq \emptyset$ a $y \setminus Rng(f) \neq \emptyset$, pak lze $g$ rozšířit o novou dvojici $(u,v)$, spor s maximalitou $g$. Pokud $Dom(f) = x$, pak $x \preceq y$. Pokud $Rng(f) = y$, pak $g^{-1}$ je prosté zobrazení $y$ do $x$, tedy $y \preceq y$.
\end{proof}

\textit{Cvičení: Sjednocení řetězce prostých zobrazení je prosté zobrazení.}

\section{Princip dobrého uspořádání (VVO)}

\begin{itemize}
	\item Každou množinu lze dobře uspořádat.
	\item Známo jako Zermelova věta.
	\item AC $\leftrightarrow$ VVO
\end{itemize}

\begin{lemma}
	VVO $\rightarrow$ AC
\end{lemma}

\begin{proof}
	$x \neq \emptyset, \emptyset \notin x$ podle VVO máme dobré uspořádání na $\bigcup x$. Každý $y \in x$ je neprázdná podmnožina $\bigcup x$, tedy má nejmenší prvek $\min_{\leq}y$. Definujeme $f: x \to \bigcup x$ jako $f(y) = \min_{\leq}(y)$. Tato $f$ je selektorem na množině $x$.
\end{proof}

\textit{Cvičení: PM $\rightarrow$ VVO}
	\chapter{Ordinální čísla}

\section{"Typy dobře uspořádaných množin.”}

\begin{itemize}
	\item Kardinální čísla $\subseteq$ ordinální čísla. Mohutnosti dobře uspořádaných množin. S (AC) mohutnosti všech množin.
	\item Ordinální čísla jsou dobře uspořádaná $\in$, platí pro ně princip transfinitní indukce.
\end{itemize}

\begin{definice}
	Třída $X$ je \textbf{tranzitivní} pokud $x \in X \rightarrow x \subseteq X$.
\end{definice}

\begin{prikl}
	$\omega$ i každé $n \in \omega$ jsou tranzitivní i $V$.
\end{prikl}

\textit{Cvičení: $X$ tranzitivní $\leftrightarrow \bigcup X \subseteq X$}

\begin{lemma}
	\begin{enumerate}
		\item Jsou-li $X,Y$ tranzitivní pak $X \cap Y, X \cup Y$ jsou tranzitivní.
		\item $X$ třída, pro kterou každé $x \in X$ je tranzitivní množina, pak $\bigcap X \text{ a } \bigcup X$ jsou tranzitivní.
		\item Je-li $X$ tranzitivní třída, pak $\in$ je tranzitivní na $X \leftrightarrow$ každý $x \in X$ je tranzitivní množina.
	\end{enumerate}
\end{lemma}

\begin{proof}
	\begin{enumerate}
		\item Je pozorování.
		\item Plyne analogicky z 1.
		\item Jako \textit{Cvičení}.
	\end{enumerate}
\end{proof}

\begin{definice}
	Množina $x$ je \textbf{ordinální číslo (ordinála)} pokud $x$ je tranzitivní množina a $\in$ je dobré uspořádání na $x$. Třídu všech ordinálních čísel značíme $On$.
\end{definice}

\begin{prikl}
	$\omega$ a každé $n \in \omega$ je ordinální číslo.
\end{prikl}

\begin{dusl}
	Pro každou nekonečnou množinu $x$ platí $\omega \preceq x$.
\end{dusl}

\begin{lemma}
	$On$ je tranzitivní třída.
\end{lemma}

\begin{proof}
	$y \in x \in On$. Máme $y \leq x, \in$ je dobré ostré uspořádání na $y$. $\in$ je dobré ostré na $x$. Z lemma 3) je $y$ tranzitivní množina. $y$ je ordinála.
\end{proof}

\begin{lemma}
	$\in$ je tranzitivní na $On$.
\end{lemma}

\begin{lemma}
	$x,y \in On$, pak:
	
	\begin{enumerate}
		\item $x \notin x$
		\item $x \cap y \in On$
		\item $x \in y \leftrightarrow x \subset y$
	\end{enumerate}
\end{lemma}

\begin{proof}
	\begin{enumerate}
		\item Sporem z antireflexivity $\in$ na $x$.
		\item Přímo z definice.
		\item $\rightarrow$ z tranzitivity $y$ a 1)
	\end{enumerate}
	
	$\leftarrow y \setminus x \neq \emptyset \subseteq y, y \setminus x$ má nejmenší prvek $z$. Platí $z = x$ (\textit{Cvičení}).
\end{proof}

\begin{thm}
	$\in$ je dobré ostré uspořádání třídy $On$.
\end{thm}

\begin{proof}
	Antireflexivita z lemma 1), tranzitivita pak dohromady dává ostré uspořádání. Trichotomie: $x \neq y \in On$ podle lemma 2) $x \cap y \in On$. Sporem kdyby $x \cap y \subset x \land x \subset y$ pak $x \cap y \in y \land x \cap y \in x$, tedy $x \cap y \in x \cap y$ a to je spor s lemma 1). Když tedy $x \cap y = x$ pak $x \subset y$ tedy $x \in y$. Z toho plyne, že se jedná o lineární uspořádání. Pro dobrost stačí existence minimálního prvku (\textit{Cvičení}).
\end{proof}

\begin{dusl}
	$On$ je vlastní třída. Je-li $X$ vlastní třída, tranzitivní, dobře uspořádaná $\in$, pak $X = On$.
\end{dusl}


\subsection{Značení:}

\begin{itemize}
	\item $\alpha, \beta, \gamma, \dots$ jsou ordinální čísla.
	\item $\alpha < \beta$ místo $\alpha \in \beta$.
	\item $\alpha \leq \beta$ místo $\alpha \in beta \lor \alpha = \beta$.
\end{itemize}

\begin{lemma}
	\begin{enumerate}
		\item Množina $x \subseteq On$ je ordinální číslo $\leftrightarrow x$ je tranzitivní.
		\item $A \subseteq On, A \neq \emptyset$, pak $\bigcap A$ je nejmenší prvek $A$ vzhledem k $\leq$.
		\item $a \subseteq On$ množina, pak $\bigcup a \in On$ a $\bigcup a = \sup_{\leq}a$.
	\end{enumerate}
\end{lemma}

\begin{proof}
	\begin{enumerate}
		\item $\rightarrow$ z definice, $\leftarrow$ z věty.
		\item Z věty a $\bigcap A = \inf A$.
		\item $\bigcup a$ je tranzitivní, $\bigcup a \subseteq On$ podle 1) je ordinální číslo.
	\end{enumerate}
\end{proof}

\begin{dusl}
	$\omega$ je supremum množiny všech přirozených čísel v $On$. Konečné ordinály jsou právě přirozená čísla.
\end{dusl}

\textit{Cvičení: Důkaz: $\bigcup \omega \in On \land \bigcup \omega = \sup_{\leq}\omega$. Zbývá ověřit $\omega = \bigcup \omega$.}

\begin{lemma}
	$\alpha \in On$, pak $\alpha \cup \{\alpha\}$ je nejmenší ordinální číslo větší než $\alpha$.
\end{lemma}

\begin{proof}
	$\alpha \subseteq On$ protože $On$ je tranzitivní. $\alpha \cup \{\alpha\}$ je tranzitivní množina ordinálních čísel. Podle lemma 1) $\alpha \cup \{\alpha\}$ je ordinální číslo. Je-li $\beta \in On, \beta \in \alpha \{\alpha\}$, pak $\beta \in \alpha \lor \beta = \alpha$ tedy $\beta \subseteq \alpha$.
\end{proof}

\begin{definice}
	$\alpha \cup \{\alpha\}$ je \textbf{následník} $\alpha$. $\alpha$ je \textbf{předchůdce} $\alpha \cup \{\alpha\}$. $\alpha$ je \textbf{izolované} pokud $\alpha = 0$ nebo pokud $\alpha$ má předchůdce, jinak je \textbf{limitní}.
\end{definice}

\begin{thm}[O typu dobrého uspořádání.]
	Je-li $a$ množina dobře uspořádaná relací $r$, pak existuje právě jedno ordinální číslo $\alpha$ a právě jeden izomorfismus $(a,r)$ a $(\alpha, \leq)$. (Bez důkazu.)
\end{thm}

\begin{definice}
	$\alpha$ je \textbf{typ} dobrého uspořádání $r$.
\end{definice}

\begin{pozn}
	Na ${On}^{2} = On \times On$ lze definovat lexikografické uspořádání i maximo-\newline -lexikografické uspořádání.
\end{pozn}

\section{Princip transfinitní indukce}

Je-li $A \subseteq On$ třída splňující $(\forall \alpha \in On)(\alpha \subseteq A \rightarrow \alpha \in A)$, potom $A = On$.

\begin{proof}
	Sporem: $On \setminus A \neq \emptyset$ díky dobrému uspořádání $\in$ existuje nejmenší prvek $\alpha \in On \setminus A$. Potom každé $\beta \in \alpha$ už je prvkem $A$, tedy $\alpha \subseteq A$, z předpokladu věty $\alpha \in A$ a to je spor.
\end{proof}

\begin{thm}[Druhá verze principu transfinitní indukce.]
	Je-li $A \subseteq On$ třída splňující:
	
	\begin{enumerate}
		\item $0 \in A$
		\item Pro každý $\alpha \in On$ platí $\alpha \in A \rightarrow \alpha \cup \{\alpha\} \in A$.
		\item Je-li $\alpha$ lineární pak $\alpha \subseteq A \rightarrow \alpha \in A$.
	\end{enumerate}
	
	Pak $A = On$.
\end{thm}

\begin{thm}[O konstrukci transfinitních rekurzí.]
	Je-li $G: V \to V$ třídové zobrazení, pak existuje právě jedno zobrazení $F: On \to V$ splňující $(\forall \alpha \in On) F(\alpha) = G(F \upharpoonright \alpha)$.
	
	Varianty:
	
	\begin{itemize}
		\item $F(\alpha = G(F[\alpha])$
		\item $F(\alpha) = G(\alpha , F \upharpoonright \alpha)$
		\item $G_{1}(F(\beta))$ je-li $\alpha$ následník $\beta$, jinak $G_{2}(F[\alpha])$ je-li $\alpha$ limitní.
	\end{itemize}
\end{thm}

\begin{proof}
	Je pomocí transfinitní indukce a axiomu nahrazení.
\end{proof}

\begin{prikl}
	$m + n: F(m) = n+m$ se dá nadefinovat jako $F(0) = n, F(S(m)) = S(F(m))$. AC $\to$ VVO: $A$ množina $g$ selektor na $\mathcal{P}(A)$ tak $f(0) = g(A)$ a $f(\beta) = g(A - f[\beta])$.
\end{prikl}
\end{document}

