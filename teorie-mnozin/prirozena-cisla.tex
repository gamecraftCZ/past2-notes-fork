\chapter{Přirozená čísla}

\begin{definice}[von Neumann]
	$0 = \emptyset; 1 = \{0\} = \{\emptyset\}; 2 = \{0,1\} = \{\emptyset, \{\emptyset \}\}; 3 = \{0,1,2\} = \dots$, Myšlenka: “Přirozené číslo je množina všech menších přirozených čísel.”
\end{definice}

\begin{definice}
	$w$ je \textbf{induktivní množina}, pokud $\emptyset \in w \land (\forall v \in w)(v \cup \{v\} \in w)$.
\end{definice}

\section{9.Axiom nekonečna (“Existuje induktivní množina.”)}

$$
(\exists z)(0 \in z \land (\forall x)(x \in z \rightarrow x \cup \{x\} \in z))
$$

\begin{definice}
	\textbf{Množina všech přirozených čísel} $\omega$ je $\bigcap\{w, w \text{ je induktivní množina}\}$.
\end{definice}

\begin{lemma}
	$\omega$ je nejmenší induktivní množina.
\end{lemma}

\begin{proof}
	$0 \in \omega$, $x \in \omega, x$ patří do každé induktivní množiny. $x \cup \{x\}$ patří do každé induktivní množiny. $x \cup \{x\} \in \omega$.
\end{proof}

Prvky $\omega$ jsou \textbf{přirozená čísla} v teorii množin.

\begin{definice}
	Funkce následník $S: \omega\to\omega$. Pro $v \in \omega: S(v) = v \cup \{v\}$. “Následník čísla $v$.”
\end{definice}

\begin{thm}[Princip (slabé) indukce pro přirozená čísla.]
	Je-li $X \subseteq \omega$ taková, že platí:
	
	\begin{enumerate}
		\item $0 \in X$,
		\item $x \in X \rightarrow S(x) \in X$. Pak $X = \omega$.
	\end{enumerate}
\end{thm}

\begin{proof}
	1 a 2 dohromady říká, že $X$ je induktivní, tedy $\omega \subseteq X$.
\end{proof}

\begin{prikl}
	Důkaz indukcí: Chceme dokázat: $(\forall n \in \omega)(\varphi(n))$. Dokazujeme: 1. $\varphi(0)$ a 2. $(\forall n \in \omega)(\varphi(n) \rightarrow \varphi(S(n)))$.
\end{prikl}

\begin{pozn}
	Princip silné indukce: 2: $((\forall m \in \omega) m \in X) \rightarrow n \in X$.
\end{pozn}

\begin{lemma}[$\in$ je ostré uspořádání]
	Pro libovolné $m,n \in \omega$ platí:
	
	\begin{enumerate}
		\item $n \in \omega \rightarrow n \subseteq \omega$
		\begin{itemize}
			\item “Prvky přirozených čísel jsou přirozená čísla.”
		\end{itemize}
		\item $m \in n \rightarrow m \subseteq n$
		\begin{itemize}
			\item “Náležení je tranzitivní na $\omega$.”
		\end{itemize}
		\item $n \nsubseteq n$
		\begin{itemize}
			\item "$\in$ je antireflexivní na $\omega$."
		\end{itemize}
	\end{enumerate}
	
	Z toho všeho plyne, že se jedná o ostré uspořádání.
\end{lemma}

\begin{proof}
	Indukcí:
	
	\begin{enumerate}
		\item $0 \subseteq \omega$, a indukční krok $n \in \omega$, předpokládáme, že $n \subseteq \omega$. Pak $\{n\} \subseteq \omega$ tedy $n \cup \{n\} \subseteq \omega$.
		\item Indukcí podle $n$:
		\begin{itemize}
			\item 1. Krok: $m \notin 0$ tím pádem implikace splněna.
			\item 2.  Krok $X = \{n, n \in \omega \land (\forall m)(m \in n \rightarrow m \subseteq n)\}$.
			\item Víme $0 \in X$.
			\item Nechť $n \in X$, víme $S(n) \in \omega$.
			\item Nechť $m \in S(n) = n \cup \{n\}$. Pak buď $m \in n$ a z IP pak $m \subseteq n$ anebo $m = n$ tím pádem také $m \subseteq n \subseteq S(n)$.
		\end{itemize}
		\item $0 \nsubseteq 0$ platí, nechť $n \in \omega$ a $n \nsubseteq n$.
		\begin{itemize}
			\item Sporem $S(n) \subseteq S(n) = n \cup \{n\}$. Z toho pak plyne, že buď $S(n) \subseteq \{n\}$ anebo $S(n) \subseteq n$. V obou případech je $S(n) \subseteq n$, ale to pak znamená, že $n \in S(n) \subseteq n$ což je spor s předpokladem.
		\end{itemize}
	\end{enumerate}
\end{proof}

\begin{lemma}
	Každé přirozené číslo je konečná množina.
\end{lemma}

\begin{proof}
	Indukcí: $Fin(\emptyset)$ víme. Podle lemma $Fin(x) \rightarrow (\forall y)Fin(x \cup \{y\})$, speciálně pro $Fin(x \cup \{x\})$ a to je následník.
\end{proof}

\begin{thm}
	Množina $x$ je konečná právě tehdy, když $(\exists n \in \omega) x \approx n$.
\end{thm}

\begin{proof}
	$\Leftarrow Fin(n)$ tedy $Fin(x)$. $\Rightarrow$ indukcí: $X = \{x; (\exists n \in \omega) x \approx n\}$. Víme, že $0 \in X$ protože $0 \approx 0$. Nechť $x \in X, y$ množina. Víme, že $(\exists n \in \omega) n \approx x$. $y \in x$ pak $x \cup \{y\} = x \approx n$, $y \notin x$ pak $x \cup \{y\} \approx S(n) = n \cup \{n\}$. K bijekci $x$ a $n$ přidáme $(y,n)$. Tedy $Fin \subseteq X$.
\end{proof}

\begin{lemma}
	Množina $\omega$ i každá induktivní množina je nekonečná.
\end{lemma}

\begin{proof}
	Podle lemma: 1 $n \in \omega \rightarrow n \subseteq \omega$, tedy $n \in \mathcal{P}(n)$ tedy $\omega \subseteq \mathcal{P}(n), \omega$ je neprázdná ale nemá maximální prvek vzhledem k inkluzi. Když $n \subseteq \omega$ pak podle lemma 3. $n \nsubseteq n$ a tedy $n \subset n \cup \{n\} = S(n)$. $\omega \subseteq W$ tedy i induktivní množiny.
\end{proof}

\textit{Cvičení: $\omega$ je Dedekindovsky nekonečná.}

\begin{lemma}[Linearita $\in$ na $\omega$.]
	$m,n \in \omega$, platí:
	
	\begin{enumerate}
		\item $m \in n \leftrightarrow m \subset n$
		\item $m \in n \lor m = n \lor n \in m$ (\textit{trichotomie})
	\end{enumerate}
\end{lemma}

\begin{proof}
	\begin{enumerate}
		\item $\rightarrow$ plyne z lemma 2 $m \in n \rightarrow m \subset n \land n \nsubseteq n$
		\begin{itemize}
			\item $\leftarrow$ indukcí podle $n$; $n = 0$ nelze splnit.
			\item Indukční krok. Nechť platí pro nějaké $n$ a $\forall m$.
			\item Nechť $m \subset S(n) = n \cup \{n\}$ a $m \subseteq n$, kdyby ne pak $n \in m$ tedy $n \subseteq m$ tedy $S(n) = n \cup \{n\} \subseteq m$ a to je spor.
			\item $m \subset n$ z IP $m \in n \subseteq S(n)$ tedy $m \in S(n)$
			\item $m = n$ pak $n \in S(n)$
		\end{itemize} 
		\item Pro $n \in \omega$ nechť $A(n) = \{m \in \omega, m \in n \lor m =n \lor n \in m\}$.
		\begin{itemize}
			\item Dokážeme, že $A(n)$ je induktivní, indukcí podle $m$.
			\item $n = 0: 0 \in A(0)$, protože $0 = 0$
			\item Je-li $m \in A(0)$, pak: $m = 0: 0 \in \{m\}$ anebo $0 \in m$ a z obou plyne $0 \in m \cup \{m\} = S(n)$.
			\item Tedy $S(n) \in A(0)$.
			\item Tedy $A(0) = \omega$.
			\item Tedy také $(\forall n \in \omega) 0 \in A(n)$.
			\item $n \in \omega, m \in \omega$, předpokládejme, že $m \in A(n)$. Ukážeme, že $S(m) \in A(n)$.
			\item $m \in n \rightarrow m \subset n; \{m\} \subseteq n$ tedy $S(m) \subseteq n$ z toho plyne, že $S(m) = n \lor S(m) \in n$.
			\item $m = n \lor n \in m$ potom $n \in m \cup \{m\} = S(m)$
			\item Ve všech případech ke $S(m) \in A(n)$.
		\end{itemize}
	\end{enumerate}
\end{proof}

\begin{thm}
	Množina $\omega$ je dobře (ostře) uspořádaná relací $\in$.
\end{thm}

\begin{proof}
	Nechť $a \subseteq \omega, a \neq \emptyset$. Zvolme $n \in a$. Není-li $n$ nejmenší (minimální), tak definuji $b = n \cap a$. $n$ je konečná, tak i $b$ je konečná a neprázdná. $b \subseteq \omega$ tedy $b$ má minimální prvek $m$ vzhledem k náležení. $m$ je minimální i v množině $a$: kdyby $(\exists x \in a) x \in m$, tak víme, že $m \in n$, tedy $m \subseteq n$, tedy $x \in n$, tedy $x \in b$. To je spor s minimalitou $m$ v $b$. $\in$ je lineární na $\omega$, tedy $m$ je nejmenší prvek v $a$. Tedy $\in$ je dobré uspořádání.
\end{proof}

\begin{pozn}
	Nekonečná množina $A$ s lineárním (ostrým) uspořádáním $<$ pro každé $a \in A: |\leftarrow, a]$ je konečná. Pak $<$ je dobré a $(A,<)$ je izomorfní $(\omega, \in)$.
\end{pozn}

\begin{thm}[Charakterizace uspořádání $\in$ na $\omega$]
	Nechť $A$ je nekonečná množina, lineárně uspořádaná (ostře) relací $<$ tak, že pro každé $a \in A$ je dolní množina $|\leftarrow , a]$ konečná. Pak $<$ je dobré a množiny $A, \omega$ jsou izomorfní vzhledem k $<, \in$.
\end{thm}

\begin{proof}
	$<$ je dobré: $\emptyset \neq c \in A$. Nechť $a \in c$, předpokládejme, že $a$ není minimální v $c$, pak definujeme $b = c \cap |\leftarrow, a]$. $b$ je konečná. Tedy má minimální prvek $m, m$ je minimální i v $c$. Protože $m \leq a$, pak $x \leq a$ tedy $x \in |\leftarrow, a]$ tedy $x \in b$ a to je spor. Izomorfismus: podle věty o porovnávání dobrých uspořádání jsou 2 možnosti:
	
	\begin{enumerate}
		\item $A$ je izomorfní s dolní podmnožinou $B \subseteq \omega$, pak $B$ není shora omezená. Neexistuje $n \in \omega (\forall b \in B) b \in n$. Sporem $B \subseteq S(n)$ tedy $B$ by byla konečná a to je spor.
		\begin{itemize}
			\item To znamená, že $(\forall n \in \omega)$ je menší než nějaký prvek $b \in B$. $B$ je dolní množina, tedy $n \in B \rightarrow \omega \subseteq B \rightarrow \omega = B$.
		\end{itemize}
		\item $\omega$ je izomorfní dolní podmnožině $C \subseteq A$. $C$ není shora omezená, kdyby ano, tak $\exists a \in A : C \subseteq |\leftarrow, a], C$ by byla konečná, spor. $(\forall a \in A, \exists c \in C: a \subseteq c, C$ je dolní, tedy $C= A$.
	\end{enumerate}
\end{proof}

\section{Spočetné množiny}

\begin{definice}
	Množina $x$ je \textbf{spočetná}, pokud $x \approx \omega$. Množina $x$ je \textbf{nejvýše spočetná}, pokud je konečná nebo spočetná. Jinak je množina \textbf{nespočetná}.
\end{definice}

\begin{thm}
	\begin{enumerate}
		\item Každá shora omezená množina $A \subseteq \omega$ je konečná, každá shora neomezená $A \subseteq \omega$ je spočetná.
		\item Každá podmnožina spočetné množiny je nejvýše spočetná.
	\end{enumerate}
\end{thm}

\begin{proof}
	\begin{enumerate}
		\item $A$ omezená, to znamená, že $\exists n: A \subseteq S(n)$. Takže $Fin(S(n)) \rightarrow Fin(A)$.
		\begin{itemize}
			\item Pokud je $A$ neomezená, pak je nekonečná. To lze dokázat sporem, že kdyby byla konečná, pak má $A$ maximální prvek $m$, tedy je shora omezená $m$, to je spor.
			\item $A$ je lineárně uspořádaná $\in$. Pro každé $n \in A$ je $|\leftarrow ,n] \subseteq S(n)$, tedy $|\leftarrow ,n ]$ je konečná. Podle charakterizační věty $A$ je izomorfní $\omega$. Takže $A \approx \omega$.
		\end{itemize}
		\item $A$ je spočetná $f: A \to \omega$ (bijekce). $B \subseteq A$, pak $B \approx f[B] \subseteq \omega$. Podle 1) je $f[B]$ spočetná anebo konečná.
	\end{enumerate}
\end{proof}

\begin{prikl}
	\textbf{Lexikografické uspořádání} na $\omega \times \omega$.
	
	$$
	(m_{1},n_{1}) <_{L} (m_{2},n_{2}) \leftrightarrow (m_{1} \in m_{2} \lor ((m_{1} = m_{2}) \land (n_{1} \in n_{2})))
	$$
\end{prikl}

\textit{Cvičení: Ověřte, že $<_{L}$ je dobré uspořádání na $\omega \times \omega$.}

\textit{Cvičení: Ověřte, že $<_{L}$ na $\omega \times 2$ je izomorfní s $(\omega, \in)$.}

\textit{Cvičení: Ověřte, že $<_{L}$ na $2 \times \omega$ není izomorfní s $(\omega, \in)$.}

\begin{definice}
	\textbf{Maximo-lexikografické uspořádání} na $\omega \times \omega$ je:
	
	$$
	\max(m,n) =
	\left\{
	\begin{array}{ll}
		m & n \in m \\
		n & \text{ jinak}
	\end{array}
	\right.
	$$
	
	$$
	\begin{array}{c}
		(m_{1},n_{1}) <_{ML} (m_{2},n_{2}) \\
		\updownarrow \\
		((\max(m_{1},n_{1}) \in \max(m_{2},n_{2})) \lor ((\max(m_{1},n_{1}) = \\
		= \max(m_{2},n_{2})) \land ((m_{1},n_{1}) <_{L} (m_{2},n_{2}))))
	\end{array}
	$$
\end{definice}

\textit{Cvičení: Ověřte, že $\omega \times \omega <_{ML}$ je izomorfní $(\omega, \in)$.}

\begin{thm}
	Jsou-li $A,B$ spočetné množiny, pak $A \cup B$ a $A \times B$ jsou spočetné.
\end{thm}

\begin{proof}
	$f: A \to \omega$ a $g: B \to \omega$ jsou bijekce. Definujeme $h: A \cup B \to \omega \times 2 \approx \omega$ jako:
	
	$$
	h(x) =
	\left\{
	\begin{array}{ll}
		(f(x),0) & x \in A \\
		(g(x),1) & x \in B \setminus A
	\end{array}
	\right.
	$$
	
	$h$ je prosté. Tedy $A \cup B \subseteq \omega \times 2 \approx \omega \land \omega \preceq A \preceq A \cup B$ a z Cantor-Bernsteinovy věty implikuje, že $\omega \approx A \cup B$. $A \times B$ definujeme $k: A \times B \to \omega \times \omega$ jako $k((a,b)) = (f(a),g(b)), k$ je bijekce. Opět mám $A \times B \approx \omega \times \omega \approx \omega$.
\end{proof}

\begin{dusl}
	$\mathbb{Z}, \mathbb{Q}$ jsou spočetné. Kde $\mathbb{Z}$ lze modelovat jako množinu dvojic, kde první je číslo a druhé bool jestli je kladné nebo ne. A $\mathbb{Q}$ jako množinu dvojic $(m,n)$ kde je číslo nejmenší společný dělitel $(m,n) = 1$ a číslo je $\frac{m}{n}$.	
\end{dusl}

\begin{dusl}
	Konečná sjednocení, konečné součiny jsou spočetné. \textbf{Dirichletův princip}: je-li $A$ nespočetná, $A = A_{1} \cup A_{2} \cup \dots \cup A_{n}$, potom aspoň jedna množina $A_{i}$ je nespočetná. Konečná podmnožina $[A]^{< \omega}$ konečné posloupnosti jsou spočetné.
\end{dusl}

\textit{Cvičení: Je-li $A$ nespočetné, $B$ spočetná, $C$ konečná, potom $A \cup C, A \setminus C$ jsou nespočetné a $B \cup C, B \setminus C$ jsou spočetné, $A \cup B, A \setminus B$ jsou nespočetné.}

\begin{pozn}
	Spočetné sjednocení spočetně mnoha množin $\bigcup A$, kde $A$ je spočetná a $(\forall a \in A)$ jsou spočetné.
\end{pozn}

\begin{thm}[Cantor]
	$$
	x \prec \mathcal{P}(x)
	$$
\end{thm}

\begin{proof}
	Pomocí diagonální metody. $\preceq : f(y) = \{y\}, f: x \to \mathcal{P}(x)$ je prosté. Definujme $y = \{t, t \in x \land t \notin f(t)\}$. Potom $y \subseteq \mathcal{P}(x)$ nemá vzor při $f$. Kdyby
	
	$$
	f(v) = y:
	\left\{
	\begin{array}{llr}
		v \in y & \text{ pak } v \notin f(v) = y & \text{ SPOR} \\
		v \notin y = f(v) & \text{ tedy } v \in y & \text{ SPOR}
	\end{array}
	\right.
	$$
\end{proof}

\begin{dusl}
	$\mathcal{P}(\omega)$ je nespočetná.
\end{dusl}

\begin{dusl}
	$V$ není množina: $\mathcal{P}(V) \subseteq V$, kdyby byla množina, pak by musela platit Cantorova věta.
\end{dusl}

\begin{thm}
	$$
	\mathcal{P}(\omega) \approx \mathbb{R} \approx [0,1]
	$$
\end{thm}

\begin{proof}
	Víme $\mathcal{P}(\omega) \approx ^{\omega}2$ podmnožiny $\leftrightarrow$ charakteristická funkce $\leftrightarrow$ posloupnosti \newline $(a_{0},a_{1},a_{2},\dots)$, kde $a_{i} \in \{0,1\}$. $[0,1] \approx ^{\omega}2: a \in [0,1]$ zapíšu v binární soustavě tak, že pokud je to nula, tak je to nekonečně nul a jinak vždy tak, aby obsahovalo nekonečno jedniček. $\leftarrow$ použijeme trojkovou soustavu. $(a_{0},a_{1},a_{2},\dots) \to a = \sum_{n = 0}^{\infty} \frac{a_{n}}{3^{n+1}}$. Cantor-Bernstein $\rightarrow [0,1] \approx ^{\omega}2$. (pozn.: Cantorovo diskontinuum). $[0,1] \subseteq \mathbb{R}$, $\mathbb{E} \to [0,1]$ nějakou vhodnou funkci např. $\frac{\pi / 2 - \arctan(x)}{\pi}$.
\end{proof}

\begin{pozn}
	Množina algebraických čísel (tj. kořeny polynomů s racionálními koeficienty) je spočetná.
\end{pozn}

\textit{Cvičení: Pokrytí $N$ intervaly.}

\begin{enumerate}
	\item \textit{Konečně.}
	\begin{itemize}
		\item $A \subseteq I_{1} \cup I_{2} \cup \dots \cup I_{n}$ pak $\sum (b_{i} - a_{i} \geq 1$
	\end{itemize}
	\item \textit{Nekonečně.}
	\begin{itemize}
		\item $\forall \epsilon > 0: \exists I_{1},I_{2}, \dots, A \subseteq \bigcup I_{i}; \sum (b_{i} - a_{i}) < \epsilon$
	\end{itemize}
\end{enumerate}

\begin{pozn}
	\textbf{Hypotéza kontinua} je, že každá nekonečná podmnožina $\mathbb{R}$ je buď spočetná anebo ekvivalentní s $\mathbb{R}$.
\end{pozn}

\section{Axiom výběru}

\subsection{Princip výběru}

Pro každý rozklad $r$ množiny $x$ existuje \textbf{výběrová množina}. To jest $v \subseteq x$, pro kterou platí $(\forall u \in r)(\exists x)( v \cap u = \{x\})$.

\begin{definice}
	Je-li $X$ množina, pak funkce $f$ definovaná na $X$ splňující $(y \in X \land y \neq \emptyset) \rightarrow f(y) \in y$ se nazývá **selektor** na množině $X$.
\end{definice}

\subsection{10.Axiom výběru (AC - axiom of choice)}

Na každé množině existuje selektor.

\subsubsection{Ekvivalentně}

Každou množinu lze dobře uspořádat. $\leq$ je trichotomická. Zornovo lemma.

\begin{dusl}
	\begin{itemize}
		\item Každý vektorový prostor má bázi.
		\item Součin kompaktních topologických prostorů je kompaktní.
		\item Hahn-Banachova věta.
		\item Princip kompaktnosti.
		\item Banach Tarski (rozdělení koule na malé části a vytvoření dvou stejně velkých koulí).
	\end{itemize}
\end{dusl}

\begin{definice}
	(Indexový) soubor množin $<F_{j}; j \in J>$. Kde $F$ je zobrazení s definovaným obrazem $J$. Pro $j \in J: F_{j} = F(j)$. $J$ je \textbf{indexová třída} a jeho prvky jsou \textbf{indexy}.
\end{definice}

Lze definovat:

$$
\left\{
\begin{array}{l}
	\bigcup_{j \in J} F_{j} \text{ jako } \{x, (\exists j \in J) x \in F_{j})\} \\
	\bigcup_{j \in J} F_{j} = \bigcup Rng(F)
\end{array}
\right.
$$

$$
\left\{
\begin{array}{l}
	\bigcap_{j \in J} F_{j} \text{ jako } \{x, (\forall j \in J) x \in F_{j})\} \\
	\bigcap_{j \in J} F_{j} = \bigcap Rng(F)
\end{array}
\right.
$$

Kartézský součin souboru množin indexovaného množinou $J$ je $X_{j \in J} F_{j} : \{f, f: J \to \bigcup_{j \in J} F_{j} \land (\forall j \in J)f(j) \in F_{j}\}$.

\begin{lemma}
	Je-li $J$ množina, pak $XF_{j}$ je množina. Je-li $(\forall j \in J) F_{j} = Y$, pak $X_{j \in J}F_{j} = ^{J}Y$.
\end{lemma}

\begin{proof}
	Axiom nahrazení. $Rng(F)$ je množina, $\bigcup Rng(F)$ je množina. $^{J}\bigcup_{j\in J}F_{j}$ je množina. $XF_{j} \subseteq ^{J}\bigcup_{j \in J} F_{j}$.
\end{proof}

\begin{lemma}
	NTJE: (Následující tvrzení jsou si ekvivalentní.)
	
	\begin{enumerate}
		\item Axiom výběru.
		\item Princip výběru.
		\item Pro každou množinovou relaci $s$ existuje funkce $f \subseteq s$ taková, že $Dom(f) = Dom(s)$.
		\item Kartézský součin $X_{i \in x} a_{i}$ neprázdného souboru neprázdných množin je neprázdný.
	\end{enumerate}
\end{lemma}

\begin{proof}
	$1 \Rightarrow 2:$ $r$ rozklad $X$, podle 1 existuje selektor $f$ na $r$. Pak $Rng(f)$ je výběrová množina. $2 \Rightarrow 3:$ BÚNO: $s \neq \emptyset$. Vytvoříme rozklad $s$. $n = \{\{i\}\times s \shortparallel\{i\}; i \in Dom(s)\} = \{\{(i,x),(i,x) \in s\}, i \in Dom(s)\}$. Výběrová množina $n$ je funkce, která je podmnožina $s$ a má stejný definiční obor. $3 \Rightarrow 4:$ Máme soubor množin $<a_{i}, i \in x>$. Vytvoříme relaci $s = \{(i,y), i \in x \land y \in a_{i}\}$. Funkce $f \subseteq s: Dom(f) = Dom(s) = x$ je prvkem $X_{i \in x}a_{i}$. $4 \Rightarrow 1:$ $x$ množina. BÚNO: $x \neq \emptyset, \emptyset \in X$. $ID \upharpoonright x$ určuje soubor $<y;y \in x>$. Každý prvek $X_{y \in x}y$ je selektor na $x$.
\end{proof}

\begin{lemma}
	Sjednocení spočetného souboru spočetných množin je spočetné. (Popřípadě je všude místo spočetné nejvýše spočetné.)
\end{lemma}

\begin{proof}
	Soubor $<B_{j};j \in J>$. BŮNO: $I = \omega$. Najděme prosté zobrazení $\bigcup_{j \in \omega} B_{j}$ do $\omega \times \omega$. Uvažujme soubor $<E_{j}; j \in \omega>$ kde $E_{j}$ je množina všech prostých zobrazení $B_{j}$ do $\omega$. Podle lemma 4) je $X_{j \in \omega}E_{j}$ neprázdný, tedy existuje soubor $<f_{j}; j \in \omega>$, kde $f_{j} \in F_{j}$. Definujme $h; \bigcup_{j \in \omega}B_{j} \to \omega\times\omega$ jako $h(x) = (j, f_{j}(x))$. Kde $j$ je nejmenší prvek $\omega$ pro který $x \in B_{j}$.
\end{proof}

\begin{pozn}
	Bez AC je bezesporné ZF a to, že "$\mathbb{R}$ jsou spočetným sjednocením spočetných množin".
\end{pozn}

\section{Princip maximality (PM)}

\begin{itemize}
	\item AC $\leftrightarrow$ PM
	\item Je-li $A$ množina uspořádaná relací $\leq$ tak, že každý řetězec má horní mez.
	\item Pak pro každé $a \in A$ existuje maximální prvek $b \in A$ takový, že $a \leq b$.
\end{itemize}

\begin{definice}
	$B \subseteq A$ je \textbf{řetězec} pokud $B$ je lineárně uspořádaná $\leq$.
\end{definice}

\begin{pozn}
	V aplikacích často pro $(A, \subseteq); A \subseteq \mathcal{P}(x)$ stačí ověřit, že $\bigcup B \in A$.
\end{pozn}

\textit{Cvičení: Ukažte pomocí PM: Je-li $(A, \leq)$ uspořádaná množina, pak pro každý řetězec $B \subseteq A$ existuje maximální řetězec $C$ splňující $B \subseteq C \subseteq A$.}

\subsection{Princip maximality II (PMS)}

Je-li $(A, \leq)$ uspořádaná množina, kde každý řetězec má suprémum, pak pro každé $a \in A$ existuje $b \in A$ maximální prvek splňující $a \leq b$.

\textit{Cvičení: Dokažte: PM$\leftrightarrow$PMS.}

\section{Princip trichotomie $\preceq$ (PT)}

Pro každé dvě množiny $x,y$ platí $x \preceq y$ nebo $y \preceq x$.

\begin{lemma}
	PM $\rightarrow$ PT.
\end{lemma}

\begin{proof}
	Definuji množinu $D = \{f, f \text{ prosté zobrazení } \land Dom(f) \subseteq x \land Rng(f) \subseteq y \}$. $(D, \subseteq)$ splňuje předpoklady PM. Tedy má maximální prvek $g$. Kdyby $x \setminus Dom(f) \neq \emptyset$ a $y \setminus Rng(f) \neq \emptyset$, pak lze $g$ rozšířit o novou dvojici $(u,v)$, spor s maximalitou $g$. Pokud $Dom(f) = x$, pak $x \preceq y$. Pokud $Rng(f) = y$, pak $g^{-1}$ je prosté zobrazení $y$ do $x$, tedy $y \preceq y$.
\end{proof}

\textit{Cvičení: Sjednocení řetězce prostých zobrazení je prosté zobrazení.}

\section{Princip dobrého uspořádání (VVO)}

\begin{itemize}
	\item Každou množinu lze dobře uspořádat.
	\item Známo jako Zermelova věta.
	\item AC $\leftrightarrow$ VVO
\end{itemize}

\begin{lemma}
	VVO $\rightarrow$ AC
\end{lemma}

\begin{proof}
	$x \neq \emptyset, \emptyset \notin x$ podle VVO máme dobré uspořádání na $\bigcup x$. Každý $y \in x$ je neprázdná podmnožina $\bigcup x$, tedy má nejmenší prvek $\min_{\leq}y$. Definujeme $f: x \to \bigcup x$ jako $f(y) = \min_{\leq}(y)$. Tato $f$ je selektorem na množině $x$.
\end{proof}

\textit{Cvičení: PM $\rightarrow$ VVO}