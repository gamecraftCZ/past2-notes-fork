\chapter{PQ trees}

\begin{notation}
	A set $\{1, \dots, n\}$ will be denoted as $[n]$. Then \textbf{permutation} of $[n]$ is a sequence $\pi = \pi_1, \pi_2, \dots, \pi_n$ in which each $i \in [n]$ appears exactly once. \textbf{Interval} in a permutation $\pi$ of $[n]$ is a set $S = \{\pi_i, \pi_{i+1}, \dots, \pi_j\}$ for some $1 \leq i \leq j \leq n$. \textbf{Cyclic shift} of $\pi_1, \pi_2, \dots, \pi_n$ is a permutation of the form $\pi_i, \pi_{i+1}, \dots, \pi_n, \pi_1, \pi_2, \dots, \pi_{n-1}$ for some $i \in [n]$. Lastly \textbf{cyclic interval} of $\pi$ is interval in a cyclic shift of $\pi$.
\end{notation}

\begin{example}
	Lets see an example for a permutation $\pi = 31524$ of $[5]$. Then $\{1,2,5\}$ is an interval and $\{2,3,5\}$ is \textbf{not} an interval. Furthermore its cyclic shift can be $52431$. Where one cyclic interval of the original permutation can be $\{2,3,4\}$.
\end{example}

Now we will introduce two problems.

\begin{itemize}[]
	\item \textit{Consecutivity}
	
	\begin{itemize}[]
		\item \textbf{Input:} $n \in \N$, sets $S_1, S_2, \dots, S_k \subseteq [n]$.
		\item \textbf{Question:} is there a permutation $\pi$ of $[n]$ in which $S_1, S_2, \dots, S_k$ are all intervals?
	\end{itemize}
	
	\item \textit{Cyclic consecutivity}
	
	\begin{itemize}[]
		\item \textbf{Input:} $n \in \N$, sets $S_1, S_2, \dots, S_k \subseteq [n]$.
		\item \textbf{Question:} is there a permutation $\pi$ of $[n]$ in which $S_1, S_2, \dots, S_k$ are all cyclic intervals?
	\end{itemize}
\end{itemize}

\begin{lemma}
	Consecutivity can be reduced to cyclic consecutivity.
\end{lemma}

\begin{proof}
	We are given $n \in \N$, and sets $S_1, S_2, \dots, S_k \subseteq [n]$. Now see the following: $\exists \pi$ of $[n]$ in which $S_1, \dots, S_k$ are intervals $\iff$ $\exists \pi^+$ of $[n+1]$ in which $S_1, \dots, S_k$ are cyclic intervals. For one way see that if we have $\pi$ which has intervals $S_1, \dots, S_k$ and create $(\pi, n+1) = \pi^+$ permutation which has cyclic intervals. For the other way lets have $\pi^+$ of $[n+1]$ that has cyclic intervals in $S_1, \dots, S_k$, choose the cyclic shift of $\pi^+$ with $n+1$ at the end. Then $\pi^+ = (\pi, n+1)$ where $\pi$ will be the permutation of $[n]$ containing $S_1, \dots, S_k$ intervals.
\end{proof}

\textbf{Cyclic permutation} is determined by a permutation $\pi$ and it is the set of all cyclic shifts of $\pi$. (Unformally we may draw a circle with the elements on the boundary and all cyclic permutations are when going clockwise around the circle.) We also denote $\cyc{n}{S_1, \dots, S_k}$ as the set of cyclic permutations of $[n]$ in which all the sets $S_1, \dots, S_k$ are cyclic intervals.

\begin{defn}
	A \textbf{PQ-tree} of order $n$ is an (unrooted, undirected0 tree with $n$ leaves labeled $1,2, \dots, n$ and two types of interval nodes: P-nodes \faCircle, Q-nodes \faCircle[regular], every internal node has a prescribed cyclic permutation of its neighbours.
\end{defn}

\begin{defn}
	$\pi_T$ for PQ-tree $T$ is the cyclic permutation of $[n]$ induced by the clockwise order of the leaves of $T$.
\end{defn}

\begin{defn}
	Two PQ-trees $T,T'$ are equivalent if $T'$ can be obtained from $T$ by a sequence of the following operations:
	
	\begin{enumerate}
		\item change the cyclic order of neighbours of P-node arbitrarily, and
		\item reverse the order of neighbours of Q-node.
	\end{enumerate}
\end{defn}

\begin{defn}
	The set of cyclic permutations represented by $T$, denoted by $R_T$ is $\{\pi_{T'} | T'$ equivalent to $T\}$.
\end{defn}

\begin{figure}[!ht]\centering
	\begin{subfigure}{.45\textwidth}
		\begin{tikzpicture}
		\end{tikzpicture}
		\caption{PQ-tree $T$ with permutation $\pi_T = 14378625$.}
	\end{subfigure}
	\begin{subfigure}{.45\textwidth}
		\begin{tikzpicture}
		\end{tikzpicture}
		\caption{Equivalent PQ-tree $T'$ with $\pi_{T'} = 15237864$.}
	\end{subfigure}
	\caption{Example of PQ-tree, its permutation and equivalent tree.}
\end{figure}

\begin{thm}
	For any $n \in \N$, sets $S_1, \dots, S_k \subseteq [n]$, we can, in time $O(n + \sum_{i=1}^k |S_i|)$ determine whether $\cyc{n}{S_1, \dots, S_k}$ is non-empty, and if it is, construct a PQ-tree $T$ such that $R_T = \cyc{n}{S_1, \dots, S_k}$.
\end{thm}

\begin{proof}[Construction of the PQ-tree]
	We won't show the whole proof, but only the construction of $T$ which will be shown by an induction on $k$. For $k=0$ we create one internal P-node which has all $[n]$ leaves ordered as $1,2, \dots, n$ clockwise. Suppose for $k > 0$ we have constructed PQ-tree $T_{k-1}$ with $R_{T_{k-1}} = \cyc{n}{S_1, \dots, S_{k-1}}$. The goal is to fing $T$ with $R_T = \cyc{n}{S_1, \dots, S_k}$.
	
	Let $e$ be an edge in $T_{k-1}$, then $T_{k-1}-e$ has two components "substrees determined by $e$". Then subtrees is \textbf{full} if each of its leaves is in $S_k$, \textbf{empty} if none of its leaves are from $S_k$ and \textbf{mixed} otherwise. Then an edge $e$ of $T_{k-1}$ is \textbf{mixed} if both subtrees of $T_{k-1}-e$ are mixed.
	
	\begin{observ}
		Mixed edges form a connected subgraph of $T_{k-1}$.
	\end{observ}
	
	\begin{proof}
		If it is not true then there is a path connecting two mixed edges, but the edges on the path has to be also mixed.
	\end{proof}
	
	\begin{observ}
		If there is a certex of $T_{k-1}$ incident to three or more mixed edges, then $\cyc{n}{S_1, \dots, S_k} = \emptyset$.
	\end{observ}
	
	Suppose mixed edges form a path $P$. Now we will show steps to create new PQ-tree.
	
	\begin{enumerate}
		\item Replace $T_{k-1}$ by an equivalent tree in which around every vertex of $P$ the edges towards full subtrees are above $P$, the edges towards empty subtrees are below. If this is not possible, then $\cyc{n}{S_1, \dots, S_k} = \emptyset$.
		\item Replace every node $v_i$ of $P$ by two nodes $v_i^+$ connected to the full subtrees only and $v_i^-$ connected to the empty subtrees only.
		\item Insert a new Q-node adjacent to $v_1^+, v_2^+, \dots, v_m^+, v_m^-, v_{m-1}^-, \dots, v_1^-$ in this order ($m$ is for the number of nodes of $P$), call the new node $w$.
		\item If $v_i^+$ or $v_i^-$ is a Q-node, then contract the edge $w v_i^+$ (or $w v_i^-$).
		\item If there is a node of degree 2, suppress it, if $v_i^+$ or $v_i^-$ has degree 1, delete it. (Where suppressing is swapping the path by a single edge.) 
	\end{enumerate}
	
	The correctness of this process involves more checking if all representations are still preserved and that all present in the new one was already there.
	
	For time complexity one must use clever data structure and use amortization arguments to obtain such result.
\end{proof}

\section{Recognition of INT in linear time}

Recall that we have already shown that INT = Chordal $\cap$ co-Co nad also $G \in$ INT $\iff$ the maximal cluques of $G$ can be arranged into a sequence $Q_1, Q_2, \dots, Q_l$ so that for every vertex $v$, the cliques containing $v$ form an interval (in the permutation of maximal cliques).

Thus we can see that this is Consecutivity problem and we only need to find all maximal cliques.

\begin{cons}
	$G \in$ INT can be tested in linear time $O(|V| + |E|)$.
\end{cons}