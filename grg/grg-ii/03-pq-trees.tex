\chapter{PQ trees}

\begin{notation}
	A set $\{1, \dots, n\}$ will be denoted as $[n]$. Then \textbf{permutation} of $[n]$ is a sequence $\pi = \pi_1, \pi_2, \dots, \pi_n$ in which each $i \in [n]$ appears exactly once. \textbf{Interval} in a permutation $\pi$ of $[n]$ is a set $S = \{\pi_i, \pi_{i+1}, \dots, \pi_j\}$ for some $1 \leq i \leq j \leq n$. \textbf{Cyclic shift} of $\pi_1, \pi_2, \dots, \pi_n$ is a permutation of the form $\pi_i, \pi_{i+1}, \dots, \pi_n, \pi_1, \pi_2, \dots, \pi_{n-1}$ for some $i \in [n]$. Lastly \textbf{cyclic interval} of $\pi$ is interval in a cyclic shift of $\pi$.
\end{notation}

\begin{example}
	Lets see an example for a permutation $\pi = 31524$ of $[5]$. Then $\{1,2,5\}$ is an interval and $\{2,3,5\}$ is \textbf{not} an interval. Furthermore its cyclic shift can be $52431$. Where one cyclic interval of the original permutation can be $\{2,3,4\}$.
\end{example}

Now we will introduce two problems.

\begin{mybox}
	
\end{mybox}

\begin{itemize}[]
	\item \textit{Consecutivity}
	
	\begin{itemize}[]
		\item \textbf{Input:} $n \in \N$, sets $S_1, S_2, \dots, S_k \subseteq [n]$.
		\item \textbf{Question:} is there a permutation $\pi$ of $[n]$ in which $S_1, S_2, \dots, S_k$ are all intervals?
	\end{itemize}
	
	\item \textit{Cyclic consecutivity}
	
	\begin{itemize}[]
		\item \textbf{Input:} $n \in \N$, sets $S_1, S_2, \dots, S_k \subseteq [n]$.
		\item \textbf{Question:} is there a permutation $\pi$ of $[n]$ in which $S_1, S_2, \dots, S_k$ are all cyclic intervals?
	\end{itemize}
\end{itemize}

\begin{lemma}
	Consecutivity can be reduced to cyclic consecutivity.
\end{lemma}

\begin{proof}
	We are given $n \in \N$, and sets $S_1, S_2, \dots, S_k \subseteq [n]$. Now see the following: $\exists \pi$ of $[n]$ in which $S_1, \dots, S_k$ are intervals $\iff$ $\exists \pi^+$ of $[n+1]$ in which $S_1, \dots, S_k$ are cyclic intervals. For one way see that if we have $\pi$ which has intervals $S_1, \dots, S_k$ and create $(\pi, n+1) = \pi^+$ permutation which has cyclic intervals. For the other way lets have $\pi^+$ of $[n+1]$ that has cyclic intervals in $S_1, \dots, S_k$, choose the cyclic shift of $\pi^+$ with $n+1$ at the end. Then $\pi^+ = (\pi, n+1)$ where $\pi$ will be the permutation of $[n]$ containing $S_1, \dots, S_k$ intervals.
\end{proof}

\textbf{Cyclic permutation} is determined by a permutation $\pi$ and it is the set of all cyclic shifts of $\pi$. (Unformally we may draw a circle with the elements on the boundary and all cyclic permutations are when going clockwise around the circle.) We also denote $\cyc{n}{S_1, \dots, S_k}$ as the set of cyclic permutations of $[n]$ in which all the sets $S_1, \dots, S_k$ are cyclic intervals.

\begin{defn}
	A \textbf{PQ-tree} of order $n$ is an (unrooted, undirected) tree with $n$ leaves labeled $1,2, \dots, n$ and two types of interval nodes: P-nodes \faCircle, Q-nodes \faCircle[regular], every internal node has a prescribed cyclic permutation of its neighbours.
\end{defn}

\begin{defn}
	$\pi_T$ for PQ-tree $T$ is the cyclic permutation of $[n]$ induced by the clockwise order of the leaves of $T$.
\end{defn}

\begin{defn}
	Two PQ-trees $T,T'$ are equivalent if $T'$ can be obtained from $T$ by a sequence of the following operations:
	
	\begin{enumerate}
		\item change the cyclic order of neighbours of P-node arbitrarily, and
		\item reverse the order of neighbours of Q-node.
	\end{enumerate}
\end{defn}

\begin{defn}
	The set of cyclic permutations represented by $T$, denoted by $R_T$ is $\{\pi_{T'} | T'$ equivalent to $T\}$.
\end{defn}

\begin{figure}[!ht]\centering
	\begin{subfigure}{.45\textwidth}\centering
		\begin{tikzpicture}[node distance = 40, p/.style = {draw, circle, fill}, q/.style = {draw, circle}, thick]
			\node[p] (c) {};
			\node[q, left of = 0] (l) {};
			\node[q, right of = 0] (r) {};
			\node[left of = l] (5) {5};
			\node[above of = l] (1) {1};
			\node[below of = l] (2) {2};
			\node[above left of = c] (4) {4};
			\node[above right of = c] (3) {3};
			\node[right of = r] (8) {8};
			\node[above of = r] (7) {7};
			\node[below of = r] (6) {6};
			\draw (c) edge (4);
			\draw (c) edge (3);
			\draw (l) edge (5);
			\draw (l) edge (1);
			\draw (l) edge (2);
			\draw (r) edge (8);
			\draw (r) edge (7);
			\draw (r) edge (6);
			\draw (r) edge (c);
			\draw (l) edge (c);
		\end{tikzpicture}
		\caption{PQ-tree $T$ with permutation $\pi_T = 14378625$.}
	\end{subfigure}
	\begin{subfigure}{.45\textwidth}\centering
		\begin{tikzpicture}[node distance = 40, p/.style = {draw, circle, fill}, q/.style = {draw, circle}, thick]
			\node[p] (c) {};
			\node[q, left of = 0] (l) {};
			\node[q, right of = 0] (r) {};
			\node[left of = l] (5) {5};
			\node[above of = l] (2) {2};
			\node[below of = l] (1) {1};
			\node[above of = c] (3) {3};
			\node[below of = c] (4) {4};
			\node[right of = r] (8) {8};
			\node[above of = r] (7) {7};
			\node[below of = r] (6) {6};
			\draw (c) edge (4);
			\draw (c) edge (3);
			\draw (l) edge (5);
			\draw (l) edge (1);
			\draw (l) edge (2);
			\draw (r) edge (8);
			\draw (r) edge (7);
			\draw (r) edge (6);
			\draw (r) edge (c);
			\draw (l) edge (c);
		\end{tikzpicture}
		\caption{Equivalent PQ-tree $T'$ with $\pi_{T'} = 15237864$.}
	\end{subfigure}
	\caption{Example of PQ-tree, its permutation and equivalent tree.}
\end{figure}

\begin{thm}
	For any $n \in \N$, sets $S_1, \dots, S_k \subseteq [n]$, we can, in time $O(n + \sum_{i=1}^k |S_i|)$ determine whether $\cyc{n}{S_1, \dots, S_k}$ is non-empty, and if it is, construct a PQ-tree $T$ such that $R_T = \cyc{n}{S_1, \dots, S_k}$.
\end{thm}

\begin{proof}[Construction of the PQ-tree]
	We won't show the whole proof, but only the construction of $T$ which will be shown by an induction on $k$. For $k=0$ we create one internal P-node which has all $[n]$ leaves ordered as $1,2, \dots, n$ clockwise. Suppose for $k > 0$ we have constructed PQ-tree $T_{k-1}$ with $R_{T_{k-1}} = \cyc{n}{S_1, \dots, S_{k-1}}$. The goal is to find $T$ with $R_T = \cyc{n}{S_1, \dots, S_k}$.
	
	Let $e$ be an edge in $T_{k-1}$, then $T_{k-1}-e$ has two components "substrees determined by $e$". Then subtrees is \textbf{full} if each of its leaves is in $S_k$, \textbf{empty} if none of its leaves are from $S_k$ and \textbf{mixed} otherwise. Then an edge $e$ of $T_{k-1}$ is \textbf{mixed} if both subtrees of $T_{k-1}-e$ are mixed.
	
	\begin{observ}
		Mixed edges form a connected subgraph of $T_{k-1}$.
	\end{observ}
	
	\begin{proof}
		If it is not true then there is a path connecting two mixed edges, but the edges on the path has to be also mixed.
	\end{proof}
	
	\begin{observ}
		If there is a vertex of $T_{k-1}$ incident to three or more mixed edges, then $\cyc{n}{S_1, \dots, S_k} = \emptyset$.
	\end{observ}
	
	Suppose mixed edges form a path $P$. Now we will show steps to create new PQ-tree.
	
	\begin{enumerate}
		\item Replace $T_{k-1}$ by an equivalent tree in which around every vertex of $P$ the edges towards full subtrees are above $P$, the edges towards empty subtrees are below. If this is not possible, then $\cyc{n}{S_1, \dots, S_k} = \emptyset$.
		\item Replace every node $v_i$ of $P$ by two nodes $v_i^+$ connected to the full subtrees only and $v_i^-$ connected to the empty subtrees only.
		\item Insert a new Q-node adjacent to $v_1^+, v_2^+, \dots, v_m^+, v_m^-, v_{m-1}^-, \dots, v_1^-$ in this order ($m$ is for the number of nodes of $P$), call the new node $w$.
		\item If $v_i^+$ or $v_i^-$ is a Q-node, then contract the edge $w v_i^+$ (or $w v_i^-$). But keep the order.
		\item If there is a node of degree 2, suppress it, if $v_i^+$ or $v_i^-$ has degree 1, delete it. (Where suppressing is swapping the 2-edge path by a single edge.) 
	\end{enumerate}
	
	The correctness of this process involves more checking if all representations are still preserved and that all present in the new one was already there.
	
	For time complexity one must use clever data structure and use amortization arguments to obtain such result.
\end{proof}

\section{Applications}

\subsection{Recognition of INT in linear time}

Recall that we have already shown that INT = Chordal $\cap$ co-Co and also $G \in$ INT $\iff$ the maximal cliques of $G$ can be arranged into a sequence $Q_1, Q_2, \dots, Q_l$ so that for every vertex $v$, the cliques containing $v$ form an interval (in the permutation of maximal cliques).

\begin{algorithm}[!ht]
	\caption{Idea of the algorithm}
	\begin{algorithmic}[1]
			\Require Graph $G = (V,E)$.
			\Ensure Is $G \in$ INT?
			\State Create a set $\{Q_{i}; i \text{ is maximal clique of } G\} = \{Q_{1}, Q_{2}, \dots, Q_{l}\}$.
			\State $\forall v \in V(G):$ create a set $S_{v} = \{i; v \in Q_{i}\}$.
			\State Solve consecutivity for $\{S_{v}; v \in V\}$ in $[l]$.
	\end{algorithmic}
\end{algorithm}

Note that from PQ-trees this can be solved in linear time $O(|V| + \sum_{v \in V} |S_{v}|)$. Now we will dig deeper into the details of each step.

\begin{enumerate}
	\item We know that $G$ is chordal $\iff$ $G$ has PES \PES. Also we have shown how to compute PES in linear time. Therefore create PES $v_{1}, \dots, v_{n}$. If it does not exists, return that $G$ is not INT. From now on assume it exists. Create $\forall i = 1, \dots, n$ cliques $Q_{i} := \{v_{i}\} \cup \{\text{left neighbours of } v_{i}\}$. Firstly observe that all maximal cliques are among these cliques (simply because all cliques have their rightmost vertex in PES, therefore its $Q_{i}$ since it is maximal). Next we will discard $Q_{i}$  if $\exists j : Q_{i} \subsetneq Q_{j}$. Note that this happens if $j > i$ and $v_{i}v_{j} \in E$ and any more to the left neighbour of $v_{j}$ is also a neighbour of $v_{i}$ so we only need to compute sizes $|Q_{i}| = S_{i}$ and denote $k$ as the number of left neighbours starting by $v_{i}$.
		\begin{itemize}
			\item If $k = S_{i}$ then $Q_{i} \subseteq Q_{j}$.
			\item If $k < S_{i}$ then $Q_{i} \nsubseteq Q_{j}$.
			\item Also $k > S_{i}$ cannot happen.
		\end{itemize}

	\item We will also use PES.

	$$
	S_{v} = (\{i\} \cup \{j | v_{j} \text{ is a right neighbour of } v_{i}\}) \cap \{k; Q_{k} \text{ is maximal}\}
	$$

	Which also implies that $|S_{v}| \leq$ right degree of $v$ + 1. Therefore $\sum_{v \in V} |S_{v}| \leq |E| + |V|$. So all together it is $O(|E| + |V|)$.
\end{enumerate}

\subsection{Planarity testing}

\begin{itemize}[]
	\item \textbf{Input:} Graph $G = (V,E)$.
	\item \textbf{Output:} Planar embedding of $G$ or a $K_{5}$ or $K_{3,3}$ minor of $G$.
\end{itemize}

For our usecase we will denote "planar embedding" as the rotation scheme, which is for any $v \in V$, cyclic order of edges incident to $v$ in a planar drawing.

\begin{defn}
	Fragment of a graph $G = (V,E)$ induced by a set $X \subseteq V$ contains

	\begin{enumerate}
			\item The subgraph of $G$ induced by $X$.
			\item For any edge $e = \{u,v\} \in E$ s.t. $u \in X$ and $v \notin X$ create new vertex $s_{e}$ ("stump of $e$") and an edge $\{u, s_{e}\}$.
	\end{enumerate}
\end{defn}

We say that fragment induced by $X$ in $G$ is \textbf{good} if both $X$ and $V \setminus X$ induces a connected subgraph of $G$. Also a planar embedding of a fragment is \textbf{good} if all the stumps are embedded on the boundary of a single face ("outer face").

\begin{observ}
	If $G$ is planar, then also any good fragment has a good embedding.
\end{observ}

From now on assume that $G$ is 2-connected.

\begin{fact}
	Any good embedding of a good fragment induces a cyclic order of the stumps. Moreover for any good fragment with at least one good embedding, there is a PQ-tree whose leaves correspond to the stumps and which represents precisely the cyclic orders of stupms in the good embeddings of the fragments.
\end{fact}

\begin{proof}[Sketch of proof]
	Firstly generally for a graph $H = (V_{H}, E_{H})$ equivalence on $E_{H}$ $e,f \in E_{h}: e \sim f$ if either $e = f$ or $e, f$ belongs to common cycle. Then classes of $\sim$ are 1. bridges and 2. biconnected components.

	Now the constructions is roughly that we will replace all biconected components by Q nodes. Then cutvertices to P nodes, stumps to leaves, bridge to edge and incidence between cutvertex and biconnected component will also form an edge.

	Every single operation with PQ-trees can be also done to change the embedding of the graph. It is also good to show the other way and that if PQ-tree wouldn't be able to keep pace with the embedding then there is a $K_{5}$ or $K_{3,3}$ minor.
\end{proof}

\begin{algorithm}[!ht]
	\caption{Planarity testing}
	\begin{algorithmic}[1]
			\Require Graph $G = (V,E)$.
			\Ensure Planar embedding of $G$ or a $K_{5}$ or $K_{3,3}$ minor of $G$.
			\State Create $T$ as a DFS tree of $G$.
			\State Number its vertices in postorder (in order left-right-root).
			\State $T_{i}:=$ subtree rooted in vertex numbered $i$ and $F_{i} :=$ good fragment induced by vertices of $T_{i}$.
			\State Proceed bottom to top and construct PQ-trees for $F_{1}, F_{2}, \dots, F_{n}$.
 	\end{algorithmic}
\end{algorithm}

The exact procedure of building fragments is that in the leaf we just create a single P node and all its stumps. Then by induction let all children have their PQ-trees. Lets take one of these PQ-trees and it has stumps which has to be connected to the current vertex and other vertices. The former vertices has to be arranged consecutive, if it is not possible then end. That is similar to the PQ-trees where we tried to put all free subtrees to the above part and others to the below, now it is pretty much the same. Also in this step we remember the orientation of the stumps that will be connected to the root. This will merge to one big Q node and connect to the root which will be new P node. Also we will create such Q nodes for all children of the current root.
