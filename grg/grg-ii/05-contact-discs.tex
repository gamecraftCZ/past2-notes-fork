\chapter{Contact graphs of discs}

Firstly we have to define what a contact graph of discs is. For $v \in V$ associate disc, if $\{xy\} \in E$ then discs touch in one point.

\begin{observ}
	$G$ has a contact disc representation $\Rightarrow G$ is planar.
\end{observ}

\begin{thm}[Koebe]
	$G$ is planar $\iff$ $G$ has a contact disc representation.
	\label{koebe}
\end{thm}

In this section we will be proving this theorem. Note that from the first observation we have already proven "$\Leftarrow$" part.

\begin{observ}
	It is enough to prove "$\Rightarrow$" when $G$ is maximal planar, i.e. a triangulation.
\end{observ}

\begin{proof}
	Because every planar graph $G$ is an \textit{induced} subgraph of a triangulation $G^+$. Note that subgraph is easy to see. And when speaking about induced graph we proceed similarly. That is introduce new edges to have triangulation and then introduce a vertex in the middle of every new edge. Lastly add other edges to obtain triangulation.
\end{proof}

\begin{proof}[Proof of Koebe theorem \ref{koebe}]
	We will set $n := |V| \geq 4$, fix a plane drawing of $G$. We will also assume that $V = \{1,2, \dots, n\}$ and vertices on the outer face are $1,2,3$. Goal is to have $(r_1, r_2, \dots, r_n) \in \R^n, r_i > 0$ s.t. there exists disc representation of $G$ where $i$ is represented by a disc of radius $r_i$, the disc touching a disc representing $i$ have the order given by the drawing of $G$.
	
	Define
	
	$$
	\mathcal{R} = \left\{(r_1, r_2, \dots, r_n) \in \R^n; \underbrace{\forall i : r_i > 0}_{\text{interior}}; \underbrace{\sum_{i = 0}^n r_i = 1}_{\text{hyperplane}}\right\}
	$$
	
	\noindent which is an $n-1$ simplex. Suppose $i,j,k$ forms a face of $G$, then: $\alpha_i(\overrightarrow{r},f)$ for $f = \{i,j,k\}$ face, $\overrightarrow{r} \in \mathcal{R}$ is the angle at $i$ in the triangle $i,j,k$. Total angle at $i$:
	
	$$
	t_i(\overrightarrow{r}) := \sum_{f \text{ face } \ni i} \alpha_i(\overrightarrow{r}, f).
	$$
	
	\noindent Now we have to show two main steps which will lead to the full proof.
	
	\begin{enumerate}[I)]
		\item Show that $\exists \overrightarrow{r^\ast} \in \mathcal{R} : (t_1(\overrightarrow{r^\ast}), t_2(\overrightarrow{r^\ast}), \dots, t_n(\overrightarrow{r^\ast})) = (\frac{2\pi}{3}, \frac{2\pi}{3}, \frac{2\pi}{3}, 2\pi, \dots, 2\pi)$. \label{koebe-first}
		\item Show that there is a contact disc representation $t^\ast$ with radii $\overrightarrow{r^\ast}$. \label{koebe-second}
	\end{enumerate}

	\noindent Ad \ref{koebe-first}: Firstly define $T : \mathcal{R} \to \R^n$, defined as $T(\overrightarrow{r}) = (t_1 (\overrightarrow{r}), t_2 (\overrightarrow{r}), \dots, t_n (\overrightarrow{r}))$. Now we have the following substeps.
	
	\begin{enumerate}[a)]
		\item $T$ is injective. \label{koebe-first-a}
		\item There is a set $A \subseteq \R^n$ s.t. $\text{Im}(T) \subseteq A$. \label{koebe-first-b}
		\item $t^\ast \in A$ \label{koebe-first-c}
		\item $T$ is onto $A$, i.e. $\text{Im}(T) = A$. \label{koebe-first-d}
	\end{enumerate}

	\begin{observ}
		$\alpha_i(\overrightarrow{r}, \{i,j,k\})$ is decreasing in $r_i$, increasing in $r_j, r_k$.
	\end{observ}

	\begin{notation}
		For $I \subseteq V: F(I)$ is the set of faces of $G$ incident to at least one vertex in $I$.
	\end{notation}

	\begin{lemma}
		$T$ is injective.
	\end{lemma}

	\begin{proof}[Proof of lemma (also step \ref{koebe-first-a})]
		Choose $\overrightarrow{r} \in \mathcal{R}, \overrightarrow{r'} \in \mathcal{R}, \overrightarrow{r} \neq \overrightarrow{r'}, I = \{i \in V : r_i < r_i^+\}$, where $I \neq \emptyset$ and also $I \neq V$.
	\end{proof}
 \end{proof}