\chapter{Recognition of Chordal graphs}

It is not surprising that chordal graphs can be recognized in polynomial time. Just search for a simplicial vertex, delete it if you find one and proceed recursively, or quit and say that the input graph
is not chordal if it has no simplicial vertex. Checking a vertex for being simplicial requires checking at most $n^2$ pairs of vertices if they are adjacent, hence a simplicial vertex can be found in $O(n^3)$ steps. Thus chordality can be checked in time $O(n^4)$. However, even if the graph is dense, i.e., if it has $m = \Omega(n^2)$, this naive algorithm still takes $\Omega(m^2)$ time. With more care, one can test chordality in time linear in the number of edges by the following algorithm.

\begin{algorithm}[!ht]
	\caption{LexBFS}
	\begin{algorithmic}[1]
		\Require A graph $G = (V,E)$ with $n$ vertices.
		\State $T := \emptyset, w(x) := \emptyset$ $\forall x \in V$
		\For{$i:= n \dots 1$}
			\State let $u \in V \setminus T$ be a vertex with lexicographicly maximum $w(u)$
			\State set $\sigma(i) := u$
			\State $T := T \cup \{u\}$
			\For{$x \in N_G (u) \cap (V \setminus T)$}
				\State $w(x) := w(x) \cup \{i\}$
			\EndFor
		\EndFor
	\end{algorithmic}
\end{algorithm}

\begin{thm}
	If $G = (V, E)$ is chordal, then $\sigma(n), \sigma(n - 1), . . . , \sigma(1)$ is a PES \faDog.
\end{thm}

\begin{proof}
	Suppose $G$ is chordal but there exist indices $i < j < h$ such that $\sigma(i)\sigma(j) \in E, \sigma(i)\sigma(h) \in E$ while $\sigma(j)\sigma(h) \notin E$. Let $i_0 < i_1 < i_2$ be such a triple and such that $i_2$ is largest possible.
	
	Let $u_i = \sigma(i), i = 1, 2, \dots , n$ be the ordering output by the Algorithm. Consider the step of the LexBFS algorithm when $i_1$ was processed. At this moment, $i_2 \in w(u_{i_0})$ while $i_2 \notin w(u_{i_1})$. Thus $u_{i_0}$ and $u_{i_1}$ do not have the same weights, and since $u_{i_1}$ was chosen for $\sigma(i_1)$, we must have $w(u_{i_0}) <_{lex} w(u_{i_1})$. Hence there must be $i_3 > i_2$ such that $u_{i_1} u_{i_3} \in E$ and $u_{i_0} u_{i_3} \notin E$. Let us choose $i_3$ as the largest index with these properties. An edge between $u_{i_2}$ and $u_{i_3}$ would imply that $G[\{u_0 , u_1 , u_2 ,u_3 \}]$ would be isomorphic to $C_4$, which would contradict the assumption that $G$ is chordal, and thus we conclude that $u_{i_2} u_{i_3} \notin E$.
	
	The same argument can now be used for $i_2$ and $i_3$ and further on:
	
	\begin{claim}
		For every $k > 3$, there exists a sequence $i_0 < i_1 < \dots < i_k$ such that $\{u_{i_0} u_{u_1}\} \cup \{u_{i_j} u_{i_{j+2}} : j = 0, 1, 2, \dots, k - 2\}$ are the only edges of $G$ among these vertices.
	\end{claim}
	
	\begin{proof}[Proof of claim]
		We proceed by induction on $k$. We have just seen that such a sequence exists for $k = 4$. Since the sequence is constructed recursively, we will further assume that in each step we have chosen $i_k$ largest possible.
		
		For the induction step $k \to k + 1$, we observe that at the moment when $i_{k-1}$ was processed, $i_k$ was in $w(u_{k-2})$ but not in $w(u_{k-1})$. Since $u_{i_{k-1}}$ was chosen for $\sigma(i_{k-1})$, it must have been $w(u_{i_{k-1}}) >_{lex} w(u_{i_{k-2}})$, and so there must be an $i_{k+1} > i_k$ such that $u_{i_{k-1}} u_{i_{k+1}} \in E$ while $u_{i_{k-2}} u_{i_{k+1}} \notin E$ (recall that we choose $i_{k+1}$ maximum possible with this property). Suppose there is a $j$, $0 \leq j \leq k - 3$ such that $u_{i_j} u_{i_{k+1}} \in E$, and consider the largest possible such $j$. If $j = k - 3$, then both $u_{i_{k+1}}$ and $u_{i_{k-1}}$ are adjacent to $u_{i_{k-3}}$ and non-adjacent to $u_{i_{k-2}}$ and this contradicts the presumed choice of
		$i_{k-1} < i_{k+1}$ as the largest possible index of this property. If $j < k - 3$, then $i_j < i_{j+2} < i_{k+1}$ would be	a triple violating the PES condition (since $u_{i_j} u_{i_{j+2}}, u_{i_j} u_{i_{k+1}} \in E$ and $u_{i_{j+2}} u_{i_{k+1}} \notin E$) with $i_{k+1} > i_2$ contradicting the choice of $i_0 < i_1 < i_2$. Thus $u_{i_j} u_{i_{k+1}} \notin E$ for $j = 0, 1, \dots, k - 2$. If $u_{i_k} u_{i_{k+1}}$ were an edge, $G[\{u_{i_h} : h = 0, 1, \dots, k + 1\}]$ would be isomorphic to $C_{k+2}$, which is impossible as $G$ is assumed to be chordal. This concludes the proof of the Claim.
	\end{proof}
	
	Since $G$ is finite, the sequence of the properties guaranteed by the Claim cannot exist, e.g., for $k > n$. So no triple of indices $i < j < h$ violating the PES condition may exist.
\end{proof}

With a suitable data structure, the LexBFS algorithm can be implemented to run in time linear in the number $m$ of edges of the input graph (the point is that this algorithm processes every edge of the graph only once).

However, this is only one half of testing for chordality. The algorithm LexBFS outputs a linear ordering of the vertices of $G$, but we need to test this ordering if it is a PES or not. This can, fortunately, also be achieved in linear time.

\begin{algorithm}[!ht]
	\caption{TestPES \faDog}
	\begin{algorithmic}[1]
		\Require A graph $G = (V, E)$ with $n$ vertices, a linear ordering $\sigma(i) \in V, i = 1, 2, \dots, n$ of its vertices.
		\Ensure \textbf{YES} or \textbf{NO} if $\sigma(n), \sigma(n-1), \dots, \sigma(1)$ is a PES.
		\State $A(x) := \emptyset$ for all $x \in V$
		\For{$i := 1 \dots n-1$}
			\State $v:= \sigma(i)$
			\State $X:= \{x : x \in N_G (v) \land \sigma^{-1}(v) < \sigma^{-1}(x)\}$
			\If{$X \neq \emptyset$}
				\State $u := \min \sigma^{-1}(X)$
				\State $A(u) := A(u) \cup (X \setminus \{u\})$
			\EndIf
			\If{$A(v) \setminus N_G (v) \neq \emptyset$}
				\State \Return \textbf{NO}
			\EndIf
		\EndFor
		\State \Return \textbf{YES} \faDog
	\end{algorithmic}
\end{algorithm}

\begin{thm}
	Algorithm TestPES correctly answers if a linear ordering $\sigma(n), \sigma(n-1), \dots, \sigma(1)$ is a PES for $G$ or not and it can be implemented in time linear in $m$.
\end{thm}

\begin{proof}
	For each vertex $u \in V$, the set $A(u)$ contains vertices that should be neighbors of $u$ if the ordering is a PES. The point is that it is not necessary to put in $A(u)$ all neighbors, and thus the running time can be significantly reduced.
	
	\begin{claim}
		If $\sigma(n), \sigma(n-1), \dots, \sigma(1)$ is not a PES, then at least one violation gets detected.
	\end{claim}
	
	\begin{proof}[Proof of claim]
		Let $i < j < h$ be a violation (i.e., $\sigma(i)\sigma(j), \sigma(i)\sigma(h) \in E$, $\sigma(j)\sigma(h) \notin E$) such that the difference $j - i$ is minimum possible. Let $v = \sigma(i)$. Then $\sigma(j), \sigma(h) \in X$ when $v$ is processed by the algorithm. Suppose there is a $j' , i < j' < j$ such that $\sigma(j') \in X$. If $\sigma(j')\sigma(j) \notin E$, $i < j' < j$ is a violation with smaller difference $j' - i < j - i$. If $\sigma(j')\sigma(h) \notin E$, $i < j' < h$ is a violation with smaller difference $j' - i < j - i$. If both $\sigma(j')\sigma(j)$ and $\sigma(j')\sigma(h)$ are edges, $j' < j < h$ is a violation with smaller difference $j - j' < j - i$. Since all possibilities lead to contradictions, we conclude that $j = \min\{l : \sigma(l) \in X\}$ and $\sigma(h)$ is put into $A(\sigma(j))$ when $v$ is being processed. Thus the non-edge $\sigma(j)\sigma(h)$ gets detected when $u = \sigma(j)$ is processed (unless the algorithm quits even sooner).
	\end{proof}
\end{proof}

\begin{rem}
	Note that the correctness of Algorithm LexBFS implies that not only one can leave any simplicial vertex as the last one in a PES, but it is also true that any vertex of a chordal graph can be used as the first vertex in a PES.
\end{rem}