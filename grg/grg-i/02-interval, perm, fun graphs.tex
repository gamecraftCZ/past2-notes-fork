\chapter{Interval, permutation and function graphs}

\section{Interval graphs}

\begin{defn}
	A graph is an \textbf{interval graph} if it is isomorphic to the intersection graph of a collection of intervals on a line.
\end{defn}

\begin{observ}
	Every interval graph has an interval representation in which all of the intervals are closed.
\end{observ}

\begin{defn}[Clique-path decomposition]
	A \textbf{clique-path decomposition} of a graph is a clique-tree decomposition in which the underlying tree is a path.
\end{defn}

\begin{thm}
	For any graph $G$, the following statements are equivalent:
	
	\begin{enumerate}
		\item $G$ is an interval graph,
		\item $G$ has a clique-path decomposition, and
		\item $G$ is an intersection graph of subpaths of a path.
	\end{enumerate}
\end{thm}

\begin{proof}
	"$1. \Leftrightarrow 3.$" is obvious.
	
	"$1. \Rightarrow 2.$" Assume $I_u , u \in V(G)$ is an interval representation of $G$. Use the fact that intervals on a line have the Helly property, i.e., if any two of a collection of intervals have a nonempty intersection, then all of them have a nonempty intersection. In other words, if $Q_i \in \mathcal{Q}$ is a maximal clique of $G$, then there exists a point $P_i$ which belongs to $\bigcap_{u \in Q_i} I_u$. Moreover, for every $v \notin Q_i , P_i \notin I_v$, since $Q_i$ is a maximal clique. (E.g., the rightmost of the left endpoints of the intervals $I_u , u \in Q_i$ is a good candidate for $P_i$.) Order the cliques of $Q$ as $Q_1, Q_2, \dots, Q_k$ so that $P_1 < P_2 < \dots < P_k$. Then the path
	$Q_1 Q_2 \dots Q_k$ is a clique-path decomposition of $G$.
	
	"$2. \Rightarrow 3.$" Given a clique-path decomposition $P = (\mathcal{Q}, F)$, define $P_u = P [\{Q : u \in Q \in \mathcal{Q}\}]$ for $u \in V(G)$. Clearly $V(P_u) \cap V(P_v) \neq \emptyset$ iff $u$ and $v$ belong to the same maximal clique of $G$, which happens if and only $u$ and $v$ are adjacent in $G$.
\end{proof}

\section{Comparability graphs}

\begin{defn}
	A graph $G$ is a \textbf{comparability graph} if there exists a partial order $P = (V(G), \leq)$ on the vertex set of $G$ (i.e., an \textbf{antireflexive}, \textbf{antisymmetric} and \textbf{transitive binary relation}) such that for any two vertices $u, v \in V(G)$, $uv \in E(G)$ if and only if $u \leq v$ or $v \leq u$ (i.e., if $u$ and $v$ are comparable in $P$). The class of comparability graphs will be denoted by CO.
\end{defn}

\begin{observ}
	A graph is a comparability graph if and only if its edges can be transitively oriented.
\end{observ}

\begin{thm}
	Comparability graphs are perfect.
\end{thm}

\begin{proof}
	Because $G \in CO$ then $\exists P = (V, \leq)$. We create a Haase diagram. Then we may see that cliques are chains and also independent sets are the antichains. Therefore we color every layer with one color and it is the same as the size of the largest clique.
\end{proof}

\begin{notation}
	If $\mathcal{A}$ is a graph class, the symbol $\text{co}-\mathcal{A}$ is used to denote the class containing the complements of the graphs in $\mathcal{A}$.
\end{notation}

\begin{observ}
	If $\mathcal{A} \subseteq \mathcal{B}$, then $\text{co}-\mathcal{A} \subseteq \text{co}-\mathcal{B}$.
\end{observ}

\begin{thm}
	All equivalencies hold:
	
	\begin{enumerate}
		\item $\text{FUN} = \text{co}-\text{CO}$
		\item $\text{PER} = \text{CO} \cap \text{co}-\text{CO}$
		\item $\text{INT} = \text{CHOR} \cap \text{co}-\text{CO}$
	\end{enumerate}
\end{thm}

\begin{proof}
	\begin{enumerate}
		\item ”FUN $\subseteq$ co-CO”: Given a collection of curves joining two vertical parallel lines (and lying in the stripe between them), for any two non-crossing curves, it is uniquely determined which one lies above the other one (this follows from the Jordan curve theorem), and this gives a transitive orientation of the complement of the intersection graph of this collection.
		
		”co-CO $\subseteq$ FUN”: Let $G = (V, E)$ be a graph and let $P = (V, \leq)$ be a partial order which corresponds to a transitive orientation of the complement of $G$. If $d$ is the dimension of $P$, $P$ is the intersection of $d$ linear orders $L_1, L_2 , \dots, L_d$ of $V$. In the plane, draw d distinct parallel (vertical) lines $l_1, l_2, \dots, l_d$, and on each $l_i$, mark distinct points $P_{iu}, u \in V$ bottom up in the order $L_i$. Consider piece-wise linear curves $c(u) = P_{1u} P_{2u} \dots P_{du}$, for $u \in V$. If $uv \in E$, $u$ and $v$ are incomparable in $P$,	and hence there are indices $i$ and $j$ such that $u <_{L_i} v$ and $v <_{L_j} u$, in other words $P_{iu}$ is below $P_{iv}$, while $P_{ju}$ is above $P_{jv}$. Hence the curves $c(u)$ and $c(v)$ cross somewhere between $l_i$ and $l_j$. If, on the other hand, $uv \notin E$, $uv$ is an edge of the complement of $G$ and hence $u$ and $v$ are comparable in $P$, say, $u \leq v$. But then $u <_{L_i} v$ for every $i = 1, 2, \dots, d$, and for each $i = 1, 2, \dots, d - 1$, the curve $c(u)$	lies below the curve $c(v)$ in the stripe between $l_i$ and $l_{i+1}$. Thus $c(u)$ and $c(v)$ are disjoint.
		
		\item Note first that co-PER $\subseteq$ PER. Indeed, given a permutation representation of a graph, swap	the order of the endpoints on one of the bounding lines to obtain a representation of the complement of the given graph. Then PER $=$ co-(co-PER) $\subseteq$ co-PER, and hence PER $=$ co-PER.
		
		”PER $\subseteq$ CO $\cap$ co-CO”: Obviously PER $\subseteq$ FUN $=$ co-CO. Then the above small observation implies PER $=$ co-PER $\subseteq$ co-(co-CO) $=$ CO as well.

		”CO $\cap$ co-CO $\subseteq$ PER”: Suppose both $G$ and its complement can be transitively oriented, say $\overrightarrow{E_1}$ be a transitive orientation of $G$ and $\overrightarrow{E_2}$ a transitive orientation of the complement $-G$ of $G$. Then $\overrightarrow{E_1} \cup \overrightarrow{E_2}$ is a transitive orientation of the complete graph $K_{V(G)}$ on the vertex set of $G$, i.e., a linear ordering of the vertices of $G$. And so is $\overrightarrow{E_1}^{-1} \cup \overrightarrow{E_2}$. Place the vertices of $G$ on two parallel lines, on one of them in the linear order given by $\overrightarrow{E_1} \cup \overrightarrow{E_2}$, on the other one in the order given by $\overrightarrow{E_1}^{-1} \cup \overrightarrow{E_2}$, and connect the two occurrences of a vertex $u$ by a straight-line segment called $s(u)$, for every vertex $u \in V(G)$. If $uv \in E(G)$, then the pair $u, v$ is ordered differently on the two lines (by $\overrightarrow{E_1}$ on one of them and by $\overrightarrow{E_1}^{-1}$ on the other one) and the segments $s(u), s(v)$ cross each other somewhere between	the two lines. If $uv \notin E(G)$, the pair $u, v$ is ordered the same way (by $\overrightarrow{E_2}$) on both of the lines, and thus the segments $s(u)$ and $s(v)$ are disjoint. So $\{s(u)\}_{u \in V(G)}$ is a permutation representation of $G$.
		
		\item ”INT $\subseteq$ CHOR $\cap$ co-CO”: Let $\{I(u)\}_{u \in V(G)}$ be an interval representation of a graph $G$. Define a transitive orientation $\overrightarrow{E_2}$ of the non-edges of $G$ by setting $uv \in \overrightarrow{E_2}$ if $\max I(u) < \min I(v)$. Thus $G \in$ co-CO. The fact that $G \in$ CHOR follows from the fact that
		
		$$
		\text{INT} = \mathcal{IG}(\{\text{connected subgraphs of paths}\}) \subseteq
		$$
		
		$$
		\subseteq \mathcal{IG}(\{\text{connected subgraphs of trees}\}) = \text{CHOR}.
		$$
		
		”CHOR $\cap$ co-CO $\subseteq$ INT”: Let $G$ be a chordal graph which allows a transitive orientation $\overrightarrow{E_2}$ of its non-edges. Define a binary relation $<$ on the set $\mathcal{Q}$ of maximal cliques of $G$ by setting
		
		$$
		Q < Q' \Leftrightarrow \exists u \in Q \exists v \in Q' : uv \in \overrightarrow{E_2}.
		$$
		
		\begin{claim}
			The relation $<$ is a partial order on $\mathcal{Q}$.
		\end{claim}
		
		\begin{proof}[Proof of claim]
			We will show all properties of partial order.
			\begin{itemize}
				\item Antireflexivity: Each $Q \in \mathcal{Q}$ is a clique, so there are no two vertices $u, v \in Q$ that would form a	non-edge of $G$. Hence $Q \not< Q$.
				
				\item Antisymmetry: Suppose for the contrary that there are $u \in Q, v \in Q'$ s.t. $uv \in \overrightarrow{E_2}$, and another pair $x \in Q, y \in Q'$ s.t. $yx \in \overrightarrow{E_2}$. First observe that $u \neq x$ and $v \neq y$ (if $u = x$, the transitivity of $\overrightarrow{E_2}$ would imply $yv \in \overrightarrow{E_2}$, which is impossible; if $v = y$, the transitivity of $\overrightarrow{E_2}$ would imply $ux \in \overrightarrow{E_2}$, which is again impossible). Next observe that both $uy$ and $xv$ must be edges of $G$ (if $uy \notin E(G)$, then either $uy \in \overrightarrow{E_2}$, or $yu \in \overrightarrow{E_2}$, yielding $ux \in \overrightarrow{E_2}$ in the former case and $yv \in \overrightarrow{E_2}$ in the letter one, both contradicting the fact that $Q$ and $Q'$ are cliques of $G$; the case of $xv \notin E(G)$ is analogous). Lastly, we conclude that $G[\{u, v, x, y\}] \simeq C_4$ , contradicting the assumption that $G$ is chordal.
				
				\item Transitivity: Suppose $Q < Q' < Q''$ and let $u \in Q, v, x \in Q'$ and $y \in Q''$ be vertices such that $uv, xy \in \overrightarrow{E_2}$. If $v = x$, the transitivity of $\overrightarrow{E_2}$ implies $uy \in \overrightarrow{E_2}$, hence $Q < Q''$. If $v \neq x$, one of $ux$, $vy$
				must be a non-edge (otherwise $G[\{u, v, x, y\}]$ would be an induced cycle of length 4, contradicting the assumption that $G$ is chordal). If $ux \notin E(G)$, $ux \in \overrightarrow{E_2}$ and the transitivity of $\overrightarrow{E_2}$ implies $uy \in \overrightarrow{E_2}$. If $vy \notin E(G)$, $vy \in \overrightarrow{E_2}$ and the transitivity of $\overrightarrow{E_2}$ implies $uy \in \overrightarrow{E_2}$. In either case, $Q < Q''$.
			\end{itemize}
		\end{proof}
		
		\begin{claim}
			The relation $<$ is a linear ordering of $\mathcal{Q}$.
		\end{claim}
		
		\begin{proof}[Proof of claim]
			Let $Q \neq Q'$ be two different maximal cliques of $G$. Their maximality implies that none of them is a subset of the other one. Hence there is a vertex $u \in Q$ which does not belong to $Q'$ . If $u$ were adjacent to all vertices of $Q'$, $Q' \cup \{u\}$ would be a clique of $G$, contradicting the maximality of $Q'$ . Hence there is a $v \in Q'$ such that $uv \notin E(G)$. Then either $uv \in \overrightarrow{E_2}$ or $vu \overrightarrow{E_2}$, thus $Q < Q'$ or $Q' < Q$.
		\end{proof}
		
		\begin{claim}
			Let $\mathcal{Q} = \{Q_1 < Q_2 < \dots < Q_k\}$ be the maximal cliques of $G$ ordered by $<$. Then $P_G = (\mathcal{Q}, \{Q_i Q_{i+1}: i = 1, 2, \dots, k - 1\})$ is a clique-path decomposition of $G$, and hence $G \in$ INT.
		\end{claim}
		
		\begin{proof}[Proof of claim]
			Indeed $P_G$ is a path whose nodes are the maximal cliques of $G$. It remains to show that vertices of $G$ appear in these cliques consecutively. Suppose $Q < Q' < Q''$ and $u \in Q \cap Q''$. If there were a vertex $v \in Q'$ nonadjacent to $u$, we would have $uv \in \overrightarrow{E_2}$ because of $Q < Q'$ and $vu \in \overrightarrow{E_2}$ because of $Q' < Q''$, contradicting the antisymmetry of $\overrightarrow{E_2}$. Hence $u$ is adjacent to all vertices of $Q'$ , and thus
			$u \in Q'$ follows from the maximality of $Q'$.
		\end{proof}
	\end{enumerate}
	
	The last claim proved the theorem as well.
\end{proof}