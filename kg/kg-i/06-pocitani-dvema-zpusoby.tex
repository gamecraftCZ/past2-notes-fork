\chapter{Počítání dvěma způsoby}

Metoda důkazů v kombinatorice. Určíme nějaký neznámý počet $X$ vyjádřením počtu $Z$ dvěma výrazy, z nichž jeden $X$ obsahuje a druhý ne $\Rightarrow$ máme vyjádření pro $X$.

\section{Cayleyho vzorec}

Kolika způsoby lze vytvořit strom na vrcholech $\{ 1, \dots , n \}$? Nebo-li jaká je počet koster $\kappa (n)$ grafu $K_{n}$? 

\begin{definice}
	Kostra grafu $G = (V, E)$ je strom $T = (V, E')$ s $E' \subseteq E$.
\end{definice}


\begin{veta}[Cayleyho vzorec]
	Pro každé $n \geq 1$ platí $\kappa (n) = n^{n-2}$.
\end{veta}

Existuje řada důkazů s velmi odlišnými myšlenkami, ukážeme si nejjednodušší založený na počítání dvěma způsoby.

\begin{proof}
	Empty.
\end{proof}

\begin{veta}
	Graf $K_{n}-e$ má $(n-2)n^{n-3}$ koster pro $n \geq 2$.
\end{veta}

\begin{proof}
	Empty.
\end{proof}


Počet koster $\kappa (G)$ grafu $G = (\{ 1, \dots , n \}, E)$ lze určit pomocí determinantu. Uvažme \textbf{Laplacián $L(G)$} grafu $G$, tedy matici $L(G) = (L_{ij})_{i,j=1}^{\infty}$, kde

$$
L_{ij} =
\left\{
\begin{array}{ll}
	\mathtt{deg}_{G}(i) \text{ pokud } i=j \\
	-1 \text{ pokud } (i,j) \in E \\
	0 \text{ jinak }
\end{array}
\right.
$$

\begin{veta}[Kirchhoffova věta]
	$\forall G : \kappa (G) = \det (L(G)^{1,1})$, kde $(L(G)^{1,1})$ je Laplacián $L(G)$ bez 1. řádků a 1. sloupce.
\end{veta}

\section{Spernerova věta}

\begin{definice}
	Systém $\mathcal{M} \subseteq 2^{ \{ 1, \dots , n \} }$ podmnožin $n$-prvkové množiny $\{ 1, \dots , n \}$ je \textbf{nezá-\newline vislý}, pokud platí: $\forall A, B \in \mathcal{M}, A \neq B: A \nsubseteq B \land A \nsupseteq B$.
\end{definice}

\begin{veta}[Spernerova věta]
	Každý nezávislý systém v $2^{ \{ 1, \dots , n \} }$ obsahuje $\leq \binom{n}{\lceil \frac{n}{2}\rceil}$ množin a tento odhad je těsný. Ekvivalentně: Nejdelší antiřetězec v $(2^{ \{ 1, \dots , n \} }, \leq)$ má právě $\binom{n}{\lceil \frac{n}{2}\rceil}$ prvků.
\end{veta}

\begin{proof}
	Empty.
\end{proof}
