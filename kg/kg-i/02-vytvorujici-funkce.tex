\chapter{Vytvořující funkce}

Početní metoda, kde spojitými funkcemi vyjadřujeme posloupnosti.

\begin{definice}
	Pro posloupnost $(a_{i})_{i=0}^{\infty}$ je mocninnou řadou $a(x) = \sum_{i=0}^{\infty}a_{i}x^{i}$.
\end{definice}

Posloupnost lze převést na funkci, ale při převodu nazpátek je třeba aby posloupnost nerostal moc rychle.

\begin{prikl}
	\begin{enumerate}
		\item $a_i = 1$ pokud $0 \leq i \leq n$ jinak $a_i = 0$. Tedy $(a_i)_{i=0}^{\infty}=(1,\dots,1,0,0,\dots)$. Potom funkce je $\frac{1-x^{n+1}}{1-x}$ z geometrické řady.
		\item $\forall i: a_i = 1$ to je potom nekonečná geometrická řada, takže je to $\frac{1}{1-x}$.
		\item $a_i = \binom{n}{i}$ potom funkce vychází z binomické věty, takže je $(1+x)^{n}$.
	\end{enumerate}
\end{prikl}

\begin{tvrz}
	Pokud pro $(a_i)_{i=0}^{\infty} \exists k \in \mathbb(R)$ takové, že $\forall i: |a_i| \leq k^i$, pak pro všechna $x \in (\frac{-1}{k}, \frac{1}{k})$ řada $a(x) = \sum_{i=0}^{\infty}a_ix^i$ konverguje absolutně a na libovolném $\epsilon$ okolí 0 určuje i $koeficienty_a$, protože $i_a = \frac{a^{(i)}(0)}{i!}$.
\end{tvrz}


Obecný postup:

\begin{enumerate}
	\item kombinatorický objekt s neznámým počtem
	\item vytvořující funkce
	\item rozklad na vytvořující funkce se známými koeficienty
	\item určení hodnoty
\end{enumerate}


\begin{table}[!h]\centering
	\begin{tabular}{| l | l | l |}
		\hline
		Operace & Posloupnosti & Funkce \\
		\hline
		Součet                   & $(a_0+b_0, a_1+b_1, \dots)$                        & $(a+b)(x) = \sum_{i=0}^{\infty}(a_i+b_i)x^i$ \\
		$\alpha$-násobek         & $(\alpha a_0, \alpha a_1, \dots)$                  & $\alpha a(x) = \sum_{i=0}^{\infty}(\alpha a_i)x^i$ \\
		Posun vpravo o $n$ pozic & $(0,0,\dots ,a_0, a_1, \dots)$                     & $x^na(x)\sum_{i=0}^{\infty}a_ix^{i+n}$ \\
		Posun vlevo o $n$ pozic  & $(a_n, a_{n+1}, dots )$                            & $\sum_{i=0}^{\infty}(a_{i+n}x^i = \frac{a(x) - \sum_{i=0}^{\infty}a_ix^i}{x^n}$ \\
		Dosazení $\alpha x$      & $(a_0, \alpha a_1, \alpha^2 a_2, \dots)$           & $a(\alpha x) = \sum_{i=0}^{\infty}a_i+ \alpha^i x^i$ \\
		Dosazení $x^n$           & $(a_0, 0, \dots, 0, a_1, 0, \dots)$ & $a(x^n) = \sum_{i=0}^{\infty}a_ix^ni$ \\
		Derivace                 & $(a_1, 2a_2, 3a_3, \dots)$                         & $a'(x) = \sum_{i=1}^{\infty}ia_ix^{i-1}$ \\
		Integrál                 & $(0, a_0, \frac{a_1}{2}, \frac{a_2}{3}, \dots)$    & $\int_{0}^{x} a(x) \mathtt{d} x = \sum_{i=0}^{\infty}\frac{a_i}{i+1}x^{i+1}$ \\
		Součin                   & $(c_0, c_1, c_2, \dots)$                           & $a(x)b(x) = \sum_{i=0}^{\infty}(c_i)x^i$ \\
		\hline
	\end{tabular}
\end{table}

\section{Řešení rekurentních rovnic}

Ukázka na Fibonacciho čísle. Určíme $F_n$ jako koeficient funkce $F(x)=\sum_{i=0}^{\infty}F_ix^i$. Máme $F_{n+2} = F_{n+1} + F_{n}, \forall n\geq 0$. \textbf{Vynásobíme rovnici $x^n$:} $F_{n+2} x^n= F_{n+1} x^n + F_{n} x^n$.\textbf{Sčítáme přes $n \geq 0$:} $\sum_{n \geq 0} F_{n+2} x^n= \sum_{n \geq 0} F_{n+1} x^n + \sum_{n \geq 0} F_{n} x^n$ to se rovná:

$$
\frac{F(x)-F_0 - F_1}{x^2} = \frac{F(x)-F_0}{x} +F(x)
$$

teď určíme

$$
F(x) = \frac{x}{1-x-x^2} = \frac{x}{(1 - \frac{1 + \sqrt{5}}{2}x)(1 - \frac{1 - \sqrt{5}}{2}} = \frac{\frac{1}{\sqrt{5}}}{1 - \frac{1 + \sqrt{5}}{2}x} - \frac{\frac{1}{\sqrt{5}}}{1 - \frac{1 - \sqrt{5}}{2}x}
$$

to už lze převést

$$
F_n = \frac{1}{\sqrt{5}}\left[ \left( \frac{1+\sqrt{5}}{2} \right)^n - \left( \frac{1-\sqrt{5}}{2} \right)^n \right]
$$

to je tzv. \textbf{Binetův vzorec}. Tento postup funguje pro všechny \textbf{Homogenní lineární rekurence $k$-tého stupně s konstantními koeficienty}, tedy typu:

$$
a_{n+k} = \alpha_{k-1}a_{n+k-1} + \dots + \alpha_0 a_n, k \in \mathbb{N}, \alpha_{k-1}, \dots, \alpha_0 \in \mathbb{R}
$$

\section{Aplikace vytvořujících funkcí}

Nadefinujeme zobecněnou \textbf{binomickou větu} pro

$$
n \in \mathbb{R}, r \in \mathbb{Z}_{0}^{+}: \binom{n}{r}:= \frac{n(n-1)(n-2)\dots(n-r+1)}{r!},
$$

speciálně $\binom{n}{0}=1$.

\begin{veta}[Zobecněná binomická věta]
	$\forall r \in \mathbb{R}$ je $(1+x)^r$ vytvořující funkcí posloupnosti $\left( \binom{r}{0}, \binom{r}{1}, \binom{r}{2}, \dots \right)$ a řada $\sum_{i=0}^{\infty}\binom{r}{i}x^{i}$ konverguje pro $\forall x \in (-1, 1)$.	
\end{veta}

\begin{dukaz}
	Empty.
\end{dukaz}

\begin{dusl}
	$\forall n \in \mathbb{N} \ \forall x \in (-1,1)$ platí $\frac{1}{(1-x)^{n}}=\sum_{i=0}^{\infty}\binom{n+i-1}{n-1} x^{i}$.
\end{dusl}

\begin{dukaz}
	Empty.
\end{dukaz}

\section{Aplikace - Catalanova čísla}

\textbf{Zakořeněný binární strom} - buď je prázdný, nebo obsahuje speciální vrchol zvaný \textbf{kořen} a pár zakořeněných stromů, které tvoří \textbf{levý} a \textbf{pravý podstrom}.

$b_{n}$ = počet bin. zak. stromů na $n \in \mathbb{N}_{0}$ vrcholech, potom $b(x) = \sum_{n = 0}^{\infty}b_{n} x^{n}$ je příslušná vytvořující funkce.

\begin{veta}
	Pro každé $n \in \mathbb{N}_{0}$ platí $b_{n}=\frac{1}{n+1}\binom{2n}{n}$. Kde se $\frac{1}{n+1}\binom{2n}{n}$ značí jako $C_n$ a říká se mu n-té \textbf{Catalanovo číslo}.
\end{veta}

\begin{dukaz}
	Empty.
\end{dukaz}

Catalanova čísla mají mnoho interpretací, například počet triviálních uzávorkování s $n$ páry závorek anebo počet triangulací.